%\subsection{Data plots, model plots, and data+model plots}

In this book, we use hundreds of graphs to illustrate aspects of discrete data,
methods of analysis, and plots for understanding and explaining results.
In addition to the overview of goals and design principles (\secref{sec:intro-goals})
shown in \figref{fig:datadisp}, another classification of such graphs is
useful to bear in mind as you read this book.  We distinguish three kinds of plots:

\begin{itemize}
\ix{data plots}
\ix{graphics!data plots}
    \item \textbf{Data plots}: these are well-known. They help answer questions like:
    \begin{seriate}
      \item What do the data look like?
      \item Are there unusual features?
      \item What kinds of summaries would be useful?
    \end{seriate}

    An immediate (but bad) example is the plot of failures of O-rings against temperature
    in \figref{fig:nasa0}. Many other examples appear throughout \chref{ch:discrete},
    using barplots for discrete distributions, and \chref{ch:twoway}, using various graphic forms
    to display frequencies in two-way tables.

    \item \textbf{Model plots}: these are less well-known as such, but also help answer important questions:
\ix{model plots}
\ix{graphics!model plots}
    \begin{seriate}
      \item What does the model ``look'' like? (plot predicted values);
      \item How does the model change when its \emph{parameters} change? (plot competing models);
      \item How does the model change when the \emph{data} is changed? (e.g., influence plots).
    \end{seriate}

    Models are simplified descriptions of data.  In \secref{sec:mosaic-struc} we use mosaic
    plots to show what \loglin models ``look like'' (e.g., \figref{fig:struc-mos3}).
    Plots for \ca methods (\chref{ch:corresp}) show the relationships among table variables
    as fitted points in a two-dimensional space.

    Effect plots (\secref{sec:logist-effplots}) show fitted values for logistic regression models.
\ix{effect plot}
\ix{graphics!effect plots}
    Models for ordinal variables in terms of log odds ratios (\secref{sec:loglin-visord})
    can also be illustrated in terms of simple models plots (\figref{fig:mental-lodds-plots}).
%    \secref{sec:loglin-visord} shows some simple plots (\figref{fig:mental-lodds-plots})
%    to illustrate the differences among models for ordinal variables in terms of log odds ratios.

    \item \textbf{Data + Model plots} combine these features, and lead to other questions:
\ix{data+model plots}
\ix{graphics!data+model plots}
    \begin{seriate}
      \item How well does a model fit the data?
      \item Does a model fit uniformly good or bad, or just good/bad in some regions?
      \item How can a model be improved?
      \item Model \emph{uncertainty}: show confidence/prediction intervals or regions.
      \item Data \emph{support}: where is data too ``thin'' to make a difference in competing models?
    \end{seriate}

    \figref{fig:spaceshuttle0} and  \figref{fig:donner0}, \figref{fig:donner0-other}
    show several data+model plots for the space shuttle and Donner data respectively,
    both showing confidence bands for predicted values.
    \figref{fig:nasa} is an another example, comparing two models for the space shuttle data.
    The model-building methods described in \chref{ch:logistic}--\chref{ch:glm} make frequent
    use of data+model plots.
\end{itemize}
