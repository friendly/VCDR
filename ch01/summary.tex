%\Section{Chapter summary}

\begin{itemize}

  \item Categorical data differs from quantitative data because the variables take on
  discrete values (ordered or unordered, character or numeric)
  rather than continuous numerical values. Consequently,
  such data often appear in aggregated form representing category frequencies or in tables.

  \item Data analysis methods for categorical data are comprised of those concerned mainly
  with testing particular hypotheses versus those that fit statistical models.
  Model building methods have the advantages of providing parameter estimates and
  model-predicted values, along with measures of uncertainty (standard errors).

  \item Graphical methods can serve different purposes for different goals
  (data analysis versus presentation), and these suggest different design
  principles that a graphic should respect to achieve a given communication goal.

  \item For categorical data, some graphic forms (bar charts, line graphs,
  scatterplots) used for quantitative data can be readily adapted to
  discrete variables.
  However, frequency data often requires novel graphics using area and other
  visual attributes.

  \item Graphics can be far more effective when categorical variables are ordered
  to facilitate comparison of the effects to be seen and rendered to facilitate
  detection of patterns, trends or anomalies.

  \item The visualization approach to data analysis often entails a sequence of
  intertwined steps  involving graphing and model fitting.

  \item Producing effective graphs for presentation is often hard work, requiring
  attention to details that support or detract from your communication goal.
\end{itemize}
