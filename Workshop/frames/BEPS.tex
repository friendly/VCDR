
\subsection{A larger example}
\begin{frame}
 \frametitle{Political knowledge \& party choice in Britain}
Example from \cite{FoxAndersen:2006}: Data from 1997 British Election Panel Survey (BEPS)
\begin{itemize}
 \item {\bfseries Response}: Party choice--- Liberal democrat, Labour, Conservative
 \item {\bfseries Predictors}
	\begin{itemize*}
	 \item Europe: 11-point scale of attitude toward European integration (high=``Eurosceptic'')
	 \item Political knowledge: knowledge of party platforms on European integration 
           (``low''=0--3=``high'')
     \item Others: Age, Gender, perception of economic conditions, evaluation of party leaders
           (Blair, Hague, Kennedy)-- 1:5 scale
	\end{itemize*}
 \item {\bfseries Model}:
	\begin{itemize*}
	 \item Main effects of Age, Gender, economic conditions (national, household)
	 \item Main effects of evaluation of party leaders
     \item Interaction of attitude toward European integration with political knowledge
	\end{itemize*}

\end{itemize}
\end{frame}

\begin{frame}[fragile]
 \frametitle{BEPS data: Fitting}
In R, generalized (multinomial) response models are fit using \func{multinom} in the \pkg{nnet}
\begin{Rin}[baselinestretch=0.8, fontsize=\footnotesize]
library(effects)  # data, plots
library(car)      # for Anova()
library(nnet)     # for multinom()
multinom.mod <- multinom(vote ~ age + gender + economic.cond.national +
    economic.cond.household + Blair + Hague + Kennedy +
    Europe*political.knowledge, data=BEPS)
Anova(multinom.mod)
\end{Rin}
\begin{Rout}[baselinestretch=0.8, fontsize=\footnotesize]
Anova Table (Type II tests)

Response: vote
                           LR Chisq Df Pr(>Chisq)    
age                            13.9  2    0.00097 ***
gender                          0.5  2    0.79726    
economic.cond.national         30.6  2    2.3e-07 ***
economic.cond.household         5.7  2    0.05926 .  
Blair                         135.4  2    < 2e-16 ***
Hague                         166.8  2    < 2e-16 ***
Kennedy                        68.9  2    1.1e-15 ***
Europe                         78.0  2    < 2e-16 ***
political.knowledge            55.6  2    8.6e-13 ***
Europe:political.knowledge     50.8  2    9.3e-12 ***
---
Signif. codes:  0 '***' 0.001 '**' 0.01 '*' 0.05 '.' 0.1 ' ' 1 
\end{Rout}
 
\end{frame}

\begin{frame}[fragile]
 \frametitle{BEPS data: Interpretation?}
How to understand the \emph{nature} of these effects on party choice?
\begin{Rin}
> summary(multinom.mod)
\end{Rin}
\begin{Rout}[baselinestretch=0.8, fontsize=\footnotesize]
Call:
multinom(formula = vote ~ age + gender + economic.cond.national + 
    economic.cond.household + Blair + Hague + Kennedy + Europe * 
    political.knowledge, data = BEPS)

Coefficients:
                 (Intercept)      age gendermale economic.cond.national
Labour               -0.8734 -0.01980     0.1126                 0.5220
Liberal Democrat     -0.7185 -0.01460     0.0914                 0.1451
                 economic.cond.household  Blair   Hague Kennedy    Europe
Labour                          0.178632 0.8236 -0.8684  0.2396 -0.001706
Liberal Democrat                0.007725 0.2779 -0.7808  0.6557  0.068412
                 political.knowledge Europe:political.knowledge
Labour                        0.6583                    -0.1589
Liberal Democrat              1.1602                    -0.1829

Std. Errors:
                 (Intercept)      age gendermale economic.cond.national
Labour                0.6908 0.005364     0.1694                 0.1065
Liberal Democrat      0.7344 0.005643     0.1780                 0.1100
...

Residual Deviance: 2233 
AIC: 2277 
\end{Rout}
	
\end{frame}


\begin{frame}
 \frametitle{BEPS data: Initial look, relative multiple barcharts}
 \begin{center}
 	 \includegraphics[height=.8\textheight]{fig/BEPS-rmb1}
 \end{center}
\end{frame}

\begin{frame}
 \frametitle{BEPS data: Effect plots to the rescue!}
 Age effect: Older more likely to vote Conservative
% \begin{Rin}
% 
% \end{Rin}

 \begin{center}
 	 \includegraphics[height=.8\textheight]{fig/BEPS2}
 \end{center}
\end{frame}

\begin{frame}
 \frametitle{BEPS data: Effect plots to the rescue!}
 Attitude toward European integration $\times$ political knowledge effect:
 \begin{center}
 	 \includegraphics[width=\textwidth]{fig/BEPS1}
 \end{center}
 \begin{itemize*}
  \item Low political knowledge: little relation between attitude and political choice
  \item As knowledge increases: more Eurosceptic views more likely to support Conservatives 
  \item $\Rightarrow$ detailed understanding of complex models depends strongly on 
   visualization!
 \end{itemize*}

\end{frame}

\begin{frame}<none>
 \frametitle{BEPS data: Effect plots to the rescue!}
 Attitude toward European integration $\times$ political knowledge effect:
 \begin{center}
 	 \includegraphics[width=.5\textwidth]{fig/BEPS1a}
 \end{center}
\end{frame}
