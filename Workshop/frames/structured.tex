\renewcommand{\FileName}{structured}
% slide template
\begin{frame}
  \frametitle{More structured tables}
  \begin{block}<1->{\large\bfseries Ordered categories}
      Tables with ordered categories may allow more \alert{parsimonious} tests of association
      \begin{itemize*}
	  \item Can represent $\lambda_{ij}^{AB}$ by a small number of parameters
	  \item $\rightarrow$ more focused and \emph{more powerful} tests of lack of independence (recall: CMH tests)
          \item Allow one to ``explain'' the \alert{pattern} of association in a compact way.
	  \end{itemize*}
  \end{block}

   \begin{block}<2->{\large\bfseries Square tables}
	For square $I \times I$ tables, where row and column variables have the
		same categories:
      \begin{itemize*}
	  	\item Can ignore diagonal cells, where association is expected and test
		remaining association (\emph{quasi-independence})
		\item Can  test whether association is \emph{symmetric} around the diagonal cells.
		\item Can test \alert{substantively important} hypotheses (e.g., mobility tables)
      \end{itemize*}
  \end{block}
  \uncover<2->{All of these require the GLM approach for model fitting}
\end{frame}

% slide template
\begin{frame}[allowframebreaks]
  \frametitle{Ordered categories}
  \begin{itemize}
	\item{\large\bfseries Ordinal scores}
      \begin{itemize*}
	  \item In many cases it may be reasonable to assign numeric scores, $\{a_i\}$
	  to an ordinal row variable and/or numeric scores, $\{b_i\}$ to an ordinal column variable.
	  \item Typically, scores are equally spaced and sum to zero, 
		$\{ a_i \} = i - (I+1)/2$, e.g., $\{ a_i \} = \{ -1, 0, 1 \}$ for I=3.
      \end{itemize*}
	  
	\item{\large\bfseries Linear-by-Linear (Uniform) Association}:
	  When \emph{both} variables are ordinal, the simplest model posits that any association
	  is \emph{linear} in both variables.
	  \[ \lambda_{ij}^{AB} = \gamma \: a_i b_j \]
    	\begin{itemize*}
		\item Only adds \alert{one additional parameter} to the independence model ($\gamma=0$).
		\item It is similar to CMH test for linear association
		\item For integer scores, the local log odds ratios for \emph{any} contiguous 2 $\times$ 2 table
		are all equal,
		\(  \log  \theta_{ij} = \gamma \)
%		\item $\gamma$ = log odds ratio for \emph{any} contiguous 2 $\times$ 2 table.
		\item This is a model of \emph{uniform association} --- simple interpretation!
		\end{itemize*}
%\end{frame}
\framebreak
For a two way table, there are 4 possibilities, depending on which variables
are ordinal, and assigned scores:

\begin{center}
 \includegraphics[width=.85\textwidth]{fig/ordered-categories}
\end{center}

\framebreak
%\begin{frame}	  
	\item{\large\bfseries Row Effects and Column Effects}:
	  When only one variable is assigned scores, we have the \emph{row effects model}
	  or the \emph{column effects model}.
	  	\begin{itemize}
		\item E.g., in the row effects model, the row variable ($A$) is treated as nominal,
	  while the column variable ($B$) is assigned ordered scores $\{ b_j \}$.
	  \[ \log  m_{ij} = \mu + \lambda_i^A + \lambda_j^B  + \alpha_i b_j \]
	  where the row parameters, $\alpha_i$, are defined so they sum to zero.
	  	\item This model has $(I-1)$ more parameters than the independence model.
		\item A Row Effects + Column Effects model allows both variables to be ordered,
		but not necessarily with linear scores.
		\end{itemize} 
%	  \end{itemize*}
	\item{\large\bfseries Fitting models for ordinal variables}
	  	\begin{itemize*}
		\item Create \emph{numeric} variables for category scores
		\item \PROC{GENMOD}: Use as quantitative variables 
		in \texttt{MODEL} statement,
		but \emph{not} listed as \texttt{CLASS} variables
		\item R: Create numeric variables with \texttt{as.numeric(factor)}
	  	\end{itemize*}
	
  \end{itemize}
\end{frame}

\begin{frame}[fragile]
  \frametitle{Ordered categories: RC models}
  \begin{itemize}
    \item{\large\bfseries RC(1) model}:  Generalizes the uniform association, R, C and
	R+C models by relaxing the assumption of specified order and spacing.
	  \[RC(1): \log  m_{ij} = \mu + \lambda_i^A + \lambda_j^B  + \phi \mu_i \nu_j \]
	  	\begin{itemize*}
			\item The row parameters ($\mu_i$)  and column parameters ($\nu_j$) are estimated from the
			data. 
			\item $\phi$ is the measure of association, similar to $\gamma$ in the uniform association model 
	  	\end{itemize*}
    \item{\large\bfseries RC(2) \dots RC(M) models}:  Allow two (or more) log-multiplicative association terms; e.g.:
	  \[RC(2): \log  m_{ij} = \mu + \lambda_i^A + \lambda_j^B  + \phi_1 \mu_{i1} \nu_{j1} + \phi_2 \mu_{i2} \nu_{j2} \]
	  Related to CA, but provide hypothesis tests, std. errors, etc.
	
	
	\item{\large\bfseries Fitting RC models}
	  	\begin{itemize*}
			\item SAS: no implementation
			\item R: Fit with \verb|gnm(Freq ~ R + C + Mult(R, C))|
	  	\end{itemize*}
	
  \end{itemize}
  
\end{frame}

\begin{frame}
  \frametitle{Relations among models}
 \begin{minipage}[c]{.6\linewidth}
  \includegraphics[width=1\linewidth]{fig/Wong-Fig2-1c}
 \end{minipage}%
 \hfill
 \begin{minipage}[c]{.39\linewidth}
 	\begin{itemize}
		\item Structured models: different ways to account for association
		\item Ordered by: df (\# of parameters)
		\item Arrows show nested models (compare directly: $\Delta \chi^2$)
		\item All can be compared using AIC (or BIC)
	\end{itemize}

 \end{minipage}
\end{frame}


\begin{frame}
  \frametitle{Example: Mental impairment and parents' SES}
  \begin{itemize}
	\item \cite{Srole-etal:78} Data on mental health status of $\sim$1600
	young NYC residents in relation to parents' SES.
      \begin{itemize*}
	  \item Mental health:  Well, mild symptoms, moderate symptoms, Impaired
	  \item SES:  1 (High) -- 6 (Low)
	  \end{itemize*}
%	\input{tab/mental-data}
	\input{tab/mental-data2}
  \end{itemize}
\end{frame}
%\framebreak

\begin{frame}
Before fitting models, it is often useful to explore the relation amongs the row/column
categories.  Correspondence analysis is a good idea!

%\begin{center}
%  \includegraphics[height=.7\textheight]{fig/mental-ca}
%\end{center}

 \begin{minipage}[c]{.6\linewidth}
  \includegraphics[width=1\linewidth]{fig/mental-ca}
 \end{minipage}%
 \hfill
 \begin{minipage}[c]{.39\linewidth}
 	\begin{itemize}
		\item Essentially 1D
		\item Both variables are ordered
		\item High SES goes with better mental health status
		\item Can we treat either or both as equally-spaced?
		\item GLM approach allows testing/comparing hypotheses vs. eye-balling
		\item Parameter estimates quantify effects.
	\end{itemize}

 \end{minipage}
\end{frame}

\begin{frame} 
\textbf{Visual assessment} of various loglin/GLM models: mosaic displays
 \begin{minipage}[t]{.49\linewidth}
  \includegraphics[width=1\linewidth]{fig/mentgen21}
 \end{minipage}%
 \hfill
 \begin{minipage}[t]{.49\linewidth}
  \includegraphics[width=1\linewidth]{fig/mentgen23}
 \end{minipage}

\begin{itemize*}
\item Residuals from the independence model show an opposite-corner pattern.
This is consistent with both:
	\begin{itemize*}
	\item Linear $\times$ linear model: equi-spaced scores for both Mental and SES
	\item Row effects model: equi-spaced scores for SES, ordered scores for Mental
	\end{itemize*} 
\end{itemize*} 
\end{frame}

%\framebreak
\begin{frame}
\textbf{Statistical assesment:}
\begin{table}[htb]
 \caption{Cumulative logit models for Mental Health data}\label{tab:mentab3}
 \begin{center}
 \begin{tabular}{c l rrr r}
  \hline
  Model & Terms        & df & \chisq & $p$-value & AIC\\ 
  \hline
  0 & \verb\_R_\       & 15 & 45.92 & 0.0001 &  15.92\\ 
  1 & \verb\_R_| SES\  & 10 &  7.75 & 0.6536 & -12.25\\ 
  2 & \verb\_R_ S\     & 14 & 10.72 & 0.7080 & -17.28\\ 
  3 & \verb\_R_|S\     & 12 &  6.48 & 0.8897 & -17.52\\ 
  4 & \verb\_R_ S S^2\ & 13 &  8.36 & 0.8192 & -17.64\\ 
  5 & \verb\_R_|S S^2\ & 11 &  3.94 & 0.9716 & -18.06\\ 
  \hline
 \end{tabular}
 \end{center}
\end{table}

\begin{itemize}
\item Both the Row Effects and Linear $\times$ linear models are significantly better
than the Independence model
\item AIC indicates a slight preference for the Linear $\times$ linear model
\item In the Linear $\times$ linear model, the estimate of the coefficient
of $ a_i b_j$ is $\hat{\gamma}=0.0907 = \widehat{\log \theta}$, so $\hat{\theta}=\exp(0.0907) = 1.095$.
\item \implies each step down the SES scale increases the odds of being classified
one step \emph{poorer} in mental health by 9.5\%.
\item Compare with purely exploratory (CA) interpretation: mental health increases with SES
\end{itemize} 
\end{frame}

%\framebreak
\begin{frame}[fragile]
Fitting these models with \PROC{GENMOD}:
\begin{Input}[fontsize=\small,label=\fbox{\texttt{mentgen2.sas}},baselinestretch=0.6]
%include catdata(mental);
data mental;
   set mental;
   m_lin = mental;    \sascomment{*-- copy m_lin and s_lin for;}
   s_lin = ses;       \sascomment{*-- use non-CLASS variables;}

title 'Independence model';
proc genmod data=mental;
   class mental ses;
   model count = mental ses / dist=poisson obstats residuals;
   format mental mental. ses ses.;
   ods output obstats=obstats;
%mosaic(data=obstats, vorder=Mental SES, resid=stresdev, 
	title=Mental Impairment and SES: Independence, split=H V);
\end{Input}
Row Effects model:
\begin{Input}[fontsize=\small,label=\fbox{\texttt{mentgen2.sas}},baselinestretch=0.7,firstnumber=16]
proc genmod data=mental;
   class mental ses;
   model count = mental ses \sasemph{mental*s_lin} / dist=poisson obstats;
   ...
\end{Input}
Linear $\times$ linear model:
\begin{Input}[fontsize=\small,label=\fbox{\texttt{mentgen2.sas}},baselinestretch=0.7,firstnumber=21]
proc genmod data=mental;
   class mental ses;
   model count = mental ses \sasemph{m_lin*s_lin} / dist=poisson obstats;
\end{Input}

\end{frame}

%\framebreak
\begin{frame}[fragile]
Fitting these models with glm() in R (see: mental-glm.R for plots)
\begin{Rin}[baselinestretch=0.85]
library(vcdExtra)
data(Mental)
# Integer scores for rows/cols 
Cscore <- as.numeric(Mental$ses)
Rscore <- as.numeric(Mental$mental)

indep <- glm(Freq ~ mental+ses, family = poisson, data=Mental)

# column effects model (ses)
coleff <- glm(Freq ~ mental + ses + Rscore:ses,
                family = poisson, data = Mental)

# row effects model (mental)
roweff <- glm(Freq ~ mental + ses + mental:Cscore,
                family = poisson, data = Mental)

# linear x linear association
linlin <- glm(Freq ~ mental + ses + Rscore:Cscore,
                family = poisson, data = Mental)

# compare models
AIC(indep, coleff, roweff, linlin)
\end{Rin}
%AIC statistics, comparing models:
%\begin{Rout}
%       df      AIC
%indep   9 209.5908
%roweff 12 174.4537
%coleff 14 179.0023
%linlin 10 174.0681
%\end{Rout}

\end{frame}


\endinput

\input{tab/vision-summary}

Independence
\begin{center}
  \includegraphics[width=.6\dispwidth,clip]{fig/mosaic10g1}
\end{center}

Linear x linear (uniform association)
\begin{center}
  \includegraphics[width=.6\dispwidth,clip]{fig/mosaic10g6}
\end{center}

Row + Col+
\begin{center}
  \includegraphics[width=.6\dispwidth,clip]{fig/mosaic10g7}
\end{center}

