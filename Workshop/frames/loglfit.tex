\renewcommand{\FileName}{loglfit}
\begin{frame}[fragile]
 \frametitle{Fitting \loglin\ models: SAS}
 \begin{block}{SAS}
  \begin{itemize}
   \item \PROC{CATMOD}
\begin{Input}[baselinestretch=0.8,fontsize=\footnotesize]
%include catdata(berkeley);
proc catmod order=data data=berkeley;
   format dept dept. admit admit.;
   \sasemph{weight freq;}                  \sascomment{/* data in freq. form */}
   model dept*gender*admit=_response_ ;
   \sasemph{loglin} admit|dept|gender @2   / title='Model (AD,AG,DG)'; run;
   \sasemph{loglin} admit|dept dept|gender / title='Model (AD,DG)';  run;
\end{Input}

   \item \PROC{GENMOD}
\begin{Input}[baselinestretch=0.8,fontsize=\footnotesize]
proc genmod data=berkeley;
   class dept gender admit;
   model freq = dept|gender dept|admit / \sasemph{dist=poisson};
run;
\end{Input}

  \end{itemize}
 \end{block}
\begin{itemize*}
  \item \texttt{mosaic} macro usually fits loglin models internally and
  displays results
  \item You can also use \PROC{GENMOD} for a more general model, and
  display the result with the \texttt{mosaic} macro.
\end{itemize*}
  
\end{frame}

\begin{frame}[fragile]
 \frametitle{Fitting \loglin\ models: R}
 \begin{block}{R}
  \begin{itemize}
   \item \func{loglm} - data in contingency table form (\pkg{MASS})
\begin{Input}[baselinestretch=0.8,fontsize=\footnotesize]
data(UCBAdmissions)
  ## conditional independence (AD, DG) in Berkeley data
mod.1 <- loglm(~ (Admit + Gender) * Dept, data=UCBAdmissions)
  ## all two-way model (AD, DG, AG) 
mod.2 <- loglm(~ (Admit + Gender + Dept)^2, data=UCBAdmissions)
\end{Input}

   \item \func{glm} - data in frequency form
\begin{Input}[baselinestretch=0.8,fontsize=\footnotesize]
berkeley <- as.data.frame(UCBAdmissions)
mod.3 <- glm(Freq ~ (Admit + Gender) * Dept, data=berkeley, 
                   \sasemph{family='poisson'})
\end{Input}

  \end{itemize}
 \end{block}

\begin{itemize*}
  \item \func{loglm} simpler for nominal variables
  \item \func{glm} allows a wider class of models
  \item \func{gnm} fits models for structured association and
  generalized \emph{non-linear} models
  \item \texttt{vcdExtra} package provides visualizations for all.
  
\end{itemize*}

\end{frame}

