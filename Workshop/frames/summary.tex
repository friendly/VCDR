\renewcommand{\FileName}{summary}
% slide template
\subsection{Summary: Part 5}
\begin{frame}
   \frametitle{Summary: Part 5}
  \begin{itemize}
	\item<+->{\large\bfseries Polytomous responses}
      \begin{itemize*}
	  \item $m$ response categories $\rightarrow$ $m-1$ comparisons (logits)
	  \item Different models for ordered vs.\ unordered categories
	  \end{itemize*}
	\item<+->{\large\bfseries Proportional odds model}
      \begin{itemize*}
	  \item Simplest approach for \emph{ordered} categories: Same slopes for all logits
	  \item Requires proportional odds asumption to be met
	  \item SAS: \PROC{LOGISTIC}; R: \func{polr}
	  \end{itemize*}
	\item<+->{\large\bfseries Nested dichotomies}
      \begin{itemize*}
	  \item Applies to ordered or unordered categories
	  \item Fit $m-1$ independent models $\rightarrow$ Additive $\chi^2$ values
	  \item SAS: \PROC{LOGISTIC}; R: \func{glm}
	  \end{itemize*}
	\item<+->{\large\bfseries Generalized (multinomial) logistic regression}
      \begin{itemize*}
	  \item Fit $m-1$ logits as a \emph{single} model
	  \item Results usually comparable to nested dichotomies
	  \item SAS: \PROC{LOGISTIC}, \texttt{LINK=GLOGIT}; R: \texttt{(multinom)}
	  \end{itemize*}
  \end{itemize}

\end{frame}

\subsection{What we've learned}
\begin{frame}
  \frametitle{Visualizing Categorical Data: What we've learned}
  \begin{itemize}
	\item<+->{\large\bfseries Categorical data}
      \begin{itemize*}
	  \item Table form vs.\ case form
	  \item Non-parametric methods vs.\ model-based methods
      \item Response models vs.\ association models
	  \end{itemize*}
	\item<+->{\large\bfseries Graphical methods for categorical data}
      \begin{itemize*}
	  \item Frequency data more naturally displayed as \textbf{count} $\sim$ {area}
	  \item Sieve diagram, fourfold \& mosaic display: compare observed vs.\ expected 
      \item Discrete response data benefits from: smoothing, effect plots
	  \item Graphical principles: Visual comparisons, effect ordering, small multiples
	  \end{itemize*}
	\item<+->{\large\bfseries Theory into practice}
      \begin{itemize*}
	  \item To be useful, statistical methods must be: 
    	\begin{itemize*}
		\item available--- implemented in standard software
		\item accessible--- easy to use (or at least easier)
		\end{itemize*}
	  \item \VCD\ provides $\sim$ 40 general macros and \IML\ programs
	  \item The \texttt{vcd} package for R does the same for R users.
      \item Effective statistical graphics is still hard work--- 80/20 rule
	  \end{itemize*}
%	\item{\large\bfseries }
  \end{itemize}
\end{frame}

