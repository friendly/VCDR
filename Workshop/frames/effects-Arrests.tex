%\documentclass[dvipsnames,pdflatex,compress,beamer]{beamer}
%%\usepackage{SweaveSlides}
%\usepackage{Sweave}
%\usepackage{mdwlist}
%\usepackage{comment}

%\definecolor{Sinput}{rgb}{0,0,0.56}
\definecolor{Sinput}{rgb}{1,0,0}
\definecolor{Scode}{rgb}{0,0,0.56}
%\definecolor{Soutput}{rgb}{0.56,0,0}
\definecolor{Soutput}{rgb}{0,0,1}
\DefineVerbatimEnvironment{Sinput}{Verbatim}{formatcom={\color{Sinput}},fontsize=\footnotesize,baselinestretch=0.9}
\DefineVerbatimEnvironment{Soutput}{Verbatim}{formatcom={\color{Soutput}},fontsize=\footnotesize,baselinestretch=0.85}
\DefineVerbatimEnvironment{Scode}{Verbatim}{formatcom={\color{Scode}},fontsize=\small}

%\begin{document}




% Simple slide

\subsection[Arrests]{Arrests for Marihuana Possession}

\begin{frame}
	\frametitle{Extended example: Arrests for Marihuana Possession}
	\framesubtitle{Context \& background}
  \begin{itemize}
  \item In Dec. 2002, the \emph{Toronto Star} examined the issue of \alert{racial profiling}, by analyzing
  a data base of 600,000+ arrest records from 1996-2002.
  
  \item They focused on a subset of arrests for which police action was \alert{discretionary}, e.g.,
  simple possession of small quantities of marijuana, where the police could:
  \begin{itemize*}
  	\item Release the arrestee with a summons--- like a parking ticket
  	\item Bring to police station, hold for bail, etc.--- harsher treatment
  \end{itemize*}

  \item \alert{Response} variable: \texttt{released} -- \texttt{Yes, No}

  \item Main \alert{predictor} of interest: skin-\texttt{colour} of arrestee (black, white)
  \end{itemize}  
%  Control variables:
%  \begin{itemize*}
%  	\item \texttt{year}, \texttt{age}, \texttt{sex}
%  	\item \texttt{employed},  \texttt{citizen } -- \texttt{Yes, No}
%  	\item \texttt{checks} --- Number of police data bases (previous
%arrests, previous convictions, parole status, etc.) in which the arrestee's
%name was found.
%  \end{itemize*}
\end{frame}


\begin{frame}[fragile]
	\frametitle{Extended example: Arrests for Marihuana Possession}
	\framesubtitle{Data}
  \alert{Control} variables:
  \begin{itemize*}
  	\item \texttt{year}, \texttt{age}, \texttt{sex}
  	\item \texttt{employed},  \texttt{citizen } -- \texttt{Yes, No}
  	\item \texttt{checks} --- Number of police data bases (previous
arrests, previous convictions, parole status, etc.) in which the arrestee's
name was found.
  \end{itemize*}

\begin{Schunk}
\begin{Sinput}
> library(effects)
> data(Arrests)
> some(Arrests)
\end{Sinput}
\begin{Soutput}
     released colour year age  sex employed citizen checks
915        No  Black 2001  35 Male      Yes     Yes      4
1568      Yes  White 2002  21 Male      Yes     Yes      0
2981      Yes  White 2000  23 Male      Yes     Yes      2
3381      Yes  Black 1998  23 Male       No     Yes      2
3516      Yes  White 2002  22 Male      Yes     Yes      0
4128       No  White 2001  29 Male      Yes     Yes      1
4142      Yes  Black 1998  23 Male      Yes     Yes      3
4634      Yes  White 2001  18 Male      Yes     Yes      0
4732      Yes  White 1999  21 Male      Yes     Yes      3
5183      Yes  White 1999  19 Male      Yes     Yes      0
\end{Soutput}
\end{Schunk}
\end{frame}

\begin{frame}[fragile]
	\frametitle{Extended example: Arrests for Marihuana Possession}
	\framesubtitle{Model}

To allow possibly non-linear effects of \texttt{year}, we treat it as a
factor:
\begin{Schunk}
\begin{Sinput}
> Arrests$year <- as.factor(Arrests$year)
\end{Sinput}
\end{Schunk}

Logistic regression model with all main effects, plus interactions of
\texttt{colour:year} and
\texttt{colour:age} 

\begin{Schunk}
\begin{Sinput}
> arrests.mod <- glm(released ~ employed + citizen + checks + colour * 
+     year + colour * age, family = binomial, data = Arrests)
> Anova(arrests.mod)
\end{Sinput}
\begin{Soutput}
Analysis of Deviance Table (Type II tests)

Response: released
            LR Chisq Df Pr(>Chisq)    
employed      72.673  1  < 2.2e-16 ***
citizen       25.783  1  3.820e-07 ***
checks       205.211  1  < 2.2e-16 ***
colour        19.572  1  9.687e-06 ***
year           6.087  5  0.2978477    
age            0.459  1  0.4982736    
colour:year   21.720  5  0.0005917 ***
colour:age    13.886  1  0.0001942 ***
---
Signif. codes:  0 '***' 0.001 '**' 0.01 '*' 0.05 '.' 0.1 ' ' 1 
\end{Soutput}
\end{Schunk}
%How to understand the nature of these effects?
\end{frame}

\begin{frame}[fragile]
	\frametitle{Effect plots: colour}
	Evidence for different treatment of blacks and whites (``racial profiling''),
	\alert{controlling} (adjusting) for other factors
	
	\setkeys{Gin}{width=.55\textwidth}

\begin{Schunk}
\begin{Sinput}
> plot(effect("colour", arrests.mod), multiline = FALSE, ylab = "Probability(released)")
\end{Sinput}
\end{Schunk}
\includegraphics{fig/eff-arrests-colour}
\end{frame}

%\begin{frame}[fragile]
%	\frametitle{Effect plots: Interactions}
%	The story turned out to be more nuanced than reported by the \emph{Toronto Star},
%	as shown in effect plots for interactions with colour.
%	\setkeys{Gin}{width=.55\textwidth}
%
%\begin{Schunk}
%\begin{Sinput}
%> plot(effect("colour:year", arrests.mod), multiline = TRUE, ...)
%\end{Sinput}
%\end{Schunk}
%\includegraphics{fig/eff-arrests-colour-year}
%\end{frame}

\begin{frame}[fragile]
	\frametitle{Effect plots: Interactions}
	The story turned out to be more nuanced than reported by the \emph{Toronto Star},
	as shown in effect plots for interactions with colour.
%	\setkeys{Gin}{width=.55\textwidth}

\begin{Schunk}
\begin{Sinput}
> plot(effect("colour:year", arrests.mod), multiline = TRUE, ...)
\end{Sinput}
\end{Schunk}
%\includegraphics{fig/eff-arrests-colour-year}
\begin{columns}
\begin{column}{0.55\textwidth}
\includegraphics[width=\textwidth,keepaspectratio=true,clip]{fig/eff-arrests-colour-year}
\end{column}
\begin{column}{0.45\textwidth}
	\begin{small}
	\begin{itemize}
	 \item Up to 2000, strong evidence for differential treatment of blacks
	       and whites
	 \item Also evidence to support Police claim of effect of training to
	 reduce racial effects in treatment
	\end{itemize}
    \end{small}
\end{column}
\end{columns}
\end{frame}

\begin{frame}[fragile]
	\frametitle{Effect plots: Interactions}
	The story turned out to be more nuanced than reported by the \emph{Toronto Star},
	as shown in effect plots for interactions with colour.
%	\setkeys{Gin}{width=.55\textwidth}

\begin{Schunk}
\begin{Sinput}
> plot(effect("colour:age", arrests.mod), multiline = TRUE, ...)
\end{Sinput}
\end{Schunk}
%\includegraphics{fig/eff-arrests-colour-age}
%% why does this generate an extra page???
\begin{columns}
\begin{column}{0.55\textwidth}
\includegraphics[width=\textwidth,keepaspectratio=true,clip]{fig/eff-arrests-colour-age}
\end{column}
\begin{column}{0.45\textwidth}
	\begin{small}
	\begin{itemize}
	 \item Opposite age effects for blacks and whites:
	 \item Young blacks treated \emph{more} harshly than young whites
	 \item Older blacks treated \emph{less} harshly than older whites
	\end{itemize}
\end{small}
\end{column}
\end{columns}
\end{frame}

%\begin{frame}[fragile]
%	\frametitle{Effect plots: Interactions}
%	The story turned out to be more nuanced than reported by the \emph{Toronto Star},
%	as shown in effect plots for interactions with colour.
%	\setkeys{Gin}{width=.55\textwidth}
%
%\begin{Schunk}
%\begin{Sinput}
%> plot(effect("colour:age", arrests.mod), multiline = TRUE, ...)
%\end{Sinput}
%\end{Schunk}
%\includegraphics{fig/eff-arrests-colour-age}
%%% why does this generate an extra page???
%%\begin{columns}
%%\begin{column}{0.59\textwidth}
%%\includegraphics[width=\textwidth,keepaspectratio=true,clip]{fig/eff-arrests-colour-age}
%%\end{column}
%%\begin{column}{0.4\textwidth}
%%	\begin{small}
%%	\begin{itemize}
%%	 \item Opposite age effects for blacks and whites:
%%	 \item Young blacks treated \emph{more} harshly than young whites
%%	 \item Older blacks treated \emph{less} harshly than older whites
%%	\end{itemize}
%%\end{small}
%%\end{column}
%%\end{columns}
%\end{frame}

\begin{frame}[fragile]
	\frametitle{Effect plots: allEffects}
  All model effects can be viewed together using \texttt{plot(allEffects(mod))}
	\setkeys{Gin}{width=.98\textwidth}

\begin{Schunk}
\begin{Sinput}
> arrests.effects <- allEffects(arrests.mod, xlevels = list(age = seq(15, 
+     45, 5)))
> plot(arrests.effects, ylab = "Probability(released)", ask = FALSE)
\end{Sinput}
\end{Schunk}
\includegraphics{fig/eff-arrests-all}
\end{frame}

%\end{document}
	
