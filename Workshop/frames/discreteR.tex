\renewcommand{\FileName}{discreteR}

\begin{frame}[fragile]
  \frametitle{Discrete distributions with R and the vcd package}
In R, discrete distributions are conveniently represented as one-way frequency tables,
\begin{Rin}[baselinestretch=0.8]
> library(vcd)
> data(Federalist)
> Federalist
\end{Rin}
\begin{Rout}
nMay
  0   1   2   3   4   5   6 
156  63  29   8   4   1   1 
\end{Rout}
The goodfit() function in vcd fits a variety of discrete distributions:
\begin{Rin}[baselinestretch=0.8]
> # fit the poisson model
> gf1 <- goodfit(Federalist, type="poisson")
> gf1
\end{Rin}
\begin{Rout}[baselinestretch=0.7,fontsize=\footnotesize]
Observed and fitted values for poisson distribution
with parameters estimated by `ML' 

 count observed       fitted
     0      156 135.89138870
     1       63  89.21114067
     2       29  29.28304617
     3        8   6.40799484
     4        4   1.05169381
     5        1   0.13808499
     6        1   0.01510854
\end{Rout}

\end{frame}

\begin{frame}[fragile]
R is object-oriented.  A goodfit object has print(), summary() and plot() methods:
\begin{Rin}
> summary(gf1)
\end{Rin}

\begin{Rout}
         Goodness-of-fit test for poisson distribution

                      X^2 df     P(> X^2)
Likelihood Ratio 25.24312  5 0.0001250511
\end{Rout}

\begin{Rin}
> plot(gf1, main="Federalist data: Poisson fit")
> plot(gf1, main="Federalist data: Poisson fit", type="dev")
\end{Rin}
% two figures
 \begin{minipage}[b]{.5\linewidth}
  \centering
  \includegraphics[width=.95\linewidth]{fig/federalist1}
 \end{minipage}%
 \begin{minipage}[b]{.5\linewidth}
  \centering
  \includegraphics[width=.95\linewidth]{fig/federalist2}
 \end{minipage}

\end{frame}

\begin{frame}[fragile]
  \frametitle{Discrete distributions with R and the vcd package}

The Poisson distribution 
\begin{Rin}
> # In a poisson, mean = var; this is 'over-dispersed'
>  mean(rep(0:6, times=Federalist))
\end{Rin}
\begin{Rout}
[1] 0.6564885
\end{Rout}
\begin{Rin}
>  var(rep(0:6, times=Federalist))
\end{Rin}
\begin{Rout}
[1] 1.007985
\end{Rout}

The negative binomial distribution, Nbin(r, p) 
allows the data to deviate from a true Poisson according to
a parameter $r>0$.
\begin{Rin}
> ## try negative binomial distribution (r, p)
> gf2 <- goodfit(Federalist, type = "nbinomial")
> summary(gf2)
\end{Rin}

\begin{Rout}
         Goodness-of-fit test for nbinomial distribution

                      X^2 df  P(> X^2)
Likelihood Ratio 1.964028  4 0.7423751
\end{Rout}
This has an acceptable fit to the Federalist data

\end{frame}

\begin{frame}[fragile]
  \frametitle{Discrete distributions with R and the vcd package}

Compare the fits side-by-side:
\begin{Rin}
> plot(gf2, main="Federalist data: Negative binomial fit")
> plot(gf1, main="Federalist data: Poisson fit")
\end{Rin}
% two figures
 \begin{minipage}[b]{.5\linewidth}
  \centering
  \includegraphics[width=.95\linewidth]{fig/federalist3}
 \end{minipage}%
 \begin{minipage}[b]{.5\linewidth}
  \centering
  \includegraphics[width=.95\linewidth]{fig/federalist1}
 \end{minipage}

{\large\bfseries Conclusions:}
\begin{itemize*}
  \item Perhaps marker words like 'may' do not occur with constant probability in all
  blocks of text
  \item Perhaps the blocks of text were written under different circumstances
\end{itemize*}

\begin{comment}
> # geometric = NBin (1,p)
> gf3 <- goodfit(Federalist, type = "nbinomial", par = list(size = 1))
> summary(gf3)

\begin{Rout}
         Goodness-of-fit test for nbinomial distribution

                      X^2 df  P(> X^2)
Likelihood Ratio 2.294143  5 0.8071267
\end{Rout}
> plot(gf3, main="Federalist data: Geometric distribution, NBin(1,p)")
\end{comment}

\end{frame}

\begin{frame}[fragile]
vcd includes Ord\_plot() and distplot() functions. E.g.,
\begin{Rin}
> Ord_plot(Federalist, 
       main = "Instances of 'may' in Federalist papers")
\end{Rin}
  \begin{center}
        \includegraphics[width=.72\dispwidth,clip]{fig/federalist5}
  \end{center}

\end{frame}

