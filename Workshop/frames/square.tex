\renewcommand{\FileName}{square}
% slide template
\begin{frame}
  \frametitle{Square tables}
  \begin{itemize}
	\item Tables where two (or more) variables have the same category levels:
      \begin{itemize*}
	  \item Employment categories of related persons (\alert{mobility tables})
	  \item Multiple measurements over time (\alert{panel studies}; longitudinal data)
	  \item \alert{Repeated measures} on the same individuals under different conditions
	  \item Related/repeated measures are rarely independent, but may have simpler
	  forms than general association
	  \end{itemize*}
	\item E.g., vision data: Left and right eye acuity grade for 7477 women
\begin{center}
  \includegraphics[width=.4\dispwidth,clip]{fig/mosaic10g1}
\end{center}
  \end{itemize}
\end{frame}

%\framebreak
\begin{frame}[fragile]
 \frametitle{Square tables: Quasi-Independence}
%  \begin{itemize}
%	\item{\large\bfseries Quasi-Independence}
		\begin{itemize}
			\item Related/repeated measures are rarely independent--- most observations often fall on \alert{diagonal cells}.
			\item \alert{Quasi-independence ignores diagonals}:  tests \alert{independence in remaining cells} ($\lambda_{ij}=0$
			for $i \ne j$).
			\item The model dedicates one parameter ($\delta_{i}$) to each diagonal cell, fitting them exactly,
	\begin{equation*}%\label{eq:saturated}
	 	\log m_{ij} = \mu + \lambda_i^A + \lambda_j^B + \delta_{i} \: I(i=j)
	\end{equation*}
	where $I(\bullet)$ is the indicator function.
			\item This model may be fit as a GLM by including indicator variables for each diagonal cell: fitted
			\alert{exactly}
        \end{itemize}
\begin{Output}[gobble=9,baselinestretch=0.85]
                   diag      4 rows      4 cols

                            \sasemph{1}         0         0         0
                            0         \sasemph{2}         0         0
                            0         0         \sasemph{3}         0
                            0         0         0         \sasemph{4}
\end{Output}
%  \end{itemize}
\end{frame}

%\framebreak
\begin{frame}[fragile]
	\begin{itemize}
			\item Using \PROC{GENMOD}
\vspace{1ex}
\begin{Input}[fontsize=\small,label=\fbox{$\cdots$ \texttt{mosaic10g.sas}},baselinestretch=0.7]
title 'Quasi-independence model (women)';
proc genmod data=women;
    class RightEye LeftEye \sasemph{diag};
    model Count = LeftEye RightEye \sasemph{diag} /
        dist=poisson link=log obstats residuals;
    ods output obstats=obstats;
%mosaic(data=\sasemph{obstats}, vorder=RightEye LeftEye, ...);
\end{Input}
Mosaic: 
\begin{center}
  \includegraphics[width=.45\dispwidth,clip]{fig/mosaic10g2}
\end{center}

	\end{itemize} 

\end{frame}

%\framebreak
\begin{frame}[fragile]
 \frametitle{Square tables: Symmetry}
%	\begin{itemize}
%	\item{\large\bfseries Symmetry}
		\begin{itemize}
		\item Tests whether the table is symmetric around the diagonal, i.e., $m_{ij} = m_{ji}$
		\item As a \loglin\ model, symmetry is  
	\begin{equation*}%\label{eq:saturated}
	 	\log m_{ij} = \mu + \lambda_i^A + \lambda_j^B + \lambda_{ij}^{AB} \comma
	\end{equation*}
	subject to the conditions
	\(
	 	\lambda_i^A = \lambda_j^B \quad\mbox{and}\quad \lambda_{ij}^{AB} = \lambda_{ji}^{AB} \period
	\)
	
			\item This model may be fit as a GLM by including \alert{indicator variables} with equal values
			for symmetric cells, and indicators for the diagonal cells (fit exactly)
		\end{itemize}
%	\end{itemize} 
\vspace{2ex}
\begin{Output}[gobble=9,baselinestretch=0.85]
                 symmetry      4 rows      4 cols)

                            \sasemph{1}        12        13        14
                           12         \sasemph{2}        23        24
                           13        23         \sasemph{3}        34
                           14        24        34         \sasemph{4}
\end{Output}

\end{frame}

%\framebreak
\begin{frame}[fragile]
	\begin{itemize}
			\item Using \PROC{GENMOD}

\begin{Input}[fontsize=\small,label=\fbox{$\cdots$ \texttt{mosaic10g.sas}},baselinestretch=0.7]
proc genmod data=women;
	class \sasemph{symmetry};
	model Count = \sasemph{symmetry} /
		dist=poisson link=log obstats residuals;
  	ods output obstats=obstats;
%mosaic(data=obstats, vorder=RightEye LeftEye, ...);
\end{Input}
Mosaic: 
\begin{center}
  \includegraphics[width=.4\dispwidth,clip]{fig/mosaic10g3}
\end{center}
  \end{itemize}

\end{frame}

% slide template
\begin{frame}[fragile]
%  \frametitle{}
  \begin{itemize}
	\item{\large\bfseries Quasi-Symmetry}
      \begin{itemize*}
	  \item Symmetry is often too restrictive: \implies equal marginal frequencies ($\lambda_i^A = \lambda_i^B$)
	  \item \PROC{GENMOD}:  Use the usual marginal effect parameters + symmetry:
\vspace{2ex} 
\begin{Input}[fontsize=\small,label=\fbox{$\cdots$ \texttt{mosaic10g.sas}},baselinestretch=0.7]
proc genmod data=women;
	class LeftEye RightEye \sasemph{symmetry};
	model Count = \sasemph{LeftEye RightEye symmetry} /
		dist=poisson link=log obstats residuals;
  	ods output obstats=obstats;
\end{Input}
\begin{center}
  \includegraphics[width=.4\dispwidth,clip]{fig/mosaic10g4}
\end{center}
	  \end{itemize*}
  \end{itemize}

\end{frame}

%\framebreak
\begin{frame}
\frametitle{Comparing models}

{\small
\input{tab/vision-summary}
}
\begin{itemize}
\item Only the \alert{quasi-symmetry} models provide an acceptable fit: 
  When vision is unequal, association is symmetric!
\item The ordinal quasi-symmetry model is \alert{most parsimonious}
\item AIC is your friend for model comparisons
\end{itemize} 
\end{frame}

\begin{frame}[fragile]
\frametitle{Using the \texttt{gnm} package in R}
 \begin{itemize*}
 	\item \func{Diag} and \func{Symm}: structured associations for square tables
 	\item \func{Topo}: more general structured associations
	\item \func{mosaic.glm} in \texttt{vcdExtra}
 \end{itemize*}
\begin{Rin}[baselinestretch=0.8]
library(vcdExtra)
library(gnm)
women <- subset(VisualAcuity, gender=="female", select=-gender)

indep <- glm(Freq ~ right + left, data = women, family=poisson)
mosaic(indep, residuals_type="rstandard", gp=shading_Friendly,
       main="Vision data: Independence (women)"  )

quasi.indep <- glm(Freq ~ right + left + Diag(right, left), 
       data = women, family = poisson)

symmetry <- glm(Freq ~ Symm(right, left), 
       data = women, family = poisson)

quasi.symm <- glm(Freq ~ right + left + Symm(right, left), 
       data = women, family = poisson)

# model comparisons: for *nested* models
anova(indep, quasi.indep, quasi.symm, test="Chisq")
anova(symmetry, quasi.symm, test="Chisq")
\end{Rin}
\end{frame}

\endinput

