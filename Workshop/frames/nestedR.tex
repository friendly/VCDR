\renewcommand{\FileName}{nestedR}
% slide template
\subsection{Nested dichotomies in R}
\begin{frame}[fragile]
  \frametitle{Nested dichotomies in R}
  In R, the steps are similar-- first create new variables, \texttt{working} and \texttt{fulltime}, using the
  \func{recode} function in the \pkg{car}:
\begin{Rin}
> library(car)   # for data and Anova()
> data(Womenlf)
> Womenlf <- within(Womenlf,{
+   working <- recode(partic, " 'not.work' = 'no'; else = 'yes' ")
+   fulltime <- recode (partic, 
+     " 'fulltime' = 'yes'; 'parttime' = 'no'; 'not.work' = NA")})
> some(Womenlf)
\end{Rin}

\begin{Rout}[baselinestretch=0.8,fontsize=\footnotesize]
      partic hincome children   region fulltime working
31  not.work      13  present  Ontario     <NA>      no
34  not.work       9   absent  Ontario     <NA>      no
55  parttime       9  present Atlantic       no     yes
86  fulltime      27   absent       BC      yes     yes
96  not.work      17  present  Ontario     <NA>      no
141 not.work      14  present  Ontario     <NA>      no
180 fulltime      13   absent       BC      yes     yes
189 fulltime       9  present Atlantic      yes     yes
234 fulltime       5   absent   Quebec      yes     yes
240 not.work      13  present   Quebec     <NA>      no
\end{Rout}
\end{frame}

\begin{frame}[fragile]
  \frametitle{Nested dichotomies in R: fitting}
  Then, fit models for each dichotomy:
\begin{Rin}
> contrasts(children)<- 'contr.treatment'
> mod.working <- glm(working ~ hincome + children, family=binomial, data=Womenlf)
> mod.fulltime <- glm(fulltime ~ hincome + children, family=binomial, data=Womenlf)
\end{Rin}
Some output from \texttt{summary(mod.working)}:
\begin{Rout}[frame=single]
Coefficients:
                Estimate Std. Error z value Pr(>|z|)    
(Intercept)      1.33583    0.38376   3.481   0.0005 ***
hincome         -0.04231    0.01978  -2.139   0.0324 *  
childrenpresent -1.57565    0.29226  -5.391    7e-08 ***
\end{Rout}
Some output from \texttt{summary(mod.fulltime)}:
\begin{Rout}[frame=single] 
Coefficients:
                Estimate Std. Error z value Pr(>|z|)    
(Intercept)      3.47777    0.76711   4.534 5.80e-06 ***
hincome         -0.10727    0.03915  -2.740  0.00615 ** 
childrenpresent -2.65146    0.54108  -4.900 9.57e-07 ***
\end{Rout}
\end{frame}

\begin{frame}[fragile]
  \frametitle{Nested dichotomies in R: plotting}
  For plotting, we need to calculate the predicted probabilities (or logits) over
  a grid of combinations of the predictors in each sub-model, using the \func{predict} function.
  
  \texttt{type='response'} gives these on the probability scale, whereas \texttt{type='link'}
  (the default) gives these on the logit scale.
\begin{Rin}
> pred <- expand.grid(hincome=1:45, children=c('absent', 'present'))
> # get fitted values for both sub-models
> p.work <- predict(mod.working, pred, type='response')
> p.fulltime <- predict(mod.fulltime, pred, type='response')
\end{Rin}
The fitted value for the fulltime dichotomy is \alert{conditional} on working outside the home;
multiplying by the probability of working gives the \alert{unconditional} probability.
\begin{Rin}
> p.full <- p.work * p.fulltime
> p.part <- p.work * (1 - p.fulltime)
> p.not <- 1 - p.work
\end{Rin}
\end{frame}

\begin{frame}
  \frametitle{Nested dichotomies in R: plotting}
  The plot below was produced using the basic R functions \func{plot}, \func{lines} and
  \func{legend}.  See the file \texttt{wlf-nested.R} on the course web page for details.
  \begin{center}
  \includegraphics[width=.98\textwidth, clip=0 10 0 30]{fig/wlf-nestedR}
  \end{center}
\end{frame}
