\renewcommand{\FileName}{genlogit}
% slide template
\subsection{Basic ideas}
\begin{frame}
  \frametitle{Polytomous response: Generalized Logits}
  \begin{itemize}
	\item Models the probabilities of the $m$ response categories 
	as \(m - 1\) logits comparing
each of the first \(m - 1\) categories to the last (reference) category.
	\item Logits for any  pair of categories can be calculated
from the \(m - 1\) fitted ones.
	\item With $k$ predictors, \(x_1, x_2, \dots , x_k\), for $j=1, 2, \dots ,
	  m-1$, 
\begin{eqnarray*}
  L_{jm}  \equiv 
    \log \left( \frac{\pi_{ij}}{\pi_{im}} \right) & = & \beta_{0j}  +
  \beta _{1j} \,  x_{i1}  +
  \beta _{2j} \,  x_{i2}  + \cdots +
  \beta _{kj} \,  x_{ik} \quad \\
%  \mbox{for } j=1, 2, \dots , m-1 \\
  & = & \vec{\beta}_j \trans \vec{x}_i
\end{eqnarray*}

      \begin{itemize*}
	  \item One set of fitted coefficients, $\vec{\beta}_j$ for each
response category except the last.
	  \item Each coefficient, $\beta_{hj}$, gives the effect on the log odds
of a unit change in the predictor $x_h$
that an observation belongs to category $j$ vs.\ category $m$.
	  \end{itemize*}
	\item Probabilities in response caegories are calculated as:
\begin{equation*}
\pi_{ij} =
 \frac{ \exp ( \vec{\beta}_j \trans \vec{x}_i ) }
      { \sum_{j=1}^{m-1} \exp ( \vec{\beta}_j \trans \vec{x}_i ) }
	  \comma \, j=1,\dots, m-1 \,;
	  \quad\quad \pi_{im} = 1 - \sum_{j=1}^{m-1} \pi_{ij}
\end{equation*}
  \end{itemize}
\end{frame}

\begin{frame}[fragile]
  \frametitle{Polytomous response: Generalized Logits}
Fitting generalized logit models with SAS:
%  \begin{itemize}
%	\item SAS:
      \begin{itemize*}
	  \item Use \PROC{LOGISTIC} with \texttt{LINK=GLOGIT} option.
    	\begin{itemize*}
		\item \ODS\ $\rightarrow$ fitted probabilities, $\widehat{\pi}_{ij}$ for all $m$ categories
		\item Overall tests and specific tests for each predictor, for all $m$ categories
		\end{itemize*}
\vspace{2ex}
\begin{Input}[numbers=none]
proc logistic data=wlfpart;
   model labor = husinc children / \sasemph{link=glogit};
   output out=results p=predict xbeta=logit;
\end{Input}
	  \item Can also use \PROC{CATMOD} with \texttt{RESPONSE=LOGITS} statement.
    	\begin{itemize*}
		\item Same model, same predicted probabilities
		\item Different syntax, \ODS\ format, plotting steps
		\item Quantitative variables: \alert{\texttt{direct}} statement
		\end{itemize*}
\vspace{2ex}
\begin{Input}[numbers=none]
proc catmod data=wlfpart;
   \sasemph{direct husinc;}
   model labor = husinc children;
   \sasemph{response logits} / out=results;
\end{Input}
	  \end{itemize*} 

%  \end{itemize}
\end{frame}

\begin{frame}<1>[label=wlfpart5g]
  \frametitle{Example: Women's Labour Force Participation}
Graphs:
% two figures 
 \begin{minipage}[b]{.5\linewidth}
  \centering
  \includegraphics[width=.99\linewidth]{fig/wlfpart51}
 \end{minipage}%
 \begin{minipage}[b]{.5\linewidth}
  \centering
  \includegraphics[width=.99\linewidth]{fig/wlfpart52}
 \end{minipage}
\end{frame}


\begin{frame}[fragile]
  \frametitle{Example: Women's Labour Force Participation}
\begin{Input}[label=\fbox{\texttt{wlfpart5.sas} $\cdots$}]
title 'Generalized logit model';
proc logistic data=wlfpart;
   model labor = husinc children / \sasemph{link=glogit};
   \sasemph{output out=results p=predict xbeta=logit;}
\end{Input}
Response profile:
\begin{Output}[gobble=7]
                       Ordered                      Total
                         Value        labor     Frequency

                             1            1            66
                             2            2            42
                             3            3           155

             Logits modeled use labor=3 as the reference category.
\end{Output}
\emph{Note:} Not working is the baseline category
\end{frame}


\begin{frame}[fragile]
Overall and Type III tests:
\begin{Output}[gobble=7,baselinestretch=0.8]
                    Testing Global Null Hypothesis: BETA=0
 
            Test                 Chi-Square       DF     Pr > ChiSq

            Likelihood Ratio        77.6106        4         <.0001
            Score                   76.4850        4         <.0001
            Wald                    58.4351        4         <.0001

                          Type III Analysis of Effects
 
                                            Wald
                  Effect        DF    Chi-Square    Pr > ChiSq

                  husinc         2       12.8159        0.0016
                  children       2       53.9806        <.0001
\end{Output}
These are comparable to the combined tests for the nested dichotomies models.
\end{frame}

\begin{frame}[fragile]
Coefficients:
\begin{Output}[gobble=2,baselinestretch=0.8, fontsize=\footnotesize]
                    Analysis of Maximum Likelihood Estimates
 
                                          Standard          Wald
  Parameter    labor    DF    Estimate       Error    Chi-Square    Pr > ChiSq

  Intercept    1         1      1.9828      0.4842       16.7709        <.0001
  Intercept    2         1     -1.4323      0.5925        5.8445        0.0156
  husinc       1         1     -0.0972      0.0281       11.9762        0.0005
  husinc       2         1     0.00689      0.0235        0.0863        0.7689
  children     1         1     -2.5586      0.3622       49.9008        <.0001
  children     2         1      0.0215      0.4690        0.0021        0.9635
\end{Output}
i.e., the fitted models are:
\begin{eqnarray*}
  \log \left( \frac{ \Pr ( \mbox{fulltime} ) }
  { \Pr ( \mbox{not working} ) } \right) & = &
  1.983 - 0.097 \,  \mbox{H\$} - 2.56 \,  \mbox{kids} \\  %\label{eq:wlfgen1}
%
  \log \left( \frac{ \Pr ( \mbox{parttime} ) }
  { \Pr ( \mbox{not working} ) } \right) & = &
  -1.432 - 0.0069 \,  \mbox{H\$} - 0.0215 \,  \mbox{kids}  % \label{eq:wlfgen2}
\end{eqnarray*}
\emph{Interpretation:} Signs for \texttt{husinc} and \texttt{children} are understandable,
but need to make a plot!
\end{frame}

\begin{frame}[fragile]
\ODS\ \texttt{results} (for plots):
\begin{Output}[baselinestretch=0.7,gobble=2]
  case   labor   husinc  children   _LEVEL_     logit    predict

    1      3       15        1         1      -2.03423   0.09333
    1      3       15        1         2      -1.30743   0.19305
    1      3       15        1         3        .        0.71363
    2      3       13        1         1      -1.83977   0.11142
    2      3       13        1         2      -1.32122   0.18715
    2      3       13        1         3        .        0.70143
    3      3       45        1         1      -4.95114   0.00528
    3      3       45        1         2      -1.10067   0.24830
    3      3       45        1         3        .        0.74642
    4      3       23        1         1      -2.81207   0.04464
    4      3       23        1         2      -1.25230   0.21238
    4      3       23        1         3        .        0.74298
    5      3       19        1         1      -2.42315   0.06486
    5      3       19        1         2      -1.27987   0.20346
    5      3       19        1         3        .        0.73168
    6      3        7        1         1      -1.25639   0.18478
    6      3        7        1         2      -1.36257   0.16616
    ...
\end{Output}
\begin{itemize*}
	\item \texttt{logit} gives the two fitted log odds vs Not working
	\item \texttt{predict} gives the predicted probability for each category of \texttt{labor}
\end{itemize*}
\end{frame}

\begin{frame}[fragile]
  \frametitle{Example: Women's Labour Force Participation}
\begin{Input}[label=\fbox{$\cdots$ \texttt{wlfpart5.sas}},baselinestretch=0.7]
proc sort data=results;
   \sasemph{by children husinc _level_;}

   \sascomment{*-- Curve labels;}
%label(data=results, x=husinc, y=predict, cvar=_level_,
   by=children, subset=last._level_, text=put(_level_, labor.), 
   pos=2, out=labels1);

   \sascomment{*-- Panel labels;}
%label(data=results, x=20, y=0.85, 
   by=children, subset=last.children, text=put(children, kids.), 
   pos=2, size=2, out=labels2);
data labels;
   set labels1 labels2;
   by children;

goptions hby=0;
proc gplot data=results;
   \sasemph{plot predict * husinc = _level_} / 
      vaxis=axis1 hm=1 vm=1 \sasemph{anno=labels} nolegend;
   by children;
   axis1 order=(0 to .9 by .1) label=(a=90);
   symbol1 i=join v=circle   c=black;
   symbol2 i=join v=square   c=red;
   symbol3 i=join v=triangle c=blue;
   run;
\end{Input}
\end{frame}

\againframe<1>{wlfpart5g}
\endinput

% slide template
\begin{frame}
  \frametitle{}
  \begin{itemize}
	\item{\large\bfseries }
      \begin{itemize*}
	  \item 
    	\begin{itemize*}
		\item 
		\item 
		\end{itemize*}
	  \item 
	  \end{itemize*}
	\item{\large\bfseries }
	\item{\large\bfseries }
  \end{itemize}
\end{frame}

