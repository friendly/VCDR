\renewcommand{\FileName}{genlogitR}
% slide template
\subsection{Generalized logit models in R}
\begin{frame}[fragile]
  \frametitle{Generalized logit models in R: Fitting}
  \begin{itemize*}
    \item In R, the generalized logit model can be fit using the \func{multinom} function in the \pkg{nnet}
	\item For interpretation, it is useful to reorder the levels of \texttt{partic} so that \texttt{not.work}
	is the baseline level.
  \end{itemize*}
  
\begin{Rin}
Womenlf$partic <- ordered(Womenlf$partic, 
    levels=c('not.work', 'parttime', 'fulltime'))
library(nnet)
mod.multinom <- multinom(partic ~ hincome + children, data=Womenlf)
summary(mod.multinom, Wald=TRUE)
Anova(mod.multinom)
\end{Rin}

The \func{Anova} tests are similar to what we got from summing these tests from the
two nested dichotomies:
\begin{Rout}[baselinestretch=0.8,fontsize=\footnotesize,frame=single]
Analysis of Deviance Table (Type II tests)

Response: partic
         LR Chisq Df Pr(>Chisq)    
hincome      15.2  2    0.00051 ***
children     63.6  2    1.6e-14 ***
---
Signif. codes:  0 '***' 0.001 '**' 0.01 '*' 0.05 '.' 0.1 ' ' 1 
\end{Rout}
\end{frame}

\begin{frame}[fragile]
  \frametitle{Generalized logit models in R: Plotting}
  \begin{itemize*}
  	\item As before, it is much easier to interpret a model from a plot than from
  	coefficients, but this is particularly true for polytomous response models
    \item \texttt{style="stacked"} shows cumulative probabilities
  \end{itemize*}
\begin{Rin}
library(effects)
plot(effect("hincome*children", mod.multinom), style="stacked")
\end{Rin}
\begin{center}
	\includegraphics[height=.6\textheight]{fig/wlf-glogit-effplot}
\end{center}
\end{frame}

\begin{frame}[fragile]
  \frametitle{Generalized logit models in R: Plotting}
  \begin{itemize*}
  	\item You can also view the effects of husband's
  	income and children separately in this main effects model with \texttt{plot(allEffects))}.
  \end{itemize*}
\begin{Rin}
plot(allEffects(mod.multinom), ask=FALSE)
\end{Rin}
\begin{center}
	\includegraphics[height=.7\textheight]{fig/wlf-glogit-effplot1}
\end{center}
\end{frame}
