\documentclass[11pt]{article}

\usepackage{sasnames,mdwlist}
\usepackage{array}
\usepackage{xspace}
\usepackage{fancyvrb}
\usepackage[bookmarks,breaklinks,
%    colorlinks=false,%  for printed version
    ]{hyperref}

\usepackage[comma]{natbib}
\bibliographystyle{abbrvnat-last}

% linked files
\newcommand{\VCDweb}{http://datavis.ca/courses/VCD/}
\newcommand{\file}[1]{\href{\VCDweb#1}{#1}\marginpar{\footnotesize{#1}}}
%\newcommand{\file}[1]{\href{\VCDweb#1}{#1}\marginpar{\footnotesize{#1}}}
\newcommand{\Vfile}[1]{\marginpar{\href{\VCDweb#1}{#1}}}

\newcommand{\button}[1]{\framebox{\textbf{#1}}}

\newcommand*{\secref}[1]{Section~\ref{#1}}
\newcommand*{\tabref}[1]{Table~\ref{#1}}
\newcommand*{\figref}[1]{Figure~\ref{#1}}

\newcommand{\given}{\ensuremath{\, | \,}}
\renewcommand{\vec}[1]{\ensuremath{\bm{#1}}}
\newcommand{\mat}[1]{\ensuremath{\bm{#1}}}
\newcommand{\trans}{\ensuremath{^\mathsf{T}}}
\newcommand{\diag}[1]{\ensuremath{\mathrm{diag} (#1)}}
\def\binom#1#2{{#1 \choose #2}}%
\newcommand*{\period}{\:\: .}
\newcommand*{\comma}{\:\: ,}
\newcommand{\implies}{\ensuremath{\Rightarrow} }
\newcommand{\ignore}[1]{}
% R stuff
\newcommand{\pkg}[1]{\texttt{#1} package}

% page dimensions
\addtolength{\hoffset}{-1.5cm}
\addtolength{\textwidth}{3cm}
\addtolength{\voffset}{-2cm}
\addtolength{\textheight}{4cm}

% paragraph formatting
\setlength{\parindent}{0pt}

\let\proglang=\textsf
\newcommand{\SAS}{\proglang{SAS}\xspace}
\newcommand{\R}{\proglang{R}\xspace}
\newcommand{\Rnote}{\marginpar{\fbox{\color{red}\proglang{R}}}}
\newcommand{\SASnote}{\marginpar{\fbox{\color{blue}\proglang{SAS}}}}


\newcommand{\Tasks}{\marginpar{\raggedleft\small\textbf{\textit{Tasks \& questions}}}}
\newcommand{\codefun}[1]{\texttt{#1()}}
%% code - allow special characters in code strings
\makeatletter
\newcommand\code{\bgroup\@makeother\_\@codex}
\def\@codex#1{{\normalfont\ttfamily\hyphenchar\font=-1 #1}\egroup}
\makeatother
%\newcommand{\code}[1]{\texttt{#1}}

\DefineVerbatimEnvironment{Rin}{Verbatim}{formatcom=\color{red},numbers=none,baselinestretch=0.9}
\DefineVerbatimEnvironment{SASin}{Verbatim}{formatcom=\color{blue},numbers=none,baselinestretch=0.9}

%\usepackage{ifthen}
%\newboolean{SCS}
%\setboolean{SCS}{true}
%%\setboolean{SCS}{false}
%\ifthenelse{SCS}%
%	{\newcommand{\SASserverdir}{N:\\Vcd}}%
%	{\newcommand{\SASserverdir}{C:\\sasuser\\Vcd}}

\newcommand{\SASserverdir}{\texttt{N:/Vcd}}
	
%{\newcommand{\SASserverdir}{C:\\sasuser\\Vcd}}


%opening
\title{Visualizing Categorical Data with \SAS and \R: Exercises}
\author{Michael Friendly \and Ernest Kwan \and Cathy LaBrish \and Bin Sun}

\begin{document}

\maketitle

\section*{Getting started}
\addcontentsline{toc}{section}{Getting started}
The goal of these workshops is to give you some hands-on experience using the methods
described in the lectures. For each lecture, the plan is to have you try to reproduce,
and possibly extend, the analysis and visualization of one or two examples used in that
lecture, with \proglang{SAS} or \proglang{R} (or both), as you are most comfortable with.
In addition, there may also be one or two new examples to try.  Some materials for these
workshops are available online at
\url{\VCDweb}.


For reference, \tabref{tab:dataset-table} lists the principal data sets that are used in the
lectures. The \texttt{SAS} column gives the name of the \texttt{.sas} file containing the
data and the name of the \proglang{SAS} data set created from this.
The \texttt{R} column gives the name of the data set in \proglang{R}. An asterisk (*) marks
the first use of each data set.


\begin{table}[htb]
\begin{center}
\renewcommand{\arraystretch}{0.85}
	\caption{Data sets used in the lectures}\label{tab:dataset-table}
\begin{tabular}{ll>{\ttfamily}l>{\ttfamily}l}
  \hline
Part &  Data set           &  SAS        & R           \\[0.5ex]  \hline   \hline
1    & Federalist *        &  madison    & Federalist  \\
     & Cholesterol-HD *    &  fat        &             \\
     & Arthritis *         &  arthrit    & Arthritis   \\[1ex]
%
2    & Berkeley *          & berkeley   & UCBAdmissions \\
     & Hair-Eye-Color *    & haireye    & HairEyeColor \\
     & Distant-Vision *    & vision     & VisualAcuity \\
     & Sex-Fun *           & sexfun     & SexualFun    \\
     & MS diagnosis *      & msdiag     & MSPatients   \\
     & Suicide-Rates *     & suicide    & Suicide      \\[1ex]
%\vskip{1ex}
3    & Berkeley            & berkeley   & UCBAdmissions \\
     & Hair-Eye-Color *    & haireye    & HairEyeColor \\
     & Mental-Health-SES * & mental     & Mental [vcdExtra] \\
     & Distant-Vision *    & vision     & VisualAcuity \\[1ex]
%\vskip{1ex}
4    & Berkeley            & berkeley    & UCBAdmissions \\
     & Arthritis           & arthrit     & Arthritis   \\
     & Volunteering *      & cowles      & Cowles [car] \\
	 & Arrests *           & arrests     & Arrests [effects]\\[1ex]
%\vskip{1ex}
5    & Arthritis           &  arthrit    & Arthritis   \\
     & Womens-Labor-Force * & wlfpart     & Womenlf [car] \\    \hline
\end{tabular}
\end{center}
\end{table}


\subsection*{SAS and R setups}
\addcontentsline{toc}{subsection}{SAS and R setups}
Here are a few things you should know to get started using \SAS and \R.

\begin{description}
 \item[SAS] All of the \proglang{SAS} programs, data sets and macros are stored in
\SASnote
 \SASserverdir.  
 Documentation for all of my macro programs is available at \url{http://datavis.ca/sasmac/},
 mostly under the \textbf{Categorical data} menu item.
% There is also an archive, \texttt{vcdprog.zip} containing these
% programs you can copy and take home with you.
 
 To use a data set in one of the exercises, start with \verb|%include catdata(dataset)|
 for example,
\begin{SASin}
%include catdata(madison);      *-- Federalist papers, 'may' by Madison;
%include catdata(arthrit);      *-- Arthritis treatment data;
\end{SASin}

The \SAS macro programs are installed in a \texttt{macros} directory that is automatically
searched when you invoke it with \verb|%macroname(...)|.  For example:
\begin{SASin}
%include catdata(madison);      *-- Federalist papers, 'may' by Madison;
%goodfit(data=madison, var=count, freq=blocks, dist=poisson);
\end{SASin}
Within \SAS, you can access the web help page for any macro with \verb|%webhelp(macroname);|.
For example,
\begin{SASin}
%webhelp(goodfit);
\end{SASin}
For help on \SAS procedures, you can use the Help menu in \SAS, or access the complete
SAS Online documentation (v. 9.1.3) via
\begin{SASin}
%webhelp(online);
\end{SASin}


 \item[R] All of the \proglang{R} programs and data sets should be pre-installed
\Rnote
 in R on the server, but some of them may be outdated.
 The \emph{first} time you start \proglang{R}, you should update the available
 packages and load the \pkg{vcdExtra} to make available some extra workshop
 materials.
%install.packages("vcdExtra")  # NB: Put this on CRAN!!
%install.packages("vcdExtra")
\begin{Rin}
update.packages(ask=FALSE)
library(vcdExtra)
\end{Rin}
 
 Thereafter, when you start \proglang{R}, you should begin with
\begin{Rin}
library(vcdExtra)
\end{Rin}
to load the \pkg{vcdExtra}.  This includes \texttt{vcd} plus some extra data sets.

To load a data set from an \R package, use the \codefun{data} function. To get
a description of any \R data set or function, use \codefun{help}, or the short-hand \verb|?|.
\begin{Rin}
data(Federalist)        # load the data
?Federalist             # read the help()
?mosaic                 # help for mosaic()
\end{Rin}

\end{description}

\subsection*{Home-work}
In the limited time available in the lab sessions, it is unlikely that you will
be able to complete all the exercises or
to master all of these techniques.  To profit more fully, and apply them
to your own data, you  should try to practice more at home or at work.

To this end, most of the materials for this course and the workshops are available at
\linebreak\url{\VCDweb}.
From there you can download the \SAS data sets and macros (in the archive
\file{vcdprog.zip})
and the necessary
\R packages.

\newpage
\section{Introduction \& overview}

%\subsection{Reading data}
%The first thing to learn is how to access the data sets and programs in \SAS and \R.
%
%In \SAS, all of the data sets for the workshops are contained in directories referred
%\SASnote
%to by the symbolic name \texttt{catdata}.  To read one into \SAS, use a
%\verb|%include| statement like the following:
%\begin{verbatim}
%%include catdata(madison);
%\end{verbatim}

\subsection{Discrete distributions}\label{sec:discrete}
In this exercise, we will examine the use of diagnostic and graphic methods for
fitting discrete distributions to the one-way frequency tables shown in
\tabref{tab:madison} and \tabref{tab:saxony}.  

\begin{table}[htb!]
\caption{Number of occurrences ($k$) and number of blocks of text ($n_k$) of the word \emph{may} in 
essays written by  Madison. \SAS: \texttt{madison}; \R: \texttt{Federalist}}\label{tab:madison}
%\begin{table}[htb]
% \caption{Number of occurrences ($k$) and number of blocks of text ($n_k$) of the word \emph{may} in 
%essays written by  Madison}\label{tab:madison}
 \begin{center}
 \begin{tabular}{l|rrrrrrr}
  \hline
  Occurrences ($k$)   &   0 &  1 &  2 & 3 & 4 & 5 & 6 \\ 
      Blocks ($n_k$)  & 156 & 63 & 29 & 8 & 4 & 1 & 1 \\ 
  \hline
 \end{tabular}
 \end{center}
%\end{table}

\end{table}
\vspace{-2ex}
\begin{table}[htb!]
\caption{Number of males in N=6115 Saxony families with 12 children. \SAS: \texttt{saxony};
\R: \texttt{Saxony}}\label{tab:saxony}
%Unequal number of columns: 14 13
%\begin{table}[htb]
% \caption{Number of males in N=6115 Saxony families with 12 children}\label{tab:saxony}
 \begin{center}
 \setlength{\tabcolsep}{3pt}
 \begin{tabular}{l|rrrrrrrrrrrrr}
  \hline
  $k$ & 0 & 1 & 2 & 3 & 4 & 5 & 6 & 7 & 8 & 9 & 10 & 11 & 12 \\ 
  \hline
  $n_k$ & 3 & 24 & 104 & 286 & 670 & 1033 & 1343 & 1112 & 829 & 478 & 181 & 45 & 7 \\ 
  \hline
 \end{tabular}
 \end{center}
%\end{table}

\end{table}
\vspace{-1ex}
For the Madison data, it might be supposed that number of occurrences of the word \emph{may}
follow a Poisson distribution.

\begin{enumerate*}
\Tasks
 \item Fit a Poisson distribution to the data.  Do you conclude that this distribution
  fits well or badly?
 \item Examine the departure between the observed and fitted frequencies with a hanging rootogram.
 Is there a pattern to the departure?
 \item Use an Ord plot to diagnose the form of the distribution.%
\footnote{
To judge whether a coefficient is positive or negative a tolerance
is used. Be
careful with the conclusions from the Ord plot, as it implements just some simple
heuristics!}
 \item If necessary, fit a different distribution and examine visually with a hanging rootogram.
\end{enumerate*}

With \SAS, you can begin these steps with statements like the following:
\SASnote
\begin{SASin}
%include catdata(madison);
%goodfit(data=madison, var=count, freq=blocks, dist=poisson, out=fit);
%rootgram(data=fit, var=count, obs=blocks);

%ordplot(data=madison, count=count, freq=blocks);
\end{SASin}

With \R the following statements will get you started:
\Rnote
\begin{Rin}
data(Federalist)
gf <- goodfit(Federalist, type='poisson')
summary(gf)
plot(gf)

Ord_plot(Federalist, main='may in Federalist papers')
\end{Rin}

For the Saxony data, it might be supposed that number of male children 
follow a Binomial distribution, perhaps also with $p(Male)=0.5$.
Repeat the steps above using the Saxony data, with a Binomial distribution.

\subsection{Testing association}\label{sec:assoc}
The Berkeley data set (Bickel et al., 1975) involves graduate school  applicants
to the six  largest departments at  University of California,  Berkeley in 1973.
The applicants are classified by \code{Admit} (admitted vs. rejected), \code{Gender} (male vs.
female), and \code{Dept} (A to F). 

This data set forms  a 3-way
table (i.e., a $2 \times 2 \times 6$ table), but in this exercise, we will first look at the
$2 \times 2$ table of \code{Admit} by \code{Gender}, collapsed over \code{Dept}. 
In other
words, we will ignore \emph(temporarily) the department to which each applicant applied.
In later lectures and other exercises we will examine the full 3-way table.

This analysis is of interest because
an association between \code{Admit} and \code{Gender}--- different rates of admissions
for male and female applicants could be considered as evidence of gender bias.

\begin{enumerate*}
\Tasks
\item  Construct the two-way frequency table by collapsing over \code{Dept},
and test whether \code{Admit} and \code{Gender} are independent.
\item Find the proportion admitted for males and females and relate these to
the marginal frequencies and expected frequencies under independence.
An independent relationship between Admit and Gender in the sample implies  that
cell frequencies would be proportionate to the product of marginal  frequencies.
For example,  59.5\% (2691/4526)  of applicants  are male,  61.2\% (2771/4526)  of
applicants are rejected.
\item  Do  these frequencies  suggest
approximately  equal  admission rates  in  males and  females?  If not,  try  to
describe the possible gender difference in admission.  What is a good statistic
to summarize the relation between gender and admission?
\end{enumerate*}

In \SAS, with a data set in frequency form, you can collapse over any variables
\SASnote
simply by omitting them from the \code{tables} statement in \PROC{FREQ}.
See \file{berkeley-freq.sas}
\begin{SASin}
%include catdata(berkeley);
proc freq data=berkeley;
   weight freq;
   tables gender*admit / chisq plots=all;
	 format admit admit.;
   run;
\end{SASin}
	
For \R, the data set \code{UCBAdmissions} is represented as a 3-way table in table form.
\Rnote
Use \codefun{margin.table} to collapse any such table over omitted dimensions.
\codefun{CrossTable} in the \pkg{gmodels} has many options for displaying
quantities of interest in contingency tables.

\begin{Rin}
data(UCBAdmissions)
UCB.GA <-margin.table(UCBAdmissions, c(1,2))
chisq.test(UCB.GA)

library(gmodels)
CrossTable(UCB.GA, prop.t=FALSE, prop.r=FALSE, expected=TRUE)
\end{Rin}


\subsection{Stratified analysis}\label{sec:stratified}
The marginal analysis of the Berkeley data in \secref{sec:assoc} is misleading, because
(as it turns out)
it falsely assumes that relation between \code{Admit} and \code{Gender} is the \emph{same}
for each department.  Only if this were true, would the collapsed table
represent the relation in \emph{all} departments.

Here, you will use a stratified analysis to examine the relation between
\code{Admit} and \code{Gender} \emph{controlling for} department
(equivalently: a test of \emph{conditional} independence \code{Admit} and \code{Gender}
\emph{given} \code{Gender}),
as opposed
to \emph{ignoring} department in the marginal analysis.  
In Lectures II and III
we will see some visualizations of these data that help to explain the
contradictory results.

\begin{enumerate*}
\Tasks
\item Carry out stratified tests of the association between \code{Admit} and \code{Gender}
controlling for \code{Dept}.
\item What do you conclude about whether there is a difference in rate of admission
in \emph{each} of the departments?
\item Test for conditional independence of \code{Admit} and \code{Gender} given \code{Dept}.
\item For one $2\times 2$ table, the odds ratio is a reasonable summary measure of
association.
For a collection of  $k$ such tables, the Breslow-Day $\chi^2$ tests
whether these are equal across strata.  What do you conclude?
\end{enumerate*}

Using \SAS, \PROC{FREQ} will carry out a stratified analysis when there are
\SASnote
\emph{more than two} factors given in the \stmt{tables}.  You get a separate page of
output for each stratum, followed by overall summary tests of the relation
between the two primary table variables, \emph{controlling for} the strata.
These are the tests and statistics of most interest here.

\begin{SASin}
%include catdata(berkeley);
proc freq data=berkeley;
  weight freq;
  tables dept * gender * admit /chisq cmh;
  format dept dept. admit admit.;
  run;
\end{SASin}
	
In \R, for $2 \times 2 \times K$ tables, the \codefun{mantelhaen.test} function provides
\Rnote
a test of overall association pooled over strata, \verb|woolf_test()| tests for
equal odds ratios across strata (similar to the Breslow-Day test),
and \codefun{CMHtest} in \pkg{vcdExtra} gives generalized CMH tests like those
provided by the \code{cmh} option in \SAS.

Tests of similar hypotheses 
can be carried out using \loglin\ models (\codefun{loglm}) and generalized
linear models (\codefun{glm}), described in later lectures.
Here is one simple way to test for association of \code{Admit} and \code{Gender}
\emph{separately} for each department. \codefun{oddsratio} calculates and tests
the (log) odds ratio for each stratum.
\begin{Rin}
for (dept in 1:6) {print(chisq.test(UCBAdmissions[,,dept]))}
summary(oddsratio(UCBAdmissions))
\end{Rin}
Some tests related to this exercise can be done as follows.  You
should try to understand what hypothesis is tested by each one.
\begin{Rin}
mantelhaen.test(UCBAdmissions)
woolf_test(UCBAdmissions)
\end{Rin}

\subsection{Ordinal factors}
The Sex-Fun data set (Hout et al., 1987) involves the responses of 91 couples to the questionnaire item:
\emph{Sex is fun for me and my partner}: (a) never or occasionally, (b) fairly often, (c) very often, (d) almost always.

The data can be regarded as a $4 \times 4$ frequency table, where both factors 
can be considered to be ordered.
For today, we will just consider testing whether there is an association between
the husband's and wife's rating, and how to take ordinality into account,
as well as the relatively small sample size.

In this problem, we could arguably be more interested in assessing to what extent the
ratings actually \emph{agree}-- there could actually be a strong \emph{negative}
association between the Husband's and Wife's rating of sexual fun!
Methods for assessing agreement will be
discussed in the lecture for Part II.

\begin{enumerate*}
\item Carry out a simple Pearson $\chi^2$ test for association of \code{Husband} and \code{Wife}.
\Tasks
Are there any reasons to question the validity of this test?
\item Carry out the analogous Cochran-Mantel-Haenszel tests.  \SAS reports three different
tests.  Which one is most appropriate here, where both variables are ordinal?
\item For relatively small total sample size, where the asymptotic distribution of
the Pearson $\chi^2$ statistic is in doubt, Fisher's exact test provides an alternative.
Do it.
\item Examine the association statistics between the ratings.  What would be a reasonable
value to take as a non-parametric measure of ``correlation'' between Husband and Wife ratings?%
\footnote{
Hint: the Cochran-Mantel-Haenszel $\chi^2_{CMH} = (n-1) r^2$ when both variables are
assigned integer scores. The Phi coefficient, $\phi = \sqrt{\chi^2_{P} / n}$,
but for larger than $2\times 2$ tables it has a range
$0 \le \phi \le \min(\sqrt{r-1}, \sqrt{c-1})$.
Cramer's V is a corrected scaling of $\phi$ so that $0 \le V \le 1$, similar to $r^2$.
}
\end{enumerate*}

For \SAS, you can carry out all these steps as follows shown below.
\SASnote
See the file \file{sexfun-cmh.sas}.
\begin{SASin}
%include catdata(sexfun);
proc freq data=sexfun order=data;
  weight count;
  tables husband * wife / cmh chisq exact nocol norow;
  run;
\end{SASin}
For comparison with the association statistics, note that you can also calculate the
Pearson correlation between the ratings using \PROC{CORR}:
\begin{SASin}
proc corr data=sexfun ;
  weight count;
  var husband wife;
  run;
\end{SASin}


In \R, the equivalent of the \code{cmh} option in \PROC{FREQ} is \codefun{vcdExtra::CMHtest}.
You can do the various steps as follows.  See \file{sexfun-chisq.R}.
\Rnote
\begin{Rin}
library(vcdExtra)
data(SexualFun)
SexualFun                # show the table
CMHtest(SexualFun)       # CMH tests
chisq.test(SexualFun)    # general association
fisher.test(SexualFun)   # exact test
assocstats(SexualFun)    # association statistics
\end{Rin}
 
\newpage

\section[Two-way and n-way tables]{Two-way and $n$-way tables}\label{sec:two-way}
In these exercises you will examine several of the methods for visualizing 
association in two-way and $n$-way tables.
The goal is not simply to see \emph{if} an association exists, but to
understand the nature or pattern of association--- \emph{why} or \emph{how}
the variables are associated.

\subsection[2 x 2 tables]{$2 \times 2$ tables}
In $2 \times 2$ tables, there are several measures of association between the row
and column variables, but one of the simplest is the \emph{odds ratio}, $\theta$,
$ 0 \le \theta \le \infty$, 
estimated in the sample as
$$
\hat{\theta} = \frac{ n_{11} / n_{12} }{ n_{21} / n_{22}}
       = \frac{ n_{11} n_{22} }{ n_{12} n_{21}}
$$
When there is no association, $\theta=1$ (or, equivalently, $\log(\theta)=0$).
For a stratified $2 \times 2 \times k$ table, another question of interest is
whether the odds ratios are the same for all strata,
$ \theta_1 = \theta_2, \dots , \theta_k$.  In this exercise, you will use
fourfold displays to examine these questions.

\subsubsection{Cholesterol and heart disease}
\begin{enumerate*}
	\Tasks
	\item Test whether there is an association between cholesterol and heart disease
	using a fourfold display.  
	\item What evidence is shown in the fourfold display regarding whether the odds
	ratio differs significantly from 1?
	\item Describe \emph{how} cholesterol and heart disease are related.
\end{enumerate*}

In \SAS, the data are in \code{fat.sas} in the \code{catdata} directory. 
\SASnote
This file also runs a \code{proc freq} to give answers to the first question.
\begin{SASin}
%include catdata(fat);
%ffold(data=fat, var=diet disease);
\end{SASin}

In \R, you can just create the \code{fat} data using \codefun{matrix}.
\Rnote
\code{chisq.test(fat)} will give an approximate answer to the first question.
\begin{Rin}
fat <- matrix( c(6, 4, 2, 11), 2, 2)
dimnames(fat) <- list(diet=c("LoChol", "HiChol"), disease=c("No", "Yes"))
fourfold(t(fat))
\end{Rin}

\subsubsection{Berkeley data}
Here, you will re-examine the Berkeley admissions data considered in
\secref{sec:assoc} and \secref{sec:stratified} using fourfold plots.
\begin{enumerate*}
	\Tasks
	\item Assess visually whether there is an association between gender and admission,
	in the marginal table that collapses over department.  
	\item Assess visually the association between gender and admission separately
	in each department.  What do you conclude?
	\item Examine the question of whether the odds ratio of admission for males and
	females is the \emph{same} for all departments.  [Hint: See the printed output
	from the \macro{ffold}, or use the \verb|woolf_test()| in \R.]
\end{enumerate*}

In \SAS, the \macro{table} can be used to collapse a frequency table, or to
\SASnote
change numeric variables into character labels, suitable for the \macro{ffold}.
The required \SAS statements for this exercise are contained in \file{berk-4fold.sas}.
The first step looks like this: 
\begin{SASin}
%include catdata(berkeley);
%table(data=berkeley, out=berk2,
   var=Admit Gender,
   weight=freq,
   order=data);
%ffold(data=berk2, var=Admit Gender);
\end{SASin}

In \R, \codefun{margin.table} collapses a table over dimensions not listed
in the second argument.  The \R statements below are contained in \file{berk-4fold.R}.
See the \code{vignette("vcd-tutorial")}
tutorial, Section 3.4, for other tests and plots of these data.
\Rnote
\begin{Rin}
UCB <- aperm(UCBAdmissions, c(2,1,3))
fourfold(margin.table(UCB, c(1, 2)))     # two-way plot

fourfold(UCB, mfrow=c(2,3))              # three-way plot
woolf_test(UCB)
\end{Rin}


\subsection{Sieve diagrams}\label{sec:sieve}
The Distant-vision data records the classification of visual acuity in the left and
right eyes for a large sample of employees of the Royal Ordnance factor
in the UK during world War II.  There are two samples--- one of men and one of
women.  Here we will just look at the female sample.

We expect to find a strong degree of association (and agreement) between the
grade recorded for the two eyes.  The more interesting question is the
\emph{nature} of the association.  A sieve diagram is useful for this purpose.

\begin{enumerate*}
	\Tasks
	\item Construct a sieve diagram for these data.
	\item Make sure you understand the principal by which this diagram is constructed,
  e.g., why are the boxes for the cells of the table aligned in rows and columns.
  \item Describe the patterns you see in the density of shading.	
\end{enumerate*}

In \SAS, the data are found in the file \texttt{vision.sas}, and contain 
\SASnote
the data set \texttt{women}, for the female sample.
See the file \file{vision-sieve.sas} for this and more.

\begin{SASin}
%include catdata(vision);
proc print data=women;
   run;
title 'Vision data: Women';
%sieveplot(data=women, var=right left, filltype=obsp);
\end{SASin}
	


For \R, you will find the \texttt{VisualAcuity} data in the \pkg{vcd}.
\Rnote
In R, \codefun{subset} allows easy selection
from a data.frame, which you can use to select the female sample.
See the file \file{vision-sieve.R} for this and more.

\begin{Rin}
library(vcd)
women <- subset(VisualAcuity, gender=="female", select=-gender)
structable(~right + left, data=women)
sieve(Freq ~ right + left,  data = women, 
      gp=shading_Friendly, labeling=labeling_values)
\end{Rin}

\subsection{Observer agreement}\label{sec:agree}


Westlund \& Kurland (1953) reported data
on the diagnosis of multiple sclerosis (MS): two 
samples of patients, one from Winnipeg and one from New Orleans, were each rated 
by two neurologists (one from each city) in four diagnostic categories:
 Certain, Probable, Possible,  and Doubtful.
The questions of interest here concern how well the neurologists agree
on diagnosis for these two samples of patients.

\begin{enumerate*}
	\Tasks
	\item Obtain and test Cohen's $\kappa$ for each of the samples,
	using both equal spacing (Cicchetti-Allison) weights and inverse-square 
	(Fleiss-Cohen) weights.  What do you conclude about the strength
	of agreement in the two samples?
  \item Obtain agreement charts for each of the samples.
  Visually, how does the strength of agreement shown in these charts
  compare to the $\kappa$ measures?
  \item Is there any evidence of difference between the two
  neurologists in their frequency of use of the diagnostic
  categories (marginal heterogeneity)?
  \item Test whether the strength of
  agreement differs between the two samples (only in \SAS).
\end{enumerate*}

For \SAS, the data are contained in the file \texttt{msdiag.sas}, as a single
\SASnote
data set (\code{msdiag}) with variables \code{Patients}, \code{N_rating} and
\code{W_rating}.  
With \PROC{FREQ}, a stratified analysis by \texttt{patients}
gives you most of the information to answer the questions concerning agreement.
For agreement plots, with the \macro{agreeplot} you need to subset the data to select
each sample.
The statements below are contained in the file \file{msdiag-agree.sas}.
\begin{SASin}
%include catdata(msdiag);
*-- Agreement, separately, and controlling for Patients;
proc freq data=msdiag;
  tables patients * N_rating * W_rating / norow nocol nopct agree plots=all;
  test kappa;
  weight count;
run;

data Winnipeg;
  set msdiag;
  where (patients="Winnipeg");
%agreeplot(data=Winnipeg, var=N_rating W_rating, title=Winnipeg patients);

data NewOrleans;
  set msdiag;
  where (patients="New Orleans");
%agreeplot(data=NewOrleans, var=N_rating W_rating, title=New Orleans patients);
\end{SASin}


In \R, the data are contained in \texttt{MSPatients}, a $4 \times 4 \times 2$ table.
Patient is the third dimension, so you can use subscripting on the table to
select the two samples separately.  
The statements below are contained in the file \file{msdiag-agree.R}.
\Rnote
\begin{Rin}
library(vcd)
Kappa(MSPatients[,,1])  # use ,weights="Fleiss" for F-C weights
Kappa(MSPatients[,,2])

agreementplot(t(MSPatients[,,1]), main = "Winnipeg Patients")
agreementplot(t(MSPatients[,,2]), main = "New Orleans Patients")
\end{Rin}

\subsection{Correspondence analysis}\label{sec:corresp}
Correspondence analysis provides one way to visualize association
between variables, both for nominal and ordinal categories.
Here, we examine the relation between mental health and partents' SES.
For a $4 \times 6$ table, the total Pearson $\chi^2$ can be accounted for
in $\min(r-1, c-1)=3$ dimensions, but we hope for a simpler explanation.

\begin{enumerate*}
\Tasks
 \item Carry out a correspondence analysis of these data.  From the printed output, 
 how many dimensions seem necessary here?
 \item Interpret the results of the graph in terms of the question of
whether both mental and ses can be considered to be linearly spaced.
\end{enumerate*}


For \SAS, the statements that carry out these steps are contained in the file \file{mental-ca.sas}.
\SASnote
\begin{SASin}
%include catdata(mental);
data mental;
    set mental;
    format mental mental. ses ses.;
run;
\end{SASin}

Use the \texttt{corresp} macro to carry out the analysis and produce
the graph.
\begin{SASin}
*-- Using the corresp macro;
%corresp(data=mental, 
    tables = mental / ses,
    weight = count, htext=1.3);
\end{SASin}

In SAS 9.3, an equivalent plot can be obtained directly with \PROC{CORRESP} (via \SAS ODS Graphics).

\begin{SASin}
*-- Using SAS 9.3 ODS Graphics;
proc corresp short data=mental;
tables mental, ses;
    weight count;
    run;
\end{SASin}

For \R, the statements that carry out these steps are contained in the file \file{mental-ca.R}.
\Rnote
Begin by creating a two-way table, then use \codefun{ca} for numerical results and
\codefun{plot} to graph the result.

\begin{Rin}
library(vcdExtra)
library(ca)
Mental.tab <- xtabs(Freq ~ mental+ses, data=Mental)
ca(Mental.tab)
plot(ca(Mental.tab))
title(xlab='Dim 1', ylab='Dim 2')
\end{Rin}


\newpage

\section[Mosaic displays and \loglin\ models]{Mosaic displays and \loglin\ models}

\subsection{Mosaic displays web applet}

A web application to fit loglinear models and produce mosaic displays
is available at\linebreak 
\url{http://datavis.ca/online/mosaics/}.
It is not as flexible as \PROC{GENMOD} in \SAS or the combination of 
\codefun{loglm} and  \codefun{mosaic} in \R,
but is easy to use and contains
several sample data sets, including the Berkelely data.
Moreover, you can use it from home with your own data,
uploaded from a file, or entered into a form.

\begin{enumerate}
\Tasks
\item Navigate to the Mosaic Displays web page,
select the Berkeley Admission Data from the Sample datasets list, and press
\button{SelectData}.
\item To illustrate, we will fit the model $[AD] [GD]$ of conditional independence
(Exercise 3 in \secref{sec:loglin}), as follows:
	\begin{enumerate*}
	\item In the Variable order box enter:  \texttt{gender admit dept}.
	\item Select \texttt{CONDIT} for FitType
	\item (You may wish to explore the Analysis Options and Display Options help page.)
	\item Press \button{GetData}
	\item Interpret each display in relation to the $\chisq$ tests printed above.
	\item What happens if you click on a tile in a mosaic?
	\end{enumerate*} 
\end{enumerate}

%\subsection{Two-way tables}

\subsection[Three-way+ tables]{Three-way$+$ tables}\label{sec:loglin}
In this exercise you will exmine fitting several different \loglin\ models for the
data on admissions to graduate programes in Berkeley.
We will use the GLM approach here.%
\footnote{
As explained in the lecture, such models can be fit directly as \loglin\ models
for the data in contingency table form, or as generalized linear models (GLM)
 with a log
link and a Poisson distribution for the frequencies.
In \SAS, this amounts to the difference between \PROC{CATMOD} vs.\  \PROC{GENMOD}.
In R, the corresponding functions are \codefun{loglm} and \codefun{glm}.
}
The table variables are \texttt{dept}, \texttt{gender} and \texttt{admit},
abbreviated as D, G and A.  Note that admission is considered the response
variable here, while gender and department are explanatory, so any reasonable
model must include the $[GD]$ association between gender and department.

\begin{enumerate*}
\Tasks
	\item Fit the homogeneous association model $[AD] [GD] [AG]$ as generalized linear
model for log frequency, and obtain tests
for each term as well as the test for residual association.
Does it appear that there is an association between admission and gender?
  \item Interpret the model-- what does each term mean? What does the test for
  residual association mean?
  \item Now fit the reduced model $[AD] [GD]$ of conditional independence of admission
and gender given department, $A \perp G \given D$.
Obtain the residuals from this model and display these in a mosaic plot.
What do you conclude?
  \item It appears that conditional independence fits well, except in Department A.
Fit a model that allows an association of admission and gender only in Department A.
\end{enumerate*}

The following steps will be helpful with \SAS.  If you need more help, see the
statements in the file \file{berkeley-glm.sas}.
\SASnote
\begin{itemize*}
  \item For display purposes it is useful to apply format statements
to use as value labels for the table variables.
\begin{SASin}
%include catdata(berkeley);
*-- Apply formats to use value labels for numeric variables; 
data berkeley;
   set berkeley;
   format dept dept. admit admit. gender $sex.;
\end{SASin}
% \KileResetHL \KateResetHL
  \item The model of homogeneous association model $[AD] [GD] [AG]$ (no 3-way association)
  can be specified using \verb#"|"# notation as shown below.
\begin{SASin}
*-- Fit model of homogeneous associations:  [AD][GD][AG];
proc genmod data=berkeley;
   class dept gender admit;
   model freq = dept|gender|admit@2 / dist=poisson type3 wald;
run;
\end{SASin}
  \item The step below shows the pattern to (a) fit a model, (b)
  obtain residuals in an \code{obstats} data set and (c) produce a mosaic
  plot for the model.
\begin{SASin}
*-- Fit model of conditional independence:  [AD][GD];
proc genmod data=berkeley;
   class dept gender admit;
   model freq = dept|gender dept|admit / dist=poisson obstats;
   ods output obstats=obstats;
run;

%mosaic(data=obstats, vorder=admit gender dept, count=freq,
        resid=streschi, cellfill=dev, split=H V,
        title=Model: [AdmitDept] [GenderDept]);
\end{SASin}
\item You can fit a model allowing an association only in department A by defining
a dummy variable that causes these cells to fit perfectly.  Add this term to the 
\stmt{model} in \PROC{GENMOD} and see what happens.
\begin{SASin}
data berkeley;
   set berkeley;
   dept1AG = (gender='F') * admit * (dept=1);
\end{SASin}

\end{itemize*}

In \R, all of these models can be fit using \codefun{glm}; 
the  \codefun{loglm} function cannot handle models with special terms (like \texttt{dept1AG}),
\Rnote
but makes it quite easy to produce mosaic displays.  The statements below are in the file
\file{berkeley-glm.R}.  In \R model formulas, \verb#()^2# means ``all main effects and
interactions up to order 2''.

\begin{Rin}
library(vcd)
data("UCBAdmissions")
structable(Dept ~ Admit+Gender,UCBAdmissions)

## conditional independence in UCB admissions data
berk.mod1 <- loglm(~ Dept * (Gender + Admit), data=UCBAdmissions)
berk.mod1
mosaic(berk.mod1, gp=shading_Friendly)

## all two-way model
berk.mod2 <-loglm(~(Admit+Dept+Gender)^2, data=UCBAdmissions)
berk.mod2
mosaic(berk.mod2, gp=shading_Friendly)

# compare models
anova(berk.mod1, berk.mod2)
\end{Rin}


\subsection{Survival on the Titanic}

These exercises examine the fitting of various loglinear models
to data about survival on the \textit{Titanic}, a 4-way table
giving the cross-classification of 2201 passengers and crew,
according to 
\begin{itemize*}
 \item Gender (G): M vs. F
 \item Age (A):  Adult vs. Child
 \item Class (C):  1st, 2nd, 3rd, Crew
 \item Survival (S):  Died vs. Survived
\end{itemize*}

\begin{enumerate*}
 \item One slight complication here is 
 that there are 8 cells with zero frequencies.
Four of these (male and female children in 1st and 2nd class who died)
should be considered \emph{sampling zeros},
but 4 (children among the crew)
should probably be considered \emph{structural zeros}-- cells where data
could not occur.
In these analyses, you can treat these all as sampling zeros
by adding a small number to each cell.

 \item It is natural to consider Survival as the natural
response variable, and the remaining variables as explanatory.
Therefored, all models should include the high-order term
among Age Gender and Class.
Therefore, the minimal ``null model'' is [AGC][S],
which asserts that survival is jointly independent
of Age, Gender and Class.
Fit this model,
and obtain a mosaic plot.
Interpret the pattern of the residuals in this mosaic plot.

 \item Fit a ``main effects''
model for survival, [AGC][AS][GS][CS], that includes 
an association of survival with each of age gender and class.
Is this an adequate fit?
What does the pattern of residuals tell you about remaining
associations?

 \item 
What model would you use to allow an interaction of Age
and Gender in their effect on Survival?
Fit this model as above, and obtain the mosaic plot.

\end{enumerate*}
	
For \SAS, the statements for this exercise are in the file \file{titanic-loglin.sas},
but you should try to figure them out for yourself first.
\SASnote
The table variables are named \texttt{sex}, \texttt{age}, \texttt{class}, \texttt{survive}.
\begin{itemize*}
	\item  Adjusting for 0 cells.
\begin{SASin}
%include catdata(titanic);
data titanic;
   set titanic;
   count = count + .5;
\end{SASin}
 \item Fitting and graphing the null model. Note how the high-order term [AGC] is specified
using $|$ notation.

\begin{SASin}
title '[AGC][S]: Baseline model'; 
proc genmod data=titanic;
   class age sex class survive;
   model count=age|sex|class  survive  
           / dist=poisson type3 obstats;
   ods output obstats=obstats;

%mosaic(data=obstats, vorder=class sex age survive,
	resid=reschi, title=[AGC][S]: Baseline model);
\end{SASin}
\end{itemize*}

In \R, it is easiest here to use the \codefun{loglm} function to fit a model,
\Rnote
and the \codefun{plot} method for \loglin\ objects to obtain the mosaic plot.
The statements below are contained in the file \file{titanic-loglin.R}.
\begin{Rin}
library(vcd)
data(Titanic)

Titanic <- Titanic + 0.5   # adjust for 0 cells
titanic.mod1 <- loglm(~ (Class * Age * Sex) + Survived, data=Titanic)
titanic.mod1
plot(titanic.mod1, main="Model [AGC][S]")

titanic.mod2 <- loglm(~ (Class * Age * Sex) + Survived*(Class + Age + Sex), 
                      data=Titanic)
titanic.mod2
plot(titanic.mod2,  main="Model [AGC][AS][GS][CS]")

titanic.mod3 <- loglm(~ (Class * Age * Sex) + Survived*(Class + Age * Sex), 
                      data=Titanic)
titanic.mod3
plot(titanic.mod3,  main="Model [AGC][AS][GS][CS][AGS]")

# compare models
anova(titanic.mod1, titanic.mod2, titanic.mod3, test="chisq")
\end{Rin}

\subsection{Ordinal factors and structured associations}\label{sec:ordinal}
When some of the variables in a contingency table represent \emph{ordered} categories, 
we can usually gain both \emph{power} and \emph{parsimony} by fitting models that take ordinality
into account.  We gain power by testing a more specific hypothesis than just a general
association; we gain parsimony by fitting fewer parameters.

\subsubsection{Mental health and SES}
In this exercise, you will examine several models for Mental-Health-SES data
described in the lecture.  In this two-way table,  parents SES and child mental
health status are both ordered factors. Simple ways of handling ordinal variable
involve assigning \emph{scores} to the table categories, and the simplest
cases are to use \emph{integer} scores, either for the row variable (``column effects'' model),
the column variable (``row effects'' model), or both (``uniform association'' model).

\begin{enumerate*}
\Tasks
 \item Fit the independence model to these data.  How well does it fit?
 Make a mosaic plot showing departures (residuals) from independence.
 What do you see here that suggests a simpler description of the association
 based on ordinality?
 \item Assign integer scores to both parents SES and child mental
health status. Fit the model of uniform association.  How
well does it fit, compared to the independence model?
Make a mosaic plot showing residuals from this model, and compare
with that for the independence model.
 \item If you have time, try also fitting the column effects model,
using scores for just mental health status, and the row effects model,
using scores for just parents SES.

\end{enumerate*}

For \SAS, most of the program statements for this exercise are contained
\SASnote
in the file \file{mentgen2.sas}.  Here are a few explanatory notes:

\begin{itemize}
 \item In \PROC{GENMOD}, a given variable can be \emph{either} categorical (a factor)
--- when declared in a \stmt{CLASS}, or numeric (a quantitative covariate),
but not both.
To allow a factor to be treated as numeric in an association term,
simply make a copy of that variable under a different name:
\begin{SASin}
%include catdata(mental);
*-- Create numeric variables for row & column effects;
data mental;
   set mental;
   m_lin = mental;
   s_lin = ses;
   format mental mental. ses ses.;
\end{SASin}
 \item Fit the independence model as follows, obtaining an output \code{obstats}
data set of fitted values, residuals, etc. that can be passed to the \macro{mosaic}.
\begin{SASin}
proc genmod data=mental;
   class mental ses;
   model count = mental ses / dist=poisson obstats residuals;
   title 'Independence';
   ods output obstats=obstats;
\end{SASin}
 \item Visualize the remaining associations (residuals) with the \macro{mosaic}.
\begin{SASin}
%mosaic(data=obstats, vorder=Mental SES, resid=stresdev, 
	title=Mental Impairment and SES: Independence, split=H V, 
	cellfill=dev 0.5, htext=2.0);
\end{SASin}
 \item Include one or both of the numeric versions of the table variables in an interaction
term.  For example, the model of uniform association, with a linear $\times$ linear
association is fit and visualized as follows:
\begin{SASin}
proc genmod data=mental;
   class mental ses;
   model count = mental ses m_lin*s_lin / dist=poisson obstats residuals;
   title 'Lin x Lin';
   ods output obstats=obstats;
run;
%mosaic(data=obstats, vorder=Mental SES, resid=stresdev, 
	title=Linear x Linear, split=H V, cellfill=dev 0.5, htext=2.0);
\end{SASin}
\end{itemize}

For \R the steps are similar. We use the data set \code{Mental} in the \pkg{vcdExtra}.
\Rnote
These models are fit using \codefun{glm}, and visualized using \codefun{mosaic.glm}
applied to the \code{glm} object.  See the file \file{mental-glm.R} for these and other
steps. 
\begin{itemize}
 \item Independence model:
\begin{Rin}
library(vcdExtra)
data(Mental)
indep <- glm(Freq ~ mental+ses,
                family = poisson, data = Mental)
mosaic(indep,residuals_type="rstandard", labeling=labeling_residuals)
\end{Rin}
 \item Other models: Row effects, col effects and uniform association--
 Create numeric equivalents of the table variables, then fit models
 with interactions using the numeric scores.
\begin{Rin}
Cscore <- as.numeric(Mental$ses)
Rscore <- as.numeric(Mental$mental)

# row effects model (mental)
roweff <- coleff <- glm(Freq ~ mental + ses + mental:Cscore,
                family = poisson, data = Mental)
               
linlin <- glm(Freq ~ mental + ses + Rscore:Cscore,
                family = poisson, data = Mental)
\end{Rin}
 \item Visualize a given model with \codefun{mosaic} as shown above for the
 independence model.
\end{itemize}

\subsubsection{Distant vision: quasi-independence, quasi-symmetry}
In \secref{sec:sieve} we examined the Distant-Vision data for women with a sieve
diagram.  Here, we will consider several specialized models that
range between the independence model and the saturated model. The goal is to
find a relatively simple model, but one that accounts for the association.

The model of quasi-independence ignores the diagonal cells. 
The symmetry model tests whether the pattern of association is symmetric
around the diagonal cells.  Quasi-symmetry tests for symmetry, while allowing for
differences in the marginal frequencies.

\begin{enumerate*}
 \Tasks
 \item Fit the models of independence, and quasi-independence ignoring the diagonal cells.
 Compare the goodness of fit of the models numerically and with mosaic displays.
 \item Fit the models of symmetry and quasi-symmetry (symmetric association,
 but allowing for differences in marginal frequencies).
 Compare the goodness of fit of the models numerically and with mosaic displays. 
 \item What do you conclude?  Which model provides the \emph{simplest, adequate}
 explanation of the relation of the relation between visual acuity in the two eyes?
\end{enumerate*}

In \SAS, recall that the data set \texttt{women} has numeric factors,
\texttt{left} and \texttt{right} for the eye grades of the two eyes.
\SASnote
The key to fitting these models is to assign new variables 
corresponding to the diagonal cells \texttt{diag}, and to the
diagonally symmetric ones \texttt{diag}.  The statements below are included
in the file \file{vision-quasi.sas}, along with others not shown here.
\begin{SASin}
%include catdata(vision);

data women;
  set women;
  if left=right then do;
    diag=left;
    symmetry=-left;
    end;
  else do;
    diag=0;
    symmetry = abs(left-right);
    end;
\end{SASin}

\begin{itemize*}	
\item Independence and quasi-independence:
The pattern is (a) fit the model with \PROC{GENMOD}, (b)
obtain the \texttt{obstats} data set with residuals, (c)
plot the data and residuals from the model with the \macro{mosaic}.

\begin{SASin}
title 'Independence model (women)';
proc genmod data=women;
  class Right Left;
  model Count = Left Right /
    dist=poisson link=log obstats residuals type3;
  ods exclude obstats;
  ods output obstats=obstats;
run;
%mosaic(data=obstats, vorder=Right Left, split=H V, htext=1.9,
	resid=reschi, title=%str(Independence, G2(9)=6671.5));

title 'Quasi-independence model (women)';
proc genmod data=women;
  class Right Left diag;
  model Count = Left Right diag /
    dist=poisson link=log obstats residuals type3;
  ods exclude obstats;
  ods output obstats=obstats;
run;
%mosaic(data=obstats, vorder=Right Left, split=H V, htext=1.9,
	resid=reschi, title=%str(Quasi-Independence, G2(5)=199.1));
\end{SASin}

\item Symmetry and quasi-symmetry:
Use \texttt{model Count = symmetry}	for strict symmetry
and \texttt{model Left Right symmetry} quasi symmetry in the above.
\end{itemize*}
	
In \R, you can fit these models with \codefun{glm} or \codefun{gnm}
in the \pkg{gnm}. The \pkg{gnm} provides special functions,
\Rnote
\codefun{Diag}, \codefun{Symm} and others that considerably extend the
range of \loglin\ models for structured tables.
See the vignette, ``Generalized nonlinear models in R,''
accessible in \R as \code{vignette("gnmOverview" package="gnm")} for more details.
At present \codefun{mosaic} doesn't quite work with \emph{all} of these more general
models. The statements below are included
in the file \file{vision-quasi.R}.

\begin{Rin}
library(vcdExtra)
library(gnm)
women <- subset(VisualAcuity, gender=="female", select=-gender)

indep <- glm(Freq ~ right + left,  data = women, family=poisson)
mosaic(indep, main="Vision data: Independence (women)"  )

quasi.indep <- glm(Freq ~ right + left + Diag(right, left), 
                   data = women, family = poisson)
mosaic(quasi.indep, residuals_type="rstandard", gp=shading_Friendly,
       main="Quasi-Independence (women)"  )

symmetry <- glm(Freq ~ Symm(right, left), 
                   data = women, family = poisson)
mosaic(symmetry, residuals_type="rstandard", gp=shading_Friendly,
       main="Symmetry model (women)"  )

quasi.symm <- glm(Freq ~ right + left + Symm(right, left), 
                   data = women, family = poisson)
mosaic(quasi.symm, residuals_type="rstandard", gp=shading_Friendly,
       main="Quasi-Symmetry model (women)")
\end{Rin}
Which model is ``best''?
You can use \codefun{anova} to compare a collection of \emph{nested} models:
\begin{Rin}
# model comparisons: for *nested* models
 anova(indep, quasi.indep, quasi.symm, test="Chisq")
 anova(symmetry, quasi.symm, test="Chisq")
\end{Rin}
The \codefun{summarise} in the \pkg{vcdExtra} also gives a compact summary of goodness-of-fit
statistics for a set of related models (a \code{glmlist}).  
AIC and BIC statistics penalize larger models (more $df$), and smaller is better for both.
\begin{Rin}
# model summaries, with AIC and BIC
models <- glmlist(indep, quasi.indep, symmetry, quasi.symm)
summarise(models)
\end{Rin}

	
\newpage
\section{Logit models and logistic regression}
Logit models are just a special case of logistic regression models
in which \emph{all} predictors are discrete. The only reasons for
treating them specially are pedagogical: (a) every logit model
for a binary response is equivalent to a \loglin\ model;
(b) it is then a simple step to logistic regression models
that include continuous predictors.

\subsection{Logit models}\label{sec:logit}
It's time to have another look at the Berkeley data, from the perspective
of logit models.  In these models, with Admit as a response,
we just specify the factors that Admit depends upon (Dept?, Gender?).


\begin{enumerate*}
 \Tasks
 \item Fit a model in which Admit depends only on Dept. (The equivalent
\loglin\ model is [AD][DG], which includes the association of Department
and Gender.)
 \item Fit the main effects model, Admit $\sim$ Dept + Gender.
 Plot the observed and fitted logits under this model.
\end{enumerate*}

In \SAS, logit models are most easily fit with \PROC{CATMOD}, and graphical
\SASnote
displays of the fitted model most easily obtained with the \macro{CATPLOT}.
For this purpose, the option \code{out=predict} on the \stmt{response}
produces an output data set continaining fitted values and standard errors.
The statements below are contained in the file \file{berkeley-logit.sas},
which includes some additional plotting statements to make plots prettier.

\begin{itemize*}	
\item Effect of Dept only:
\begin{SASin}
%include catdata(berkeley);
*-- logit (Admit) ~ Dept;
proc catmod order=data data=berkeley;
   format dept dept. ;
   population dept gender;
   weight freq;
   response / out=predict;
   model admit = dept  / ml noiter noprofile title="Model (AD, DG)";
run;
\end{SASin}
\item Main effects of both Dept and Gender:
\begin{SASin}
proc catmod order=data data=berkeley;
   weight freq;
   response / out=predict;
   model admit = dept gender / ml noiter noprofile title="Model (AD, AG, DG)" ;
  run;
\end{SASin}
\item Plotting the observed and fitted values (on the logit scale) with the \macro{CATPLOT}:
\begin{SASin}
%catplot(data=predict, class=gender, x=dept, type=FUNCTION, z=1.96);
\end{SASin}
\end{itemize*}
	

In \R, the parallel analysis as a logit model is easiest if the
data \code{UCBAdmissions} is first be converted from a $2 \times 2 \times 6$ table
\Rnote
to a \texttt{data.frame} in frequency form.  Models can then be
expressed for the variable \texttt{Admit}, using the \texttt{Freq}
variable as a weight.  The statements below are contained in the file
\file{berkeley-logit.R}.
\begin{itemize*}	
\item Effect of Dept only:
\begin{Rin}
library(car)    # for Anova()
data(UCBAdmissions)
UCB.df <- as.data.frame(UCBAdmissions)
berk.mod1 <- glm(Admit=="Admitted" ~ Dept, data=UCB.df, 
                 weights=UCB.df$Freq, family="binomial")
Anova(berk.mod1, test="Wald")
summary(berk.mod1)
\end{Rin}
\item Adding a main effect of gender:

\begin{Rin}
berk.mod2 <- glm(Admit=="Admitted" ~ Dept+Gender, data=UCB.df, 
                 weights=UCB.df$Freq, family="binomial")
Anova(berk.mod2, test="Wald")
summary(berk.mod2)
\end{Rin}

\item Plotting fitted values.  The \pkg{effects} is explored in more detail later in this section.
\begin{Rin}
library(effects)   ## load the effects package
berk.eff2 <- allEffects(berk.mod2)
plot(berk.eff2)
plot(effect('Dept:Gender', berk.mod2), multiline=TRUE)
\end{Rin}
\end{itemize*}	

\textbf{Note}: The \pkg{effects} issues Warnings when you ask to plot a higher-order term not contained
in the model.  This is OK; the correct plots should be produced.

\subsection{Logistic regression}\label{sec:logistic}

\subsubsection{Arthritis treatment data}\label{sec:arth}
In previous lectures, I used the Arthritis data, but always ignoring
the covariate Age, so I could treat it as a contingency table,
Sex $\times$ Treatment  $\times$ Improve.  Here, we will fit
logistic regression models, but only for a binary response,

$$\textrm{better} = \left\{ 
	\begin{array}{ll}
   {0,\mbox{  improve = 'None'}}  \\
   {1,\mbox{  improve = 'Some','Marked'}}  \\
\end{array} \right.
$$

\begin{enumerate*}
 \Tasks
 \item Fit a logistic model predicting $\Pr (better=1)$ from Age alone.
 Make a graph to visualize how better changes with Age.
 \item Add the effects of Sex and and Treatment to this model.  
 Make sure you understand how these effects are parameterized in your model.
 Interpret the coefficients for Age, Sex and Treatment on the probability
 of improvement.
 \item Make a reasonable graph to visualize the effects of Age, Sex and Treatment
 on the probability of improvement.
 \item Consider the possibilities there may be two-way interactions among
  Age, Sex and Treatment and the  effect of Age may be non-linear
  (e.g., Age, Age$^2$).
  Fit a model that tests for these additional terms over and above the
  main effects.
\end{enumerate*}

For \SAS, the variable \code{better} has been defined in the
data step that creates the \code{arthrit} data set.
\SASnote
Use the option \code{descending} to model the probability \code{better=1}.
In SAS 9.3, the \stmt{effectplot}\  produces reasonable plots either on the
probability scale or logit scale.  
The statements below 
are contained in the file \file{arthritis-logistic.sas}.

\begin{itemize*}	
\item Logistic regression on age alone:
\begin{SASin}
%include catdata(arthrit);
title 'Simple logistic regression on age';
proc logistic data=arthrit descending;
   model  better = age ;
   effectplot fit / obs(jitter(y=0.02));
   effectplot fit / obs(jitter(y=0.02)) link;  * display on logit scale;
run;
\end{SASin}
\item Fit the main effects model.  Note the use of the \code{ref=} option
to assign reference categories for Sex and Treat, and the \stmt{output}
to obtain a data set for plotting with the \macro{meanplot}.
\begin{SASin}
title 'main effects model';
proc logistic data=arthrit descending;
   class sex (ref=last) treat (ref=first) / param=ref;
   model  better = sex  treat  age ;
   output out=results p=prob l=lower u=upper
          xbeta=logit stdxbeta=selogit / alpha=.33;
   effectplot slicefit (sliceby=treat) / at(sex=all) clm alpha=0.33;
   effectplot interaction (x=treat sliceby=sex) / at(age=30 60) clm alpha=0.33 noobs;
	run;
*-- or use meanplot macro;
%meanplot(data=results, response=prob, class=age treat sex, pmean=no);
\end{SASin}
\item To test potential high-order terms, you can use forward selection,
starting with the main effects model.
\begin{SASin}
title 'screening for higher-order effects';
proc logistic data=arthrit descending;
   class sex(ref=last) treat(ref=first) / param=ref;
   model  better = age sex treat
                   age | sex | treat @2  age*age
                   / selection=forward
                     slentry=0.1   /* be a bit liberal */
                     start=3 ;     /* Start after all main effects */
   effectplot slicefit (sliceby=treat) / at(sex=all)clm alpha=0.33;
   title2 'Testing all interactions via forward selection';
	run;
\end{SASin}
\end{itemize*}
	
In \R, use the \code{Arthritis} data in the \pkg{vcd}.
The variable \code{Better} must be calculated as shown below
\Rnote
from \code{Improved}.
Logistic regression models are fit with \codefun{glm},
and fitted values can be obtained with \codefun{predict} for the \code{glm} object.
The statements below are contained in the file \file{arthritis-logistic.R}.

\begin{itemize*}	
\item Logistic regression on age alone:
\begin{Rin}
library(vcd)
library(car)
data(Arthritis)

# define Better
Arthritis$Better <- Arthritis$Improved > 'None'

# simple linear regression
arth.mod0 <- glm(Better ~ Age, data=Arthritis, family='binomial')
anova(arth.mod0)

# plot, with +-1 SE
plot(Better ~ Age, data=Arthritis, ylab="Prob (Better)")
pred <- predict(arth.mod0, type="response", se.fit=TRUE)
ord <-order(Arthritis$Age)

lines(Arthritis$Age[ord], pred$fit[ord], lwd=3)
upper <- pred$fit + pred$se.fit
lower <- pred$fit - pred$se.fit
lines(Arthritis$Age[ord], upper[ord], lty=2, col="blue")
lines(Arthritis$Age[ord], lower[ord], lty=2, col="blue")
\end{Rin}
%
\item Fitting the main effects model:
\begin{Rin}
# main effects model
arth.mod1 <- glm(Better ~ Age + Sex + Treatment , data=Arthritis, 
                 family='binomial')
Anova(arth.mod1)

# same, using update()
arth.mod1 <- update(arth.mod0, . ~ . + Sex + Treatment)
Anova(arth.mod1)
\end{Rin}

\item Forward selection:
\begin{Rin}
# forward selection from the main effects model
step(arth.mod1, direction="forward", scope=.~ (Age+Sex+Treatment)^2 + Age^2)
\end{Rin}
\end{itemize*}
	
\subsubsection{Volunteering for psychology experiments}\label{sec:cowles}
Cowles and Davis (1987) examined personality factors that might 
relate to whether an individual volunteered to participate in a psychology
experiment.  The predictors used here are scales of Neuroticism and Extraversion
from the Eysenck personality inventory, plus Sex of the participant.
  There are
1421 observations.

\begin{enumerate*}
 \Tasks
 \item Fit a main effects model predicting Pr (Volunteer = "Yes") from Sex, 
 Neuroticism and Extraversion.  How well does this model fit?
 \item Fit a model that includes all two-way interactions among
 Sex, Neuroticism and Extraversion.  How well does this model fit?
 Which interaction terms appear not to contribute to prediction?
 \item Fit a reduced model including only the important interaction(s)
 from the all two-way model.  What do you conclude?
\end{enumerate*}
	
In \SAS, the data are available in \texttt{cowles.sas} and some statements for
\SASnote
this exercise in the file \file{cowles-logistic.sas}. The statements below
will get you started with the main effects model. Note the use of the 
\texttt{descending} option to model the event \texttt{Volunteer=1} (``yes'').
For logistic models, the \texttt{lackfit} option on the \stmt{model} \ 
gives the Hosmer-Lemeshow lack of fit test

\begin{SASin}
%include catdata(cowles);
*-- apply formats to Sex and Volunteer;
data cowles;
  set cowles;
  format Sex sex. Volunteer volun.;
proc print data=cowles(obs=20);
  run;
	
*-- main effects model;
proc logistic data=cowles descending ;
  class Sex;
  model Volunteer = Sex  Extraver  Neurot /  lackfit ;
  run;
\end{SASin}

For other models, you can use \verb#|# notation in the 	\stmt{model};
here, \verb#@# forms all high-order terms up to a given degree.
Thus, the all two-way model can be specified as
\begin{SASin}
*-- all two-way model;
proc logistic data=cowles descending ;
  class Sex;
  model Volunteer = Sex | Extraver | Neurot @2 /  lackfit ;
  run;
\end{SASin}


For \R, the data frame \texttt{Cowles} is contained in the \pkg{car},
\Rnote
which also has an \codefun{Anova} function providing Type II and Type III
tests, more useful than the sequential Type I tests provided by
\codefun{anova}.  The \R statements for this exercise are provided in the
file \file{cowles-logistic.R}. For the main effects model, the 
commands are shown below.

\begin{Rin}
library(car)       ## for Anova: type II tests
data(Cowles)
mod.cowles0 <- glm(volunteer ~ sex + neuroticism + extraversion, 
    data=Cowles, family=binomial)
summary(mod.cowles0)
Anova(mod.cowles0)
\end{Rin}

\subsubsection{Marijuana arrests data}\label{sec:arrests}
In this exercise you will fit and examine graphically a more realistic (and complex) model.
The data concern police treatment of individuals arrested in Toronto for  simple
possession of small quantities of marijuana.%
\footnote{
This was part of a larger data set on racial profiling, featured
in  a series  of articles  in the  \emph{Toronto Star}  newspaper, December, 2002.}
Under  these
circumstances police have the option of releasing an arrestee with a summons  to
appear in court � similar to a traffic ticket; alternatively, the individual may
be brought to  the police station  for questioning and  possible indictment. 

The
principal question of interest is whether and how the probability of release  is
influenced by the subject�s race, age, and other characteristics. Beyond this,
there is also interest in whether race interacts with other variables.

There are 5226 observations. The variables are:
\begin{description*}
\item[\texttt{released}] Whether the arrestee was released with a summons: ``No'' or ``Yes'' 
\item[\texttt{colour}] The arrestee's race: ``Black'', ``White''. 
\item[\texttt{year}] 1997 through 2002. Although numeric, we will treat \texttt{year} as a
factor here.
\item[\texttt{age}] in years
\item[\texttt{sex}] ``Female'', ``Male''. 
\item[\texttt{employed}] ``No'', ``Yes'' 
\item[\texttt{citizen}] ``No'', ``Yes''
\item[\texttt{checks}]
Number of police data bases (of previous arrests, previous convictions, parole status, etc. -- 6 in all) on which the arrestee's name appeared
\end{description*}

\begin{enumerate*}
\Tasks
\item Fit a main effects logistic regression model predicting \texttt{released} from
all the other variables.
\item Fit a model that includes all possible two-way interactions as well as all main effects.
Which two-way interactions appear to be important?
\item Fit a reduced model that includes only the important two-way interactions as well as 
all main effects. Methods for model selection is a complex topic--- here, we will only
use backward selection from the all two-way model to illustrate one simple approach.
\end{enumerate*}


For \SAS, the data are read from the file \texttt{arrests.sas}, and some code for
\SASnote
this exercise is in the file \file{Arrests-logistic.sas}.
\begin{SASin}
%include catdata(arrests);
*-- main effects model;
proc logistic data=arrests;
        class colour year sex citizen employed;
        model released = colour year age sex employed citizen checks;
        run;

*-- all two-way model, with backward elimination;
proc logistic data=arrests;
        class colour year sex citizen employed;
        run;
        model released = colour | year | age | sex | employed | citizen | checks@2 /
              selection=backward;
        run;
\end{SASin}


For \R, the \texttt{Arrests} data are contained in the \pkg{effects}, and
\Rnote
some statements for this exercise are contained in the file \file{Arrests-logistic.R}.
Note that, in model formulas, \verb|~.| stands for ``is modeled by all predictors'';
\verb|~.^2| is a short-hand for ``all main effects plus two-way terms.''
\begin{Rin}
library(effects)    # for Arrests data
library(car)        # for Anova()
data(Arrests)
Arrests$year <- as.factor(Arrests$year)

# all main effects
arrests.mod1 <- glm(released ~ ., family=binomial, data=Arrests)
Anova(arrests.mod1)

# all two-way effects
arrests.mod2 <- glm(released ~ .^2, family=binomial, data=Arrests)
Anova(arrests.mod2)
\end{Rin}

To eliminate unnecessary two-way terms, try \codefun{stepAIC},
that tries to minimize the AIC statistic.  
\begin{Rin}
# backward selection, using AIC
arrests.step <- stepAIC(arrests.mod2, direction="backward")

anova(arrests.mod1, arrests.step, arrests.mod2, test="Chisq")
\end{Rin}
	


\subsection{Effect plots}\label{sec:effect}
Effect plots are among the most useful graphical displays for understanding 
the pattern of effects in complex models. 
As explained in the lecture, the idea is relatively simple:  to visualize
a high-order term in a model, plot the predicted values (``adjusted means'')
for that term,
averaging over all other factors \emph{not included} in that term.

\subsubsection{Volunteering for psychology experiments}\label{sec:cowles2}

\begin{enumerate*}
 \Tasks
 \item Construct an effects plot to visualize the Neuroticism * Extraversion
 interaction for the Cowles data.
 \item Give a verbal description of \emph{how} probability of volunteering changes
 with both neuroticism and extraversion.
\end{enumerate*}
	
In \SAS 9.3, effect plots are easily provided by the \stmt{effectplot}
\SASnote
The program statements for
this exercise are contained in the file \file{cowles-effect.sas}.
\begin{SASin}
%include catdata(cowles);
proc logistic data=cowles outest=parm  descending ;
    class Sex;
	model Volunteer = Sex   Extraver | Neurot / lackfit ;
	effectplot slicefit(x=Extraver sliceby=Neurot) / at(sex=1.5) noobs;
	effectplot slicefit(x=Neurot sliceby=Extraver) / at(sex=1.5) noobs;
	effectplot contour(x=Neurot y=Extraver) / at(sex=1.5) noobs;
	run;
\end{SASin}

In \R, effect plots for a wide class of linear models are provided in the
\pkg{effects}.  The function \codefun{allEffects} creates the predicted
effect values for all high-order terms in the model.  
These can then be plotted with the \codefun{plot} method.
\Rnote
The statements below are contained in the file \file{cowles-effect.R}.
They show two different forms of effect plots for the \texttt{neuroticism:extraversion}
effect.

\begin{Rin}
library(effects)   ## load the effects package
data(Cowles)
mod.cowles <- glm(volunteer ~ sex + neuroticism*extraversion, 
    data=Cowles, family=binomial)
summary(mod.cowles)

eff.cowles <- allEffects(mod.cowles, 
	xlevels=list(neuroticism=seq(0, 24, 6), 
               extraversion=seq(0, 24, 8)))
#-- separate panels
plot(eff.cowles, 'neuroticism:extraversion', ylab="Prob(Volunteer)",
    ticks=list(at=c(.1,.25,.5,.75,.9)), layout=c(4,1))
#-- multiline plot
plot(eff.cowles, 'neuroticism:extraversion', multiline=TRUE, 
    ylab="Prob(Volunteer)", 
    key.args=list(x = .8, y = .9))
\end{Rin}

\subsubsection{Marijuana arrests data}\label{sec:arrests2}
Here we examine effect plots for high-order terms in one model
for \texttt{released} in the \texttt{Arrests} data.
This model allows for interactions of \texttt{color}
with both \texttt{year} and \texttt{age}.  As mentioned above,
these plots are easy to do with the \pkg{effects} in \R,
but more difficult in \SAS.

\begin{enumerate*}
 \Tasks
 \item Fit the model that includes terms for \texttt{colour*year} and \texttt{colour*age},
 as well as other variables as main effects.
 \item Obtain an effect plot for the 3-way term \texttt{colour:year:age} 
 (not included in the model).  Interpret this plot in relation to how 
 the probability of release varies with both Year and Age.
 \item Obtain separate effect plots for the 2-way terms \texttt{colour:age}
 and \texttt{colour:year}
\end{enumerate*}

In \SAS 9.3, we can again use the \stmt{effectplot}.  Note how the various options
control the details. The statements below are contained in 
\SASnote
the file \file{Arrests-effect.sas}.
\begin{SASin}
proc logistic data=arrests descending;
   class colour year sex citizen employed;
   model released = 
         colour|year colour|age sex employed citizen checks / clodds=wald;
   effectplot interaction (x=year sliceby=colour) / clm alpha=0.33 noobs yrange=clip;
   effectplot slicefit (x=age sliceby=colour) / 
         clm alpha=0.33 obs(fringe jitter) yrange=(.7, 1);
   run;
\end{SASin}

The statements below are contained in 
\Rnote
the file \file{Arrests-effects.R}.
Fit the model:
\begin{Rin}
library(effects)
data(Arrests)
Arrests$year <- as.factor(Arrests$year)

arrests.mod <- glm(released ~ employed + citizen + checks
       + colour*year + colour*age, family=binomial, data=Arrests)
\end{Rin}
	
Three-way effect plot:
\begin{Rin}
plot(effect("colour:year:age", arrests.mod, xlevels=list(age=15:45)),
  multiline=TRUE, ylab="Probability(released)", rug=FALSE)
\end{Rin}

Two-way effect plots. Note the difference between the style of plot with
\texttt{multiline=TRUE} and \texttt{multiline=FALSE}.
\begin{Rin}
plot(effect("colour:age", arrests.mod),
      multiline=TRUE, ylab="Probability(released)")

plot(effect("colour:age", arrests.mod),
      multiline=FALSE, ylab="Probability(released)")
\end{Rin}

For complex models, a convenient feature of the \pkg{effects} is the ability to 
obtain the necessary statistics for \emph{all effects} simultaneously,
and plot various terms by selection from a menu.

\begin{Rin}
arrests.effects <- allEffects(arrests.mod, xlevels=list(age=seq(15,45,5)))
plot(arrests.effects, ylab="Probability(released)")
\end{Rin}

	
\subsection{Diagnostic plots}\label{sec:diag}
In this exercise, the goal is to explore diagnostic plots for generalized linear models,
which are simple extensions of similar plots for ordinary linear models
(regression and ANOVA).
We return to the Berkeley data one more time.

\begin{enumerate*}
\Tasks
\item Fit the model $[AD] [GD]$ of conditional independence, as before.
\item Construct plots to show the distribution of residuals from this model
and to identify influential cells in terms of residual and leverage.
\item Interpret what you see in these plots in relation to other plots of these
data in previous exercises.
\end{enumerate*}

In \SAS, you can use the \texttt{inflglim} and \texttt{halfnorm} macros,
\SASnote
as well as the ODS Graphics facility in \SAS 9.2+ to obtain a wide variety of
diagnostic plots. See \file{berkeley-diag.sas}.

\begin{itemize*}
 \item Using the \texttt{inflglim} and \texttt{halfnorm} macros.
For these plots, begin by constructing a character variable that joins the
factor levels into a single cell variable for labels in the plots.
\begin{SASin}
%include catdata(berkeley);
*-- make a cell ID variable, joining factors;
data berkeley;
   set berkeley;
   cell = trim(put(dept,dept.)) ||
        gender ||
          trim(put(admit,yn.));
run;
\end{SASin}
\begin{SASin}
*-- Conditional independence;
proc genmod data=berkeley;
   class dept gender admit;
   model freq = dept|gender dept|admit / dist=poisson obstats residuals;
   ods output obstats=obstats;
\end{SASin}
\item Use the \texttt{inflglim} macro to produce influence plots for this
model.
\begin{SASin}
%inflglim(data=berkeley, class=dept gender admit,
        resp=freq, model=dept|gender dept|admit, dist=poisson, id=cell,
        gx=hat, gy=stresdev);

%inflglim(data=berkeley, class=dept gender admit,
        resp=freq, model=dept|gender dept|admit, dist=poisson, id=cell,
        gx=hat, gy=difdev);
\end{SASin}

\item Use the \texttt{halfnorm} macro to produce a half-normal residual plot for this
model, with a simulated 95\% envelope for the mean response in the model.
\begin{SASin}
%halfnorm(data=berkeley, class=dept gender admit,
   resp=freq, model=dept|gender dept|admit, dist=poisson, id=cell);
\end{SASin}


 \item Using ODS graphics with \code{plot=all}.  This only works in 
 \SAS  9.2+, and the plots are all index plots-- variable plotted against observation
 (cell) number.  
\begin{SASin}
proc genmod data=berkeley plots=all;
   class dept gender admit;
   model freq = dept|gender dept|admit / dist=poisson;
run;
\end{SASin}
	
\end{itemize*}
	

In \R, some of these plots can be obtained from the \codefun{plot} method for
\texttt{glm} objects.  An influence plot is provided by \codefun{influencePlot}
in \pkg{car}.  See \file{berkeley-diag.R}.
\Rnote
\begin{Rin}
library(car)
berkeley <- as.data.frame(UCBAdmissions)
cellID <- paste(berkeley$Dept, substr(berkeley$Gender,1,1), '-', 
                substr(berkeley$Admit,1,3), sep="")
rownames(berkeley) <- cellID

berk.mod <- glm(Freq ~ Dept * (Gender+Admit), data=berkeley, family="poisson")
influencePlot(berk.mod, labels=cellID, id.n=3)
plot(berk.mod)
\end{Rin}
%	


\newpage

\section{Polytomous response models}

\subsection{Proportional odds model}\label{sec:propodds}

\subsubsection{Arthritis data}
In this exercise you will try to fit a proportional odds model to
the Arthritis data, modeling the 3-category response
Improve (``None'', ``Some'', or ``Marked'')
and using Sex, Treatment and Age as predictors.

\begin{enumerate*}
 \Tasks
 \item Fit a main effects model predicting \texttt{improve} from Sex, Treatment and Age.
 Note that both Sex and Treatment are binary factors (CLASS variables in SAS),
 so the interpretation of the coefficients depends on how they are coded.
 \item How well does the proportional odds model fit for these data?%
\footnote{
Caution: The score test for the proportional odds model is known to be
anti-conservative, i.e., it yields $p$-values that are too small in many
cases.
} 
 \item Interpret the coefficients for Sex, Treatment and Age in this model.
 For example, how would you interpret the coefficient for Sex in relation to the
 odds of a better outcome?
 \item Obtain and plot predicted (fitted) values on either the probability scale
 or the logit scale.
 Interpret this plot in relation to the numerical results (coefficients, odds ratios, etc.)
\end{enumerate*}
 
 For \SAS, the steps are illustrated in the lecture notes, ``Proportional odds model:
 Fitting and plotting.''
 \SASnote
 Below, I show just the steps to fit the model, and obtain an output data set with the
 fitted probabilities and logits.  Note that Sex is coded so that Male is the reference
 category, and Treat so that Placebo is the reference category.
 The additional steps to produce nicely formatted plots
 are contained in the file \file{arthritis-propodds.sas}.

\begin{SASin}
title 'Logistic Regression: Proportional Odds Model';
%include catdata(arthrit);
 
proc logistic data=arthrit descending;
   class sex (ref=last) treat (ref=first) / param=ref;
   model  improve = sex  treat  age ;
   output out=results p=prob l=lower u=upper
          xbeta=logit stdxbeta=selogit / alpha=.33;

proc print data=results(obs=6);
   id id treat sex;
   var improve _level_ prob lower upper logit;
   format prob lower upper logit selogit 6.3;
run;
\end{SASin}

In SAS 9.2+ it is easier to plot results using ODS graphics and the \stmt{effectplot}.
The code below is contained in \file{arthritis-propodds-ods.sas}
\begin{SASin}
proc logistic data=arthrit descending  ;
   class sex (ref=last) treat (ref=first) / param=ref;
   model  improve = sex  treat  age / clodds=wald expb;
   effectplot slicefit(sliceby=improve plotby=Treat) / at(sex=all) clm alpha=0.33 ;
   effectplot interaction(x=Treat sliceby=improve) / at(sex=all) noobs clm alpha=0.33;
run;
\end{SASin}

 
In \R, we can fit the proportional odds model with \codefun{polr} in the \pkg{MASS}. 
 \Rnote
There is no simple equivalent for the score test of the proportional odds assumption.
The statements below are contained in the file \file{arthritis-propodds.R}.

\begin{Rin}
library(vcd)
library(car)         # for Anova()

arth.polr <- polr(Improved ~ Sex + Treatment + Age, data=Arthritis)
summary(arth.polr)
Anova(arth.polr)      # Type II tests
\end{Rin}

In \R it is easy to obtain effect plots showing the fitted probabilities
in relation to any of the terms in the model (or even terms \emph{not}
in the model).
	
\begin{Rin}
library(effects)
arth.effects <- allEffects(arth.polr, xlevels=list(age=seq(15,45,5)) )
plot(arth.effects)  # visualize all main effects

# plot terms not in model
plot(effect("Sex:Age", arth.polr))
plot(effect("Treatment:Age", arth.polr))
\end{Rin}

\subsection{Nested dichotomies}\label{sec:nested}
This exercise concerns data on women's labour-force participation
collected at York University in a
1977 survey of the Canadian population.
The three-level response has categories Not working outside the home ($n=155$),
working Part-time ($n=42$) and working Full-time ($n=66$).
The predictors are Husband's income (\$1000s), Presence of children in the household,
and Region of Canada.  Region turns out not to be an important predictor, so we will
ignore it here.

\begin{enumerate*}
 \Tasks
 \item The first question is whether the proportional odds model provides a reasonable
 account of these data--- if so, it would give a simple description.
 Fit the proportional odds model, and test whether the proportional odds assumption
 is tenable (only in \SAS).
 \item Fit separate logistic models for the two nested dichotomies:
 	\begin{itemize*}
    \item Working (full- or part-time) vs.\ Not Working; and
    \item Working Full-time vs.\ Part-time.
\end{itemize*}
 \item Interpret the coefficients in the two models.  What do they say about the
 effect of Husband's income and Children on the probabilities for the two dichotomies?
 \item Plot the fitted probabilities for the 3 categories of the response in relation
 to Husband's income and Children.
\end{enumerate*}

For \SAS, the data are contained in the data set \code{wlfpart}. The 3-level response here
\SASnote
is \texttt{labor}, with values 1 (Full-time), 2 (Part-time), 3 (Not working).
When the data are read in, two dichotomous responses are created:
\texttt{working = labor < 3}, and \texttt{fulltime = labor=1} (but just for
working women.  The statements below (and more) are contained in the file
\file{wlf-nested.sas}.

Fitting the proportional odds model:
\begin{SASin}
%include catdata(wlfpart);
proc logistic data=wlfpart nosimple descending;
   model labor  = husinc children ;
   title2 'Proportional Odds Model for Fulltime/Parttime/NotWorking';
run;
\end{SASin}

Fitting and plotting the models for the nested dichotomies. Note how an
\stmt{ODS OUTPUT} can be used to capture the global test statistics.
See the \file{wlf-nested.sas} file for how to combine these to give
overall tests for the 3-level nested dichotomy model.
\begin{SASin}
proc logistic data=wlfpart nosimple descending;
   model working = husinc children ;
   output out=resultw p=predict;
   ods output GlobalTests=gtests1;
   title 'Nested Dichotomies';
proc plot;
   plot predict * husinc = children;

proc logistic data=wlfpart nosimple descending;
   model fulltime = husinc children ;
   output out=resultf p=predict;
   ods output GlobalTests=gtests2;
proc plot;
   plot predict * husinc = children;
run;
quit;
\end{SASin}



In \R, the data are in the data frame \texttt{Womenlf} in the \pkg{car}.
\Rnote
The 3-level response is called \texttt{partic} here, an unordered factor
with values
\texttt{fulltime},  \texttt{not.work} and  \texttt{parttime}.
One easy way to create the two dichotomous responses is with the
\codefun{recode} function in \pkg{car}.  
The file \file{wlf-nested.R} contains the statements below,
and also shows how to produce a plot of the fitted values.

\begin{Rin}
library(car)
data(Womenlf)
attach(Womenlf)
working <- recode(partic, " 'not.work' = 'no'; else = 'yes' ")
fulltime <- recode (partic, 
    " 'fulltime' = 'yes'; 'parttime' = 'no'; 'not.work' = NA")
\end{Rin}
Fitting nested dichotomies models:  We use treatment contrasts
for \texttt{children}, so the coefficient represents the
effect for having children.

\begin{Rin}
contrasts(children)<- 'contr.treatment'
mod.working <- glm(working ~ hincome + children, family=binomial)
summary(mod.working)
mod.fulltime <- glm(fulltime ~ hincome + children, family=binomial)
summary(mod.fulltime)
Anova(mod.working)  
Anova(mod.fulltime)
\end{Rin}

Obtaining predicted values for plotting:
The general idea is to create a data frame with the combinations of the
predictor values you want to see, and use the \codefun{predict} method
to get the fitted values.  For binomial \codefun{glm} models, \code{type='response'}
gives fitted values on the probability scale.%
\footnote{
The default, \code{type='link'}, gives fitted values on the logit scale.
}
\begin{Rin}
# get fitted values for both sub-models
predictors <- expand.grid(hincome=1:45, children=c('absent', 'present'))

p.work <- predict(mod.working, predictors, type='response')
p.fulltime <- predict(mod.fulltime, predictors, type='response')
\end{Rin}

One slight complication:  Because the model for the \code{fulltime} dichotomy
is \emph{conditional} on working, you need to do some calculations to obtain the
\emph{unconditional} probabilities for each of the three response categories.
\begin{Rin}
# calculate unconditional probs for the three response categories
p.full <- p.work * p.fulltime
p.part <- p.work * (1 - p.fulltime)
p.not <- 1 - p.work
\end{Rin}


\subsection{Generalized logits}\label{sec:genlogit}
The generalized (or multinomial) logit model is an alternative to
using nested dichotomies for a polytomous reponse.
One advantage is that you can fit this as a single model, without
resorting to the special tricks used for  nested dichotomies.

In using the generalized logit model, you should take care to
specify the baseline response category, so you understand 
how to interpret the fitted coefficients.  As always, plots of
the fitted model (probabilities or logits) facilitate 
interpretation.

In this exercise, we continue with the data on women's labor
participation from \secref{sec:nested}, but now fit a 
generalized logit model.

\begin{enumerate*}
 \Tasks
 \item Fit the generalized logit model, using Not working as the
 baseline response category.
 \item Examine the coefficients for Husband's income and Children
 in the comparisons of Part-time and Full-time vs. Not working.
 On which comparison to the predictors show a greater impact?
 \item Plot the fitted probabilities of all three response categories
 over the range of Husbabd's income for both levels of Children.
 Compare this plot to the analogous plot for the nested dichotomies model.
\end{enumerate*}

In \SAS, you can fit the generalized logit model for the polytomous
response \code{labor} simply by specifying the option \code{link=glogit}
\SASnote
on the \stmt{model}. Use the \stmt{output} to obtain fitted values,
both on the probability scale \code{p=predict} and on the logit scale
\code{xbeta=logit}.

\begin{SASin}
*-- Fit generalized logit model;
proc logistic data=wlfpart;
   model labor = husinc children / link=glogit;
   output out=results p=predict xbeta=logit;
\end{SASin}

In the output data set \code{results}, the three response categories are
identified by a variable \code{_level_}.  The simplest plot of the
fitted probabilities is produced with \PROC{PLOT} as shown below.
See the file \file{wlf-glogit.sas} for  details on plotting the fitted
results with \PROC{GPLOT} and better labels for the curves.

\begin{SASin}
proc sort data=results;
	by children husinc _level_;

*-- simple plot;
proc plot data=results;
	plot predict * husinc = _level_ ;
	by children;
\end{SASin}

	

For \R, it is helpful to first turn \code{partic} into an ordered factor,
so that Not working is the baseline category.
\Rnote
The multinomial logit model cannot be fit directly with \codefun{glm}; it
\emph{can} be fit using the \codefun{multinom}
function in the \pkg{nnet}, which also requires the \texttt{MASS}
library.  See the file \file{wlf-glogit.R} for more details on plotting the fitted
results.

Fitting the multinomial logit model:

\begin{Rin}
library(car)
data(Womenlf)
attach(Womenlf)
participation <- ordered(partic, 
    levels=c('not.work', 'parttime', 'fulltime'))
library(nnet)
library(MASS)
mod.multinom <- multinom(participation ~ hincome + children)
summary(mod.multinom, cor=F, Wald=T)
Anova(mod.multinom)
\end{Rin}

Obtaining predicted values for plotting:  Here, we get fitted values on the
probability scale with \code{type='probs'}.
\begin{Rin}
predictors <- expand.grid(hincome=1:45, children=c('absent', 'present'))
p.fit <- predict(mod.multinom, predictors, type='probs')
\end{Rin}
	
\bibliography{graphics}

\end{document}
