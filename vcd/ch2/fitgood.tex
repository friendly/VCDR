\subsection{The \macro{GOODFIT}}
The \macro{GOODFIT} (see \macref{mac:goodfit}) carries out Pearson \chisq{} and \LR{} goodness-of fit tests
for the uniform, binomial, Poisson, negative binomial,
 logarithmic series, and geometric distributions,
as well as any discrete (multinomial) distribution whose probabilities you can specify.
The data may consist either of individual observations on a single
variable, or a grouped frequency distribution in the form
shown in \tabref{tab:horskick}.
The parameter(s) of the distribution may be specified as constants
or may be estimated from the data.

\begin{comment}  %%% begin stuff deleted
The macro is used as follows%
\footnote{In subsequent descriptions of macros in the text
we simply give references to the documentation provided in Appendix \ref{ch:macros} and provide examples of usage.}%
:
\aunote{Should we take this out?}
\begin{listing}
\%goodfit(data=\emph{SASdatasetname},
   var=\emph{variablename},
   freq=\emph{variablename},
   dist=\emph{distribution},
   parm=\emph{parameters},
   sumat=\emph{value},
   format=\emph{SASformat},
   out=\emph{outputdatasetname},
   outstat=\emph{statisticsdatasetname});
\end{listing}
\end{comment}  
The macro parameters are described in \macref{mac:goodfit}.
We illustrate its use in \exref{ex:weldon} and \exref{ex:federalist} below.

\begin{Example}[weldon]{Weldon's dice}
The data from \tabref{tab:dice}
can be fit to a binomial distribution as shown below.
Note that, because the frequencies have been lumped for 10--12
successes, it is necessary to
\begin{seriate}
\item Input frequencies for all values of $k = 0, \dots, 12$,
using missing values for the frequencies beyond $k=10$;
\item specify \texttt{sumat=10} in the macro call.
\end{seriate}
%% table from data set dice (dice.sas) generated 26DEC97
\begin{table}[htb]
\caption{Frequencies of a 5 or 6 in throws of 12 dice}
\label{tab:dice}
 \begin{center}
  \begin{tabular}{rr}
  \hline
Number of  & Frequency \\
5s or 6s ($k$) & ($n_k$) \\
  \hline
0 & 185 \\
1 & 1149 \\
2 & 3265 \\
3 & 5475 \\
4 & 6114 \\
5 & 5194 \\
6 & 3067 \\
7 & 1331 \\
8 & 403 \\
9 & 105 \\
10+ & 18 \\
    & N=26306 \\
  \hline
  \end{tabular}
 \end{center}
\end{table}


The first call to the \macro{GOODFIT} fits the binomial distribution
with parameter $p = \frac13$, assuming the dice to be fair,
and produces the output shown in \outref{out:dice.1} and \outref{out:dice.2}.
The \chisq{} statistics indicate that the fit is poor,
and the pattern of residuals
suggests that $p > \frac13$
(the observed frequencies for larger values of $k$ are all
greater than the expected frequencies).
\begin{Output}
\caption{Fitting Binomial(12,$\frac13$) to Weldon's dice data: Observed and fitted frequencies}\label{out:dice.1}
\verbatiminput{ch2/out/dice.1}
\end{Output}
\begin{Output}
\caption{Fitting Binomial(12,$\frac13$) to Weldon's dice data: Goodness of fit tests}\label{out:dice.2}
\verbatiminput{ch2/out/dice.2}
\end{Output}

The second call to the \macro{GOODFIT} allows the
 parameter $p$ to be estimated from the data, giving $\hat{p} = .3377$,
and produces the output shown in \outref{out:dice.3} and \outref{out:dice.4}.
The fit is much better---in fact, quite satisfactory.
So, Weldon's dice differed minutely from being absolutely fair,
but with over 26,000 tosses it is easy to detect the difference.
\begin{Output}
\caption{Fitting Binomial(12,$p$) to Weldon's dice data: Observed and fitted frequencies}\label{out:dice.3}
\verbatiminput{ch2/out/dice.3}
\end{Output}
\begin{Output}
\caption{Fitting Binomial(12,$p$) to Weldon's dice data: Goodness of fit tests}\label{out:dice.4}
\verbatiminput{ch2/out/dice.4}
\end{Output}
\end{Example}

\begin{Example}[federalist]{Federalist papers}
The data on the occurrences of the word \emph{may} in Madison's
Federalist Papers (\tabref{tab:madison})
are fit to both the Poisson and Negative binomial distributions as shown below.  In each case, the parameters are estimated from the data.  The output for the Poisson distribution appears
in \outref{out:madfit.1} and \outref{out:madfit.2}.
The results for the Negative binomial distribution appear
in \outref{out:madfit.3} and \outref{out:madfit.4}.
\begin{listing}
%include catdata(madison);
%goodfit(data=madison, var=count, freq=blocks, dist=poisson);

%goodfit(data=madison, var=count, freq=blocks, dist=negbin);
\end{listing}

\begin{Output}
\caption{Fitting the Poisson($\lambda$) to the Federalist Papers data: Observed and fitted frequencies}\label{out:madfit.1}
\small
\verbatiminput{ch2/out/madfit.1}
\end{Output}
\begin{Output}
\caption{Fitting the Poisson($\lambda$) to the Federalist Papers data: Goodness of fit tests}\label{out:madfit.2}
\verbatiminput{ch2/out/madfit.2}
\end{Output}

\begin{Output}
\caption{Fitting the Negative binomial($n, p$) to the Federalist Papers data: Observed and fitted frequencies}\label{out:madfit.3}
\small
\verbatiminput{ch2/out/madfit.3}
\end{Output}
\begin{Output}
\caption{Fitting the Negative binomial($n, p$) to the Federalist Papers data: Goodness of fit tests}\label{out:madfit.4}
\small
\verbatiminput{ch2/out/madfit.4}
\end{Output}
\end{Example}
