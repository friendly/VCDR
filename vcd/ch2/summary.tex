\section{Chapter summary}
\begin{itemize}
\item Discrete distributions typically involve basic \emph{counts} of occurrences
of some event occurring with varying frequencies.
\item The most commonly used discrete distributions include the binomial,
Poisson, negative binomial, geometric, and logarithmic series distributions.
Happily, these are all members of a family called the
power series distributions.
Methods of fitting an observed \Dset\ to any of these distributions are
described, and implemented in the \macro{GOODFIT}.
\item After fitting an observed distribution it is useful to plot the observed
and fitted frequencies.
Several ways of making these plots are described, and implemented in the
macro{ROOTGRAM}.
\item A graphical method for identifying which discrete distribution is most
appropriate for a given set of data involves plotting ratios
$k n_k / n_{k-1}$ against $k$.
These plots are constructed by the \macro{ORDPLOT}.
\item A more robust plot for a Poisson distribution involves plotting
a count metameter, $\phi ( n_k ) $ against $k$, which
gives a straight line (whose slope estimates the Poisson parameter)
when the data follows a Poisson distribution.
This plot provides robust confidence intervals for individual points
and provides a means to assess the influence of individual points
on the Poisson parameter.
These plots are provided by the \macro{POISPLOT}.
\item The ideas behind the Poissonness plot can be applied to the other
discrete distributions, as implemented in the \macro{DISTPLOT}.
\end{itemize}
