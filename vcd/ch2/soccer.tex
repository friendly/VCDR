\begin{Example}[soccer]{UK Soccer scores}
\tabref{tab:soccer1}  gives the distributions of goals scored by
the 20 teams in the  1995/96 season of the
 Premier League of the UK Football Association
as presented by
\citet{Lee:97}.%
\footnote{\citet[p. 16]{Lee:97} apparently has the home and away labels reversed in
his table.
The row and column labels in \tabref{tab:soccer1} give means of 1.48
for home teams and 1.06 for away teams.  The raw data were verified
from that listed at \url{http://users.aol.com/mabstabs/soccer.html}}
Over a season
each team plays each other team exactly once, so there are a total of
$20 \times 19 = 380$ games.
Because there may be an advantage for the home team,
the goals scored have been classified as ``home team'' goals
and ``away team'' goals in the table.
%% input: /users/faculty/friendly/sasuser/catdata/soccer1.sas
%% last modified: 09-Jan-98 16:32
\begin{listing}
title 'UK Soccer scores 95/96 season';
data soccer;
   input away @;
   do home = 0 to 4;
      total = home+away;
      input freq @;
      output;
      end;
datalines;
0   27 29 10  8  2
1   59 53 14 12  4
2   28 32 14 12  4
3   19 14  7  4  1
4    7  8 10  2  0
;
proc freq;
   weight freq;
   tables total;
run;

proc means mean var vardef=weight;
   var away home total;
   weight freq;
\end{listing}


If we assume that in any small interval of time there is a small, constant
probability that the home team or the away team may score a goal,
the distributions of the goals scored by home teams
(the row totals in \tabref{tab:soccer1})
may be modeled as Pois($\lambda_H$) and the distribution of
the goals scored by away teams (the column totals)
may be modeled as Pois($\lambda_A$).

If the number of goals scored by the home and away teams are independent%
\footnote{This question
is examined visually in \chref{ch:mosaic} (\exref{ex:soccer2})
and \chref{ch:corresp} (\exref{ex:soccer3}), where we find that the answer
is ``basically, yes''.},
we would expect that the total number of goals scored in any
game would be distributed as Pois($\lambda_H + \lambda_A$).
These totals are shown in \tabref{tab:soccer2}.
As preliminary check of the distributions for the home and away goals,
we can determine if the means and variances are reasonably close
to each other.
If so, then the total goals variable should also have a mean and variance
equal to the sum of those statistics for the home and away goals.
\begin{table}[!hb]
\caption{Total goals scored in 380 games in the Premier
Football League, 1995/95 season}
\label{tab:soccer2}
\vspace{.1in}
\begin{center}
\begin{tabular}{l|rrrr rrrr}
\hline
Total goals      &  0  &  1  &  2  &  3  &  4  &  5  &  6  &  7  \\
\hline
Number of games  & 27  & 88  & 91  & 73  & 49  & 31  & 18  &  3  \\
  \hline
\end{tabular}
\end{center}
\end{table}


The statements below read the data from \tabref{tab:soccer1}, calculate
the \texttt{TOTAL} goals, and find the distribution of \texttt{TOTAL} goals
shown in \tabref{tab:soccer2}.  The \PROC{MEANS} step produces the
mean and variance of each variable, shown in \outref{out:soccer1.2}.
%% input: /users/faculty/friendly/sasuser/catdata/soccer1.sas
%% last modified: 09-Jan-98 16:32
\begin{listing}
title 'UK Soccer scores 95/96 season';
data soccer;
   input away @;
   do home = 0 to 4;
      total = home+away;
      input freq @;
      output;
      end;
datalines;
0   27 29 10  8  2
1   59 53 14 12  4
2   28 32 14 12  4
3   19 14  7  4  1
4    7  8 10  2  0
;
proc freq;
   weight freq;
   tables total;
run;

proc means mean var vardef=weight;
   var away home total;
   weight freq;
\end{listing}


\begin{Output}[htb]
\caption{UK Soccer data, assessing Poissonness}\label{out:soccer1.2}
\small
\verbatiminput{ch2/out/soccer1.2}
\end{Output}
The means are all approximately equal to the corresponding variances.
More to the point, the variance of the \texttt{TOTAL} score
is approximately equal to the sum of the individual variances.
Note also there does appear to be an advantage for the home team,
of nearly half a goal.
\end{Example}
