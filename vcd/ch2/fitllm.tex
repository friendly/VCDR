\subsection{Fitting discrete distributions as \loglin\ models}
In \secref{sec:pwrseries}, I described how the common discrete distributions
are all members of the general power series family.
Another general family of distributions---the exponential family---%
includes most of the common continuous distributions:
the normal, gamma, exponential, and others,
and is the basis of the class of generalized linear models fit
by \PROC{GENMOD}.

\citet{LindseyMersch:92}, \citet[6.1]{Lindsey:95} have shown how various discrete
(and continuous)
distributions can be fit to frequency data using Poisson \loglin{} models
available in \PROC{GENMOD}.  The uniform, geometric, binomial, and the
Poisson distributions may all be fit easily in this way.
A clear advantage is that this method gives estimated standard errors for the
distribution parameters as well as estimated confidence intervals
for fitted probabilities.

The essential idea is that, for frequency data, any distribution in the
exponential family may be represented by a linear model for the logarithm
of the cell frequency, with a Poisson distribution for errors,
otherwise known as a ``Poisson \loglin\ regression model''.
These have the form
\begin{equation*}
\log (N \pi_k) = \textrm{ offset } + \beta_0 + \vec{\beta}\trans \vec{S}(k)
 \comma
\end{equation*}
where $\vec{S}(k)$ is a vector of zero or more sufficient statistics for the
canonical parameters of the exponential family distribution,
and the offset term is a value which does not depend on the
parameters.  \tabref{tab:expfamily} shows the sufficient statistics and
offsets for several discrete distributions.
See \citet{LindseyMersch:92} for further details, and definitions
for the double-binomial distribution.

\begin{table}[tb]
 \caption{Poisson \loglin\ representations for some discrete distributions}\label{tab:expfamily}
 \begin{center}
{\renewcommand{\arraystretch}{1.2}
 \begin{tabular}{lll}
  \hline
  \tableheader
  Distribution & Sufficient statistics & Offset \\
  \hline
  Geometric & $k$ \\
  Poisson & $k$ & $-\log(k!)$ \\
  Binomial & $k$ & $\log{\binom{n}{k}}$ \\
  Double binomial & $k, k \log(k) + (n-k) \log(n-k)$ & $\log{\binom{n}{k}}$ \\
  \hline
 \end{tabular}
}
 \end{center}
\end{table}


\begin{Example}[saxony2]{Families in Saxony}
The binomial distribution and the double binomial can both be fit to frequency data as a Poisson
regression using $\log \binom{n}{k}$ as an offset.
We only display results for the binomial model.
\begin{listing}
*-- calculate offset variables for binomial and double binomial;
data saxony;
   set saxony;
   logkn = log( gamma(12+1) / (gamma(males+1) * gamma(12-males+1)) );
   if 0 < males < 12
      then ylogity = -males * log(males/(12-males));
      else ylogity = 0;

   *-- fit binomial (12,p);
proc genmod data=saxony;
   model families = males /
      dist=poisson offset=logkn obstats ;

   *-- fit double binomial (12,p, psi);
proc genmod data=saxony;
   model families = males ylogity /
      dist=poisson offset=logkn obstats ;
\end{listing}
The goodness of fit tests shown in \outref{out:saxony2.1}
are equivalent to those calculated directly by the
\macro{GOODFIT} in \outref{out:saxony.2}.
The parameter estimate for \texttt{MALES}, $\beta_1 = 0.0769$
is actually estimating the logit of $p$, $\log p / (1-p)$,
so the inverse transformation gives
$\hat{p} = \frac{\exp (\beta_1)}{1 + \exp (\beta_1)} = 0.5192$,
as we had before.
The fitted frequencies (shown in \outref{out:saxony2.2}), given by the \opt{OBSTATS}{GENMOD}
on the \stmt{MODEL}{GENMOD} are the same as those
in \outref{out:saxony.1}.
The standard error for \texttt{MALES}, $s_{\beta_1} = 0.0074$
could also be transformed back to the probability scale in the same
way.
\begin{Output}
\caption{Fit of the Binomial($12, p$) to the Families in Saxony data: Goodness of fit tests}\label{out:saxony2.1}
\small
\verbatiminput{ch2/out/saxony2.1}
\end{Output}
\begin{Output}
\caption{Fit of the Binomial($12, p$) to the Families in Saxony data: Observed and fitted frequencies}\label{out:saxony2.2}
\small
\verbatiminput{ch2/out/saxony2.2}
\end{Output}
\end{Example}
