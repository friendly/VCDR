
\section{Correspondence analysis}
\ix{contingency table!correspondence analysis}
\ix{correspondence analysis|(}
Correspondence analysis is a technique related to the biplot
\ix{biplot!compared to correspondence analysis}
(\secref{sec:5mu@bi})
and to principal components analysis (\secref{sec:5ms@pc})
\ix{principal components analysis}
which finds a multidimensional representation
of the association between the row and column categories of a two-way
contingency table.  Like the biplot, the technique uses the
singular value decomposition of a matrix
\ix{singular value decomposition}
to find scores for the
row and column categories on a small number of dimensions.
Like principal components analysis,
these dimensions account for the greatest proportion of the
\(\chi^2\) for association between the row and column categories,
just as
components account for maximum variance.
For graphical display two (or at most three) dimensions are
typically used to give a reduced rank approximation to the
data.

Correspondence analysis has a very large literature and is also
discussed under the names ``dual scaling'',
``reciprocal averaging'', and ``canonical analysis of
categorical data''.
See Greenacre (1984), Nishisato (1980),
or Lebart, Morineau & Tabard (1977)
for a detailed treatment of the method and other applications.
Greenacre & Hastie (1987) provide an excellent discussion of
the geometric interpretation, and van her Heijden & de Leeuw (1985)
develop some of the relations between correspondence analysis
and log-linear methods.


The matrix which is decomposed is a matrix of deviations from
independence, \(\mathbf{D}\), expressed in terms of the proportions,
\(p_{ij}\),
\begin{equation}\label{eq:devind}
\mathbf{D} = { d_{ij} } = \frac{( p_{ij} - p_{i+} p_{+j} )}{\sqrt{ p_{i+}  p_{+j} }}
 = 
\frac{( f_{ij} - e_{ij} )}{\sqrt{ e_{ij} }  \sqrt{n}}
\quad .
\end{equation}
Thus, each \(d_{ij}\) is \(( 1 / \sqrt{n} )\) times that
cell's contribution to the \(\chi^2\) statistic, and the sum of
squares of the \(d_{ij}\) over all cells in the contingency
table is \(\chi^2 / n\).
Thus, correspondence analysis is designed to show
how the data deviate from expectation when the row and column variables
are independent, as in the association plot.  However, the association
plot depicts every
\emph{cell} in the table, and for large tables it may be
difficult to see patterns.  Correspondence analysis shows only
row and column \emph{categories} in
the two
(or three) dimensions which account for the greatest proportion of
deviation from independence.


The points for each row and column are found from the singular value
decomposition of the residual matrix \(\mathbf{D}\),
\begin{equation}\label{eq:casvd}
\mathbf{D} = \mathbf{U}  \mathbf{\Lambda}  \mathbf{V} prime
\quad ,
\end{equation}
where \(\mathbf{ U prime  U } = \mathbf{I}\),
\(\mathbf{ V prime  V } = \mathbf{I}\),
and \(Lambda\) is a diagonal matrix with singular values \(lambda_{i}\)
in descending order.
\(\mathbf{U}\) and \(\mathbf{V}\) contain the scores for the row and column categories.
These scores are standardized as follows to give
the \boldital{principal coordinates} for rows and columns:
\begin{equation}\label{eq:caxr}
\mathbf{X}_{r} = \mathbf{R}^{- 1 / 2}  \mathbf{U}  \mathbf{\Lambda}
\end{equation}
\begin{equation}\label{eq:caxc}
\mathbf{X}_{c} = \mathbf{C}^{- 1 / 2}  \mathbf{V}  \mathbf{\Lambda}
\quad ,
\end{equation}
where \(\mathbf{R} = bold diag { \bvec{p}_{i+} }\)
and \(\mathbf{C} = bold diag { \bvec{p}_{+j} }\) are
diagonal matrices containing the row and column totals of \(\mathbf{P}\),
respectively.
Other scalings of the coordinates are possible; however
the principal coordinates are most useful for graphical
display since the distances among the row points and the
distances among column points can be interpreted in the same way.
Distances \emph{between} the row and column points, however,
are not directly meaningful in any scalings of the coordinatess.


In correspondence analysis this solution has the properties that:
\begin{itemize}
\item the distances between row points or between column points
in (Euclidean) space are equal to Chi-squared distances.  That is,
row (or column) points which are close together in the space correspond
to rows (columns) with similar profiles in the frequency table.
The profiles of the marginal row and column frequencies
are both projected to the origin in this space.  Thus, when the profile
of a given row (column) is similar to the marginal profile, the
corresponding distance of the row (column) point to the origin is small.
\item the squared singular values,
\(lambda_{i}^{2}\), (called \boldital{principal inertias})
are proportional to the contribution
of the corresponding dimension to the overall \(\chi^2\)
for association between row and column variables.
In the usual two-dimensional solution, the proportion of the
\(\chi^2\) explained is therefore
\(n ( lambda_{1}^{2} + lambda_{2}^{2} ) / \chi^2\).
\item assigning the scores in the first column of \(\mathbf{X}_{r}\)
to the row categories and in the first column of \(\mathbf{X}_{c}\)
to the column categories provides the maximum (Pearson) correlation,
equal to \(lambda_{1}\),
between these optimally scaled variables.
As in canonical correlation
analysis, the second and subsequent columns of scores have the greatest
correlation (equal to the corresponding \(lambda_{i}\)),
subject to being uncorrelated with the optimally
scaled variables of the previous columns.
\end{itemize}
Thus, correspondence analysis can be seen either as a technique for
decomposing the overall \(\chi^2\) into a small number of independent
dimensions, or as a method for quantifying or scaling the two
categorical variables to produce the maximal
correlations.
In Version 6 of the SAS System, correspondence analysis is carried out by
the CORRESP procedure.
A SAS macro program, CORRESP, is also provided here for
Version 5 users.

\subsection{\PROC{CORRESP}}
\ix{correspondence analysis!CORRESP procedure}
\ix{CORRESP procedure|(}
In Version 6, correspondence analysis is performed using PROC
CORRESP in SAS/STAT.
The procedure can accept
data either in the form of a contingency
table, or in the form of raw responses on two or more
categorical variables.
An OUT= data set from \PROC{CORRESP} contains the row and column
coordinates, which can be plotted with \PROC{PLOT} or \PROC{GPLOT}.
The procedure has many options for scaling row and column coordinates,
and
for printing various statistics which aid interpretation.
Only the basic use of the procedure is illustrated here.
For further details see :SUG MAN=STAT6..

\example{Hair and Eye Color}
The program below reads the hair and eye color data into the data set
COLORS, and calls the CORRESP procedure.
This example illustrates the use of the Annotate facility with
\PROC{GPLOT} to produce a labelled display of the correspondence analysis solution.
To input a contingency table in the CORRESP step,
the hair colors (columns) are specified as the variables in the
VAR statement, and the eye colors (rows)
are indicated as the ID variable.
:inclsas file=corresp3 name=corres3a edit=yes place=any

The printed output from the CORRESP procedure
is shown in \figref{fig:ca4}.
(Additional printed output, giving
other statistics for the points, was suppressed in this
example with the SHORT option.)
The section labelled ``Inertia, ... '' indicates that
over 98\% of the \(\chi^2\) for
association is accounted for by two dimensions, with
most of that attributed to the first dimension.
The coordinates (using the default scaling PROFILE=BOTH)
are computed according to Eqns.
\eqref{eq:caxr} and \eqref{eq:caxc}.
.*===========================F=I=G=U=R=E=============================*
\begin{figure}\label{ca4}
:FIGC source=CORRESP3 ftype=LISTING file=XHB11CA4
\input{xhb11ca4}
.* figure body here
\caption{\PROC{CORRESP} results for hair and eye color data}
\end{figure}


The singular values, \(lambda_{i}\), in Eqn.
\eqref{eq:casvd},
are also the (canonical) correlations between the optimally
scaled categories.
Thus, if the DIM1 scores for hair color and eye color are assigned to the
592 observations in the table, the correlation of these variables
would be 0.4569.
This interpretation provides an additional way to understand the
strength of the association between hair and eye color in these data.
The DIM2 scores give a second,
orthogonal scaling of these two categorical
variables, whose correlation would be 0.1491.

A plot of the row and column points can be constructed from
the OUT= data set COORD requested in the \PROC{CORRESP} step.
\ix{CORRESP procedure!OUT= data set}
The variables of interest in this example are shown in
\figref{fig:ca5}.
Note that row and column points are distinguished by the
variable _TYPE_.
(The observation with _TYPE_='INERTIA' is ignored in this
application.)
The QUALITY variable gives a measure of the proportion of a point's
distance from the origin which is accounted for by the two-dimensional
solution.
This measure indicates that all the points are represented
extremely well in two dimensions; the smallest value, 0.88 for
hazel eyes is quite adequate.
.*===========================F=I=G=U=R=E=============================*
\begin{figure}\label{ca5}
:FIGC source=CORRESP3 ftype=LISTING file=XHB11CA5
\input{xhb11ca5}
.* figure body here
\caption{Output data set COORD from \PROC{CORRESP}}
\end{figure}


A labelled \PROC{GPLOT} display of the correspondence analysis solution
is constructed with a DATA step to produce an Annotate data set LABELS from
the COORD data set,
and a \PROC{GPLOT} step shown below.
In the \PROC{GPLOT} step, it is crucial
to scale the plot so that the number of
data units/inch are the same for both dimensions.
Otherwise, the distances in this
plot would not be represented accurately.
This is done
with the AXIS statements: AXIS1 specifies a length and
\ix{AXIS statement!LENGTH= option}
range which are both twice that in the AXIS2 statement, so that
the ratio of data units to plot units is the same in both
dimensions.
\ix{equating axes}
\ix{axes!equating}
Note that it is not necessary to make the axes the same length;
rather, a one-unit distance must be the same physical length
on both axes.
:inclsas file=CORRESP3 name=CORRES3B edit=yes place=any

\subsection{Interpreting the correspondence analysis plot}
\ix{correspondence analysis!interpretation|(}
The two-dimensional plot shown in \figref{fig:cahair2}
is interpreted in terms of distances among the points representing
the row categories and the distances among the points for
the column categories.
(Recall that the origin in the plot represents both the marginal
row and column profiles.)
Dimension 1 is a ``light--dark''
contrast between blond and black hair: these two columns differ
most in the profiles of eye color frequency.
This dimension also contrasts blue and brown eyes.
The vertical axis reflects the difference between the profiles
for red hair from the other hair colors, and a corresponding
difference of
green and hazel eyes from the profiles for brown and blue eyes.
These differences represent
a second, orthogonal
dimension of association between hair color and eye color.

Notice that in the description above we do not interpret
the \emph{distance}
between the hair and eye color points.
However, the relative positions of the row and column points
\emph{on the coordinate axes} can be interpreted as follows.
From \eqref{eq:casvd} it can be shown (Greenacre, 1984)
that
each deviation from independence, \(d_{ij}\),
is proportional to the weighted sum, across all dimensions,
of products of the coordinates,
\(\bvec{x}_{r} ( i )\) for row \(i\)
with \(\bvec{x}_{c} ( j )\) for column \(j\).
Thus, pairs of row-column points
with similar coordinates on both axes,
like blue-blond, and brown-black,
represent cells with large positive deviations from independence,
that is \(f_{ij} \gg e_{ij}\);
pairs of
points with large coordinates of opposite sign, like blue-black
reflect cells whose frequency is far less than independence predicts,
\(f_{ij} \ll e_{ij}\).
Thus, we can say that the correspondence analysis plot shows that
blue eyes go along
with blond hair, while brown eyes are associated with black hair;
but to be precise, this interpretation
is based on comparing their (signed) distances from the
origin rather than the distance between the row and column points.
\ix{correspondence analysis!interpretation|)}

Comparing the correspondence analysis plot in
\figref{fig:cahair2} to the association plot
in \figref{fig:asspl1}, we see they show similar patterns of
association between hair and eye color.
For example, both plots show the positive association of
blue eyes with blond hair and brown eyes with black hair.
Correspondence analysis,
however, resolves the pattern of association and the overall
\(\chi^2\)
into orthogonal dimensions
and summarizes the data in two dimensions, whereas the association
mosaic plots display the whole pattern.
Experience with these graphical displays seems to indicate that correspondence analysis is
particularly useful for large tables, while the association
plot and mosaic display, being more direct, are suitable for
smaller tables.
\ix{correspondence analysis!compared to association plot}
.*===========================F=I=G=U=R=E=============================*
\begin{figure}\label{cahair2}
:FIGC SOURCE=CORRESP1
.* figure body here
\caption{Correspondence analysis plot for hair-eye color data}
\end{figure}
\ix{CORRESP procedure|)}

\subsection{CORRESP macro}
\ix{CORRESP macro|(}
For Version 5 of the SAS System, correspondence analysis can be carried out using
the SAS macro program, CORRESP, provided in
\secref{sec:a1@ca}.
The macro was designed to be similar in use to \PROC{CORRESP},
and to produce similar results to the basic analysis
illustrated in the previous section.
As with the association plot, the calculations for correspondence
analysis are done most easily with \PROC{IML}.
\ix{IML procedure}
The parameters of the CORRESP macro are shown below:
:inclsas file=CORRESP name=CORRESPM edit=YES
The input to the CORRESP macro
must be a two-way table of frequencies.
The columns of the table are specified by the variables
listed in the VAR= parameter; the rows of the table are identified
by the ID= parameter.


The CORRESP macro computes the row and column coordinates
in two dimensions, which are returned in the variables DIM1 and DIM2
in the output data set specified by the OUT= parameter.  The macro also
constructs an Annotate data set to label the row and column points in the
output data set specified by the ANNO= parameter.
The OUT= data set is plotted in a separate step with \PROC{GPLOT},
as illustrated in the examples below.

\example{Hair and Eye Color}
The program below carries out the same analysis of the
hair and eye color data using the CORRESP macro.
The %CORRESP statement is very similar to the \PROC{CORRESP} step
in the previous example.
The hair colors (columns)
are specified as the variables (VAR=), and the eye colors (rows)
are indicated as the ID= parameter.
:inclsas file=CORRESP1 edit=YES
\figref{fig:cahair1} shows the printed output from the CORRESP
macro. It contains essentially the same information as
the output from \PROC{CORRESP} (\figref{fig:ca4}).
.*===========================F=I=G=U=R=E=============================*
\begin{figure}\label{cahair1}
:FIGC source=CORRESP1 ftype=LISTING file=XHB11CA0
.* figure body here
\input{XHB11CA0}
\caption{CORRESP macro results for hair and eye color data}
\end{figure}
\ix{CORRESP macro|)}

\subsection{Multi-way tables}
\ix{correspondence analysis!multi-way tables}
\ix{contingency table!multi-dimensional|(}
A three- or higher-way table can be analyzed by correspondence analysis in several ways.
One approach, called \boldital{multiple correspondence analysis},
starts with a matrix, \(\mathbf{Z}\) of
indicator (dummy) variables, one for
each category of each variable, with one row for each respondent.
For the hair-eye color data, this matrix would have \(n = 592\)
rows and 8 columns (4 hair color categories and 4 eye color categories.).
The matrix \(\mathbf{Z} prime  \mathbf{Z}\), called the \boldital{Burt matrix}
is then an \(8 \times 8\) matrix which is analyzed in
multiple correspondence analysis.
This scheme generalizes to three or more variables as follows:
For a three-way table, with \(I + J + K\) categories,
the indicator matrix has \(( I + J + K )\) columns,
and the Burt matrix is of order
\(( I + J + K ) \times ( I + J + K )\).
This analysis is provided by the MCA option of \PROC{CORRESP}.
\ix{CORRESP procedure!MCA option}
See :SUG MAN=STAT6., ``The CORRESP Procedure'',
Example 2, for an example of multiple correspondence analysis of a seven-way table.
However, for more than two variables
the geometric representation in multiple correspondence analysis is
not a straightforward generalization of simple correspondence analysis.
See Greenacre & Hastie (1987) for a more detailed discussion.

A second approach, which does maintain the same geometric
interpretation is called ``stacking''.
A three-way table, of size \(I \times J \times K\) can be sliced
into \(I\) two-way tables, each \(J \times K\).
If the slices are concatenated vertically, the result is one
two-way table, of size \(( I \times J ) \times K\).
In effect, the first two variables are treated as a single
composite variable,
which represents the main effects and
interaction between
the original variables that were combined.
Van der Heijden and de Leeuw (1985) discuss this use of correspondence analysis
for multi-way tables and show how each way of slicing and
stacking a contingency table corresponds to the analysis of
a particular log-linear model.
\ix{correspondence analysis!and log-linear models}

\example{Suicide Rates}
\ix{Data sets!Suicide data}
To illustrate this second
method of analysis for three-way tables, we use data
on suicide rates in West Germany, classified
by age, sex, and method of suicide used.
The data, from Heuer (1979, Table 1),
have been discussed by van der Heijden & de Leeuw (1985) and others.
The table is a three-way contingency table, \(2 \times 17 \times 9\).
In this example, the table is structured as \((2 \times 17)\)
rows by 9 columns, so the rows represent the joint effects of sex
and age.


The data set SUICIDE is created in the following DATA step.
Here we have defined a new variable, SEXAGE, to represent the
age--sex combinations.
\figref{fig:suidata} shows the data set created in this step.
:inclsas file=SUICIDE place=any edit=YES
.*===========================F=I=G=U=R=E=============================*
\begin{figure}\label{suidata}
:FIGC source=SUICIDE ftype=LISTING file=XHB11CA1
.* figure body here
\input{XHB11CA1}
\caption{SUICIDE data set}
\end{figure}
:PERMIT BIB='van der Heijden & de Leeuw (1985)' desc='Suicide data (Table 1, p.438)' COPY='Psychometric Society'


The program below applies the CORRESP macro to the SUICIDE data
and uses a \PROC{GPLOT} step to
plot the OUT=COORD data set with the labels given in the ANNO=LABELS
data set.
The axes are again equated by the use of LENGTH options on the
AXIS statements.
\ix{AXIS statement!LENGTH= option}
The two-dimensional plot is shown in \figref{fig:ca1}.
.*===========================F=I=G=U=R=E=============================*
\begin{figure}\label{ca1}
:FIGC SOURCE=SUICIDE1 colors=1
.* figure body here
\caption{Two dimensional correspondence analysis of suicide data}
\end{figure}
:inclsas file=SUICIDE1 edit=yes
\figref{fig:ca2} gives the printed output from the analysis
(excluding the row and column coordinates).
The \(\chi^2\) for
``Overall Association'' indicates a highly significant
relation between method and the age--sex combinations.
The second table in \figref{fig:ca2} lists the singular values of
the deviations matrix
and the contribution of dimension
each to the overall \(\chi^2\) value.
For these data, the two dimensions shown in \figref{fig:ca1}
account for almost 90\% of the overall association.
.*===========================F=I=G=U=R=E=============================*
\begin{figure}\label{ca2}
:FIGC source=SUICIDE1 ftype=LISTING file=XHB11CA2
.* figure body here
\input{XHB11CA2}
\caption{Printed output from CORRESP macro for suicide data}
\end{figure}

Dimension 1 in the plot separates males and females.  This dimension,
which by itself accounts for over 51\% of the \(\chi^2\) value,
indicates a strong difference between suicide profiles of males
and females.
The second
dimension, accounting for 38\% of the association,
is mostly ordered by age with younger groups at the top
and older groups at the bottom.
Note also that the positions of the age groups are approximately
parallel for the two sexes.
Such a pattern indicates that sex and age do not interact in this
analysis.
The relation between the age--sex
groups and methods of suicide can be interpreted in terms of
similar
distance and direction from the origin, which represents the marginal
row and column profiles.
Young males are more likely to commit suicide
by gas or a gun, older males by knife or hanging, while young
females are more likely to ingest some toxic agent and older females
by jumping or drowning.

The suicide data in \figref{fig:suidata} are actually a set of
profiles for each age--sex combination in an eight-dimensional
space:dash.there are nine variables, but the row percentages must
add to 100\%.  Correspondence analysis gives a two-dimensional summary,
which in this case accounts for 90\% of the association between the
age--sex groups and methods of suicide.

It should be emphasized that correspondence analysis, like the association and
mosaic plots, may suggest patterns of association among the
variables in contingency tables, but does not establish whether
those patterns are significant.
Log-linear methods provide \(\chi^2\) tests of independence
and generalized independence for multi-way tables, but do not
reveal the pattern of non-independence for significant effects
in the model.
\ix{correspondence analysis!and log-linear models}
These confirmatory methods should therefore be used to determine
\emph{which} variables are related; the exploratory methods of
this chapter can then be used to understand \emph{how} those
variables are related.
\ix{contingency table!multi-dimensional|)}
\ix{correspondence analysis|)}
