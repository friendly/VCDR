\subsection{Geometric and statistical properties}\label{sec:ca-properties}
\ixon{correspondence analysis!properties}
We summarize here some geometric and statistical properties of the
\CA\ solutions which are useful in interpretation.

\begin{description}
\item[nested solutions:] Because they use successive terms of the SVD
  \eqref{eq:cadij}, \CA\ solutions are \emph{nested}, meaning that the first
  two dimensions of a three-dimensional solution will be identical
  to the two-dimensional solution.

\item[centroids at the origin:] In both principal coordinates and standard
coordinates the points representing the row and column profiles have their
centroids (weighted averages) at the origin.
Thus, in \CA\ plots, the origin represents the (weighted) average
row profile and column profile.

\item[reciprocal averages:]
The column scores are proportional to the weighted averages of the row
scores, and vice-versa.

\item[chi-square distances:]  In principal coordinates, the row coordinates
may be shown equal to the row profiles $\mat{D}_r^{-1} \mat{P}$, rescaled inversely by the square-root of the column masses, $\mat{D}_c^{-1/2}$.
Distances between two row profiles, $\mat{R}_i$ and $\mat{R}_{i^\prime}$
is most sensibly defined as $\chi^2$ distances, where the squared
difference $[\mat{R}_{ij} -\mat{R}_{i^\prime j}]^2$ is inversely weighted
by the column frequency, to account for the different relative
frequency of the column categories.
The rescaling by $\mat{D}_c^{-1/2}$ transforms this weighted $\chi^2$
metric into ordinary Euclidean distance.
The same is true of the column principal coordinates.

\item[interpretation of distances:]
In principal coordinates,
the distance between two row points may be interpreted as described
above, and so may the distance between two column points.
The distance between a row and column point, however, does not have
a clear distance interpretation.

\item[residuals from independence:]
The distance between a row and column point do have a rough
interpretation in terms of residuals or the difference between
observed and expected frequencies, $n_{ij} - m_{ij}$.
Two row (or column) points deviate from the origin (the average
profile) when their profile frequencies have similar values.
A row point appears near a column point when  $n_{ij} - m_{ij} >
0$, and away from that column point when the residual is negative.
\end{description}

Because of these differences in interpretations of distances, there
are different possibilities for graphical display.
A joint display of principal coordinates for the rows and standard
coordinates for the columns (or vice-versa), sometimes called
an \emph{asymmetric map} is suggested by
\ix{correspondence analysis!asymmetric map}
\citet{GreenacreHastie:87} and by \citet{Greenacre:89} as the plot
with the most coherent geometric interpretation
(for the points in principal coordinates) and is widely
used in the French literature.
The options \pname{PROFILE=ROW} and \pname{PROFILE=COLUMN}
in \PROC{CORRESP} generate the asymmetric map.

\ix{correspondence analysis!symmetric map}
Another common joint display is the \emph{symmetric map} of the principal
coordinates in the same plot, produced with the option \pname{PROFILE=BOTH}.
In the author's opinion, this produces better graphical displays, because
both sets of coordinates are scaled with the same weights for each axis.
Symmetric plots are used exclusively in this book, but that should
not imply that these plots are universally preferred.
Another popular choice is to avoid the possibility of misinterpretation
by making separate plots of the row and column coordinates.
The different scalings, and the valid distance interpretations for each
are described in detail in the Algorithms section of
\STUGref{19}{The CORRESP Procedure}.
\ixoff{correspondence analysis!properties}
