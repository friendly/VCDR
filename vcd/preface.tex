\chapter{Preface}
%\addcontentsline{toc}{chapter}{\numberline{}Preface}
\begin{epigraphs}
\qitem{The \emph{practical} power of a statistical test is the product
of its' statistical power times the probability that one will use it.}
{J. W. Tukey, 1959\nocite{Tukey:59}}

\qitem{Theory into Practice}{Chairman Mao Zedong, \emph{The Little Red Book}}
\end{epigraphs}

This book aims to provide a practically-oriented, in-depth treatment
of modern graphical methods and visualization techniques for
categorical data---discrete response data and frequency data.  It
describes the underlying statistical theory and methods, and presents
nearly 40 general-purpose SAS macros and programs to perform these
specialized analyses and produce insightful visual displays.

The application of these techniques to a large number of substantive
problems and \Dset{}s is another focus: how to analyze the data,
produce the graphs, and understand what they have to say about the
problem at hand.  A number of these examples are re-viewed from
several different perspectives.

In a real sense, this book continues where \SSSG\ leaves off.
That book ended with a chapter titled ``Displaying categorical data.''
It surveyed some of the few graphical methods available for discrete
data at that time, saying ``while graphical display techniques are
common adjuncts to analysis of variance and regression, methods for
plotting \ctab\ data are not as widely used. (p. 499)''

There were not many graphical methods available, many of those
in the literature were designed for specialized situations
(two-way tables, or just $2 \times 2$ tables), and, except for
correspondence analysis, very few were available in standard statistical
software.  Of the methods I had encountered, association plots
and mosaic displays seemed sufficiently promising to include in that
chapter with simple SAS programs and examples.

The disparity between availability and use of graphical methods
for quantitative data, on the one hand, and analogous methods for
categorical data on the other seemed particularly curious---almost
paradoxical.  The statistical methods for the later (\loglin\ models,
logistic regression) are such close analogs of the standard method
for the former (ANOVA, regression) that I found the contrast in
visualization methods puzzling.

Since that time, I and others have actively worked on the
development of new graphical methods for categorical data,
with the goals of (a) providing visualization techniques for data
exploration and model fitting comparable in scope to those used
for quantitative data, (b) implementing these methods in
readily available software, and (c) illustrating how these methods
may be used in understanding real data---theory into practice.

Beginning somewhat earlier, the development of the generalized
linear model (e.g., \citet{McCullaghNelder:89}) created the
statistical machinery to integrate many aspects of classical linear
models for quantitative data with similar linear models for discrete
responses and freqency data within a common framework.  As a result,
many of the commonly-used diagnostic displays (normal probability
plots of residuals, added-variable plots, influence plots, etc.)
described in \SSSG\ could be adapted or extended to categorical
data.

Altogether, what seemed worth only a brief skeleton chapter in
1991, has progressed to meat on the bones,
perhaps worthy of an extended treatment, to which you are now
invited.

\section*{How to use this book}
\begin{changebar}
This book was written to make graphical methods for categorical
data \emph{available} and \emph{accessible}.

\textbf{Available methods should be conveniently collected, described and illustrated.}
\emph{Available} means that I try to show how graphs, some old,
and some quite novel, can be used to expose or summarize important
features of categorical data.  
I try to collect
and describe methods I consider useful all together here.
In part, this is done by example,
quite a few of which are treated several times throughout the text,
from different perspectives, or to illustrate different views of the
same data.  If you are looking for a graphical method for to help
you understand some particular type of data, the examples may help
you find some useful candidates.  Use the Example Index to track
the various analyses applied to a given data set.

\textbf{Accessible methods should be easy to use.}  
\emph{Accessible} reflects the opening quotations of this Preface.
Some technique may be well-described somewhere, but inaccessible,
because it is hard for you to use with your own data.
I have tried to provide general tools, conceptual and computational,
for \emph{thinking about} and \emph{doing} categorical data analysis
guided by visualization.  The statistical theory for the methods
described is, of necessity, somewhat abbreviated, but oriented toward
understanding how to apply these methods.  You may wish to refer to
cited references for more detail.  The programs developed for this
book (described in \appref{ch:macros}) reflect my aim to make it easy
for you to use these methods with your own data.  If you are not familiar
with the use of SAS macros, it will take you a bit of effort to begin
to use these programs, but I strongly believe that small effort will
empower you greatly, and help \emph{you} convert theory into practice.

Beyond the ``information'' provided, I also tried to make the \emph{structure}
of this information available and accessible, within the confines of
a printed (and therefore linear) work.
The following subsection provides a synopsis of the contents of each chapter of
this book.
Each chapter begins with visual thumbnail images of some of the
graphical methods which are described there, and a capsule summary.  
Each chapter ends with a
summary of the main points and methods.  
\end{changebar}

\subsection*{Overview}
\textbf{\chref{ch:intro}: ``Introduction''} introduces some aspects of categorical data,
distinctions among different types of data, and different strategies
for analysis of frequency data and discrete response data.  We
discuss the implications of these features of categorical data for
visualization techniques, and outline a strategy of data analysis
focused on visualization.

\textbf{\chref{ch:discrete}: ``Fitting and graphing discrete distributions''} describes the well-known discrete
frequency distributions: the binomial, Poisson, negative binomial,
geometric, and logarithmic series distributions,
along with methods for fitting these to empirical data.
Graphic displays are used to visualize goodness of fit,
to diagnose an appropriate model, and determine the impact of
individual observations on estimated parameters.


\textbf{\chref{ch:twoway}: ``Two-way contingency tables''} presents methods of analysis designed
mainly for two-way tables of frequencies (\ctab{}s), along with
graphical techniques for understanding the patterns of associations
between variables.
Different specialized displays are focused
on visualizing an odds ratio (a fourfold display for
$2 \times 2$ tables), or the general
pattern of association
(sieve diagrams),  the agreement between row and column
categories (agreement charts), and relations in $n \times 3$ tables
(trilinear plots).

\textbf{\chref{ch:mosaic}: ``Mosaic displays for n-way tables''} introduces the mosaic display, a general method
for visualizing the pattern of associations
among variables in two-way and larger tables.
Extensions of
this technique can reveal partial associations, marginal associations,
and shed light on the structure of \loglin\ models themselves. 
 
\textbf{\chref{ch:corresp}: ``Correspondence analysis''} discusses
correspondence analysis, a technique designed to
provide visualizations of associations in a two-way \ctab\
in a small number of dimensions.
Multiple correspondence analysis extends this technique to \nway\
tables.  Other grahical methods, including mosaic matrices and biplots
provide complementary views of \loglin\ models for two-way and \nway\
\ctab{}s.

\textbf{\chref{ch:logistic}: ``Logistic regression''}
introduces the model-building approach of
logistic regression, designed to describe the relation between 
a discrete response, often binary, and a set of explanatory variables.
Smoothing techniques are often crucial in visualizations for such
discrete data.  The fitted model provides both inference and
prediction, accompanied by measures of uncertainty.  Diagnostic
plots help us to detect influential observations which may distort
our results.

\textbf{\chref{ch:loglin}: ``Loglinear and logit models''}
extends the model building approach to loglinear and logit models.
These are most easily interpreted through
visualizations, including mosaic displays and plots of associated
logit models.  As with logistic regression, diagnostic plots
and influence plots help to assure that the fitted model is
an adequate summary of associations among variables.

\textbf{\appref{ch:data}: ``SAS programs and macros''}
documents all the SAS macros and programs
illustrated in the book.

\textbf{\appref{ch:data}: ``Data sets''}
lists the \Dstp{}s used to create the
principal data sets used in the book.

\textbf{\appref{ch:tables}: ``Tables''}
lists two tables of the the values of the \chisq\ distribution,
along with a SAS program which may be customized to provide
similar information in any desired format.

\section*{Acknowledgements}
Many colleagues, friends, students, and internet-acquaintances have contributed
directly and indirectly to the preparation of this book.

The seed for this book was planted during a staff seminar series
on categorical data analysis held by the Statistical Consulting
Service at York University in 1991--92;
over several subsequent years I taught a short course on
Graphical Methods for Categorical Data, and received valuable feedback
from many colleagues and students.
I am grateful to my colleagues, friends and students
in the Statistical Consulting Service:
John Fox,
Georges Monette,
Mirka Ondrack,
Peggy Ng,
Roman Konarski,
Tom Buis,
Ernest Kwan, and
Tom Martin.

A number of people reviewed the book at various stages, offering helpful suggestions
and comments: John Fox, Ernie Kwan, Rick Wicklin,
Sanford Gayle,
Duane Hayes,
Mary Rios, and
Kathy Shelley.
\begin{changebar}
Russ Tyndall carefully reviewed most of the macro programs.
\end{changebar}
I also benefited from discussions of some of the topics discussed
with Forrest Young, Michael Greenacre, Antoine de Falguerolles and 
Howard Wainer, and with the participants of the Workshop on ``Data Visualization in Statistics'' organized by Andreas Buja.

At SAS Institute Publications, David Baggett encouraged and supported this work.
Julie Platt, and later Patsy Poole served as editors, helping me to
refine the text and produce a printed version with the look and feel
I desired.
\begin{changebar} 
\aunote{Cover design?  Copy editing? Who else should be mentioned???}
It was a pleasure to work with them all.

Along the way, I faced many SAS programming challenges.  As always,
the contributors to ``SAS-L'' (\url{news:comp.soft-sys.sas}) were generous
in their assistance.  Among the many who helped,
Jack Hamilton,
Melvin Klassen,
Ian Whitlock, and
Matthew Zack deserve particular mention.
\end{changebar}


I wrote this book using \LaTeX, and the learning curve was steeper
than I imagined.  Donald Arseneau, David Carlisle, David Kastrup,
Bernd Schandl,
Paul Thompson, and other contributors to \url{comp.text.tex} were
generous with their time and expertise, helping me often to translate
my ideas into boxes which looked right on the printed page.
\begin{changebar}
Demian Conway and others on the \url{comp.lang.perl.*} newsgroups
helped me to construct some Perl tools which considerably eased the
work.
\end{changebar}

At York University,
Ryan McRonald provided heroic support and wizardry to keep my aging
``Hotspur'' workstation going, and rescued me from several crises
and catastrophes over the course of this work.  
Mike Street, Tim
Hampton, and other members of the Academic Technical Support Group
ably assisted me in setting up and tuning many
software components needed during this project.
Marshal Linfoot and other members of the Unix Team answered many
questions and provided invaluable support.

Finally, I am grateful to the National Sciences and Engineering Research
Council of Canada for research support on my project
``Graphical Methods for Categorical Data'' (Grant 8150)
and to York University for a sabbatical leave and a research grant
in 1998--99, during which most of this book was drafted.

Whatever defects remain, after all this help, are entirely my
responsibilty.
\vspace*{1in}

\begin{tabular}{l}
\textsc{Michael Friendly} \\ Toronto, Ontario \\ January, 2000
\end{tabular}

\begin{comment}
\section{About this draft}
This is now an essentially complete draft of the book.  However,

\begin{itemize*}
\item No attempt has been made to position floating material
(figures, tables, and outputs) optimally in relation to the 
relevant text.
\item No attempt has been made to avoid splitting non-floating display material (program listings, un-numbered output) across pages.
\item Some color figures display poorly in b/w.  No attempt has been
made to fix this yet, nor has a decision been made about the amount
of color printing.
\item Figure captions often use a style consisting of a figure title
plus an additional figure description.  I know that this is at variance
with most SAS publications, but I believe it is important to include
certain information related to interpreting the figure in the caption,
rather than burying it in the text.  Figures should be readable on their
own.  I would like to retain this style.
\item The index consists mainly of those terms that were added automatically
by \LaTeX\ macros.  There should also be an author index.
\item Appendix \ref{ch:data} is still incomplete;  there needs to be some descriptive
text for each \Dset.
\item Subfigures are sometimes composed into a single .eps file 
(with \PROC{GREPLAY}), sometimes composed with \LaTeX \verb|\minipage|.
Occasionally (after I learned how) I used the \LaTeX  \verb|\subfigmatrix|
package.
\end{itemize*}

The lists of Figures, Tables, and Outputs are included merely for checking
cross-references.  They are not intended to be part of the printed book.
\end{comment}


