\begin{Example}[berkeley7]{Berkeley admissions}
The homogeneous association model, $[AD] [AG] [DG]$
did not fit the Berkeley admissions data very well,
and we saw that the term $[AG]$ was unnecessary.
Nevertheless, it is instructive to consider the equivalent logit model.
We illustrate the features of the logit model which lead to the same
conclusions and simplified interpretation from graphical displays.

Because Admission
is a binary response variable, model \eqref{eq:berk1} is equivalent
to the logit model,

\begin{equation}\label{eq:berk3}
  \log \left(
  \frac{m_{ \mbox{\scriptsize{Admit}} (ij) }} {m_{ \mbox{\scriptsize{Reject}} (ij) }}
  \right)
  =
  \alpha   +  \beta _i^{\mbox{\scriptsize Dept}}
  +  \beta _j^{\mbox{\scriptsize Gender}}
  \period
\end{equation}
That is, the logit model \eqref{eq:berk3} asserts that department and
gender have additive effects on the odds of admission.
This model may be fit with \PROC{CATMOD}
as shown below, using the variable \pname{admit} as the response
and \pname{dept} and \pname{gender} as predictors.
The option \pname{order=data} is used so that \PROC{CATMOD} will
form the logit for 'Admitted', the category which appears first
in the \Dset.
The \stmt{RESPONSE}{CATMOD} is used
to create an \ODS\
containing observed and fitted logits, which are graphed
(see \exref{ex:berkeley8}) in
\figref{fig:catberk2}.

\begin{listing}
proc catmod order=data
            data=berkeley;
   weight freq;
   response / out=predict;
   model admit = dept gender / ml noiter noprofile ;
\end{listing}

The model fit statistics and parameter estimates for the model
\eqref{eq:berk3} are shown in \outref{out:catberk2.1}.
Note that the \LR\ \GSQ\ for this model is the same as
that for the \loglin\ model $[AD][AG][DG]$
shown in \outref{out:catberk5.1} and in \outref{out:genberk2.1}
and the Wald \chisq\ values for \pname{dept} and \pname{gender}
in \outref{out:catberk2.1}
are similar to the \chisq\ values for the association of each
of these with \pname{admit}
in the \loglin\ model.

\begin{Output}[htb]
\caption{Berkeley admissions data, fit statistics and parameter estimates for the logit model \eqref{eq:berk3}}\label{out:catberk2.1}
\small
\verbatiminput{ch7/out/catberk2.1}
\end{Output}

As in logistic regression models, parameter estimates may be interpreted
as increments in the log odds, or $\exp(\beta)$ may be interpreted
as the multiple of the odds associated with the explanatory categories.
Because \PROC{CATMOD} uses zero-sum constraints,
$\sum \beta_i^{\mbox{\scriptsize   Dept}} =0$ and
$\sum \beta_j^{\mbox{\scriptsize   Gender}} =0$, the parameters for the
last level of any factor
is found as the negative of the sum of the parameters listed.

Thus, $\beta_1^{\mbox{\scriptsize   Gender}} = -0.0499$
is the increment to the log odds of admission for men,%
\footnote{$\beta_1^{\mbox{\scriptsize   Gender}}$ refers to
the first level of \pname{gender} to appear in the \pname{berkeley}
\Dset, because \pname{order=data} was used on the \PROC{CATMOD} statement.}
and therefore $\beta_2^{\mbox{\scriptsize Gender}} = +0.0499$ for women.
Overall, but controlling for department, women were $\exp(2 \times 0.0499) = 1.105$
times as likely to be admitted to graduate school than male applicants
in 1973.
The logit parameters for \pname{dept} in \outref{out:catberk2.1} decrease
over departments A--E; the value for department F is $-(1.274 + 1.231 + \cdots
-0.465) = -2.032$.
These values correspond to the decline in the fitted logits
over department seen in \figref{fig:catberk2}.

Logit models are easier to interpret than the corresponding \loglin\
models because there are fewer parameters,
and because these parameters pertain to the odds of a response category
rather than to cell frequency.
Nevertheless, interpretation is often easier still from a graph than from the
parameter values.
\end{Example}
