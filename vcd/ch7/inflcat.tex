\subsection{Model diagnostics with \PROC{CATMOD}}\label{sec:loglin-inflcat}

For situations where an equivalent model cannot be fit with \PROC{GENMOD},
the hat values and Cook's D can still be obtained from the \ODS\
from \PROC{CATMOD}.
The technique
described here is based on the idea \citep{Christensen:97}
that a \loglin\ model is essentially a regression model for log
frequency, which can be fit by weighted least squares regression.
The steps are:
\begin{enumerate}

\item Fit the \loglin\ model, obtaining the cell frequencies
	\(n_i\) and estimated expected frequencies \(\widehat{m}_i\) in a
	\Dset\ (see the statement {\tt RESPONSE / OUT=PREDICT}
	above).

\item Use \PROC{REG} to calculate a weighted regression and obtain
       the regression diagnostics (e.g., \pname{h} and \pname{cookd})
		 with an \stmt{OUTPUT}{REG}.  Fit the regression with:

	\begin{description}
	\item[independent variables:]  Dummy variables for all
			marginals and association terms in the \loglin\
			model. (\PROC{REG} does not provide a CLASS statement.)
	\item[dependent variable:]  The ``working'' response, \(y_i  =
			\log ( \widehat{m}_i )  +  ( n_i - \widehat{m}_i ) /  \widehat{m}_i\)
	\item[weights:]  \(\widehat{m}_i\)
	\end{description}

\item Leverages will be correctly reported in the output.  The
	adjusted residuals, however, will have been divided by
	\(\sqrt{MSE}\), and the Cook's D values will have been divided
	by \(MSE\).  For a model with \(k\) parameters and \(N\) cells
	the average leverage will be \(k/N\), so a value \(h_{ii} > 2 k
	/  N\) would be considered ``large''.

\end{enumerate}

\begin{Example}[berkeley10]{Berkeley admissions}
We continue with the results from model \([AD]  [GD]\),
to illustrate the computations with the results from \PROC{CATMOD}.
The expected frequencies were obtained
using the option \pname{PRED=FREQ} on the \stmt{MODEL}{CATMOD}, as 
\begin{listing}
proc catmod order=data data=berkeley;
   weight freq;
   model dept*gender*admit=_response_ /
         ml  noiter noresponse nodesign noprofile pred=freq ;
   response / out=predict;
   loglin admit|dept dept|gender / title='Model (AD,DG)';
\end{listing}
 In the \pname{PREDICT} \Dset, \(\widehat{m}_i\) is named
\verb|_PRED_|, \(n_i\) is \verb|_OBS_|, and \(e_i = n_i - \widehat{m}_i\)
is \verb|_RESID_|.
The working response may be calculated in a \Dstp\ as follows:
\begin{listing}
data rdat;
   set predict;
   drop _sample_ _type_ _number_;
   where (_type_='FREQ');
   cell = trim(put(dept,dept.)) ||
          gender ||
          trim(put(admit,yn.));
    *-- Working response;
   y = log(_pred_) + _resid_/_pred_;
\end{listing}
Fitting
the regression model for the working response using \PROC{REG} is
conceptually simple, though tedious because \PROC{REG} 
(like \PROC{LOGISTIC}) 
cannot generate the dummy variables itself.  This may be
done in a data step, or with \PROC{GLMMOD}, or using the
\macro{DUMMY} (\macref{mac:dummy}) and the \macro{INTERACT} (\macref{mac:interact}).
We illustrate using the macro programs; by default they append
the new variables to the input \Dset.
\begin{listing}
%dummy(data=rdat, var=admit gender dept, prefix=a g d);
%interact(v1=a0, v2=d1 d2 d3 d4 d5, prefix=ad);
%interact(v1=gf, v2=d1 d2 d3 d4 d5, prefix=gd);

proc reg data=rdat outest=est;
   id cell;
   weight _pred_;
   model y = a0 gf d1-d5 ad11-ad15 gd11-gd15;
   output out=regdiag
          h=hat cookd=cookd student=studres;
\end{listing}
In the final step, the $\sqrt{MSE}$ is obtained from the \pname{OUTEST}
\Dset, and the adjusted residuals and Cook's D may be calculated.
\begin{listing}
data regdiag;
   set regdiag;
   retain _rmse_;
   if _n_=1 then set est(keep=_rmse_ );
   adjres = studres * _rmse_;
   cookd = cookd * _rmse_**2;
\end{listing}
These quantities, shown in \outref{out:catberk4.1},
may then be plotted in the forms shown earlier using the
\macro{INFLGLIM}.

\begin{Output}[htb]
\caption{Diagnostics for the model $[AD][GD]$ calculated from \PROC{CATMOD} output}\label{out:catberk4.1}
\small
\verbatiminput{ch7/out/catberk4.1}
\end{Output}
\end{Example}


