\begin{Example}[berkeley5]{Berkeley admissions}
Data on admission to the six largest graduate departments at
Berkeley was examined graphically in \chref{ch:twoway}
and in \chref{ch:mosaic}.  The data are
contained in the \Dset\ \pname{berkeley}, listed in
\datref{dat:berkeley}.
The \loglin\ model \eqref{eq:berk1} can be fit to this data
with \PROC{CATMOD} as shown below.

\begin{listing}
proc catmod order=data data=berkeley;
   format dept dept. admit admit.;
   weight freq;
   model dept*gender*admit=_response_ /
         ml noiter noresponse nodesign noprofile pred=freq ;
   loglin admit|dept|gender @2 / title='Model (AD,AG,DG)';
 run;
\end{listing}
On the \stmt{LOGLIN}{CATMOD}, the ``bar'' notation (\pname{admit|dept|gender @2}) means all terms up to two-way associations.
The printed output includes the table of fit statistics shown in \outref{out:catberk5.1}, which indicates that only
the two-way terms \pname{DEPT*ADMIT} and \pname{DEPT*GENDER} are significant.  In
particular, there is no association between Gender and Admission,
controlling for Department.
Several models may be fit within one \PROC{CATMOD} step.
We drop the \pname{GENDER*ADMIT} term
in the following model, giving the model fit statistics in
\outref{out:catberk5.2}.

\begin{listing}
   loglin admit|dept dept|gender / title='Model (AD,DG)';
 run;
\end{listing}
which gives the fit statistics shown in \outref{out:catberk5.2}.

\begin{Output}[htb]
\caption{Berkeley admissions data: Model [AD] [AG] [DG], fit with \PROC{CATMOD}}\label{out:catberk5.1}
\small
\verbatiminput{ch7/out/catberk5.1}
\end{Output}

\begin{Output}[htb]
\caption{Berkeley admissions data: Model [AD] [DG], fit with \PROC{CATMOD}}\label{out:catberk5.2}
\small
\verbatiminput{ch7/out/catberk5.2}
\end{Output}

The fit of the model $[AD] [DG]$ is not much worse than that of the
model $[AD] [AG] [DG]$.
Nevertheless, neither model fits very well, as judged by the
\LR\ \GSQ\ statistics.
We will see why in the next Example.
\end{Example}
