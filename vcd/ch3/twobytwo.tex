\subsection{2 by 2 tables}\label{sec:twoway-twobytwo}

The $2 \times 2$ \ctab{} of applicants to Berkeley graduate programs in \tabref{tab:berk22} may be regarded as an example of a
\glossterm{cross-sectional study}.
The total of $n = 4,526$ applicants in 1973 has been classified by both
gender and admission status.
Here, we would probably consider the total $n$ to be fixed,
and the cell frequencies $n_{ij},  \: i=1,2; j=1,2$
would then represent a single
\glossterm{multinomial sample} for the cross-classification by two
binary variables,
with probabilities cell $p_{ij},  \: i=1,2; j=1,2$
such that
\begin{equation*}
 p_{11} + p_{12} + p_{21} + p_{22} = 1
 \period
\end{equation*}
The basic null hypothesis of interest for a multinomial sample is that
of independence.  Are admission and gender independent of each other?

Alternatively, if we consider admission the response variable, and
gender an explanatory variable, we would treat the numbers of male
and female applicants as fixed
and consider the cell frequencies to represent two independent
\glossterm{binomial samples} for a binary response.
In this case, the null hypothesis is described as that of homogeneity
of the response proportions across the levels of the explanatory variable.

\subsubsection{Odds and odds ratios}\label{sec:twoway-odds}
\ixon{odds ratio}
Measures of association are used to quantify the strength of association
between variables.  Among the many measures of association for
\ctabs, the \glossterm{odds ratio} is particularly useful for
$2 \times 2$ tables, and is a fundamental parameter in several
graphical displays and models described later.
Other measures of strength of association for $2 \times 2$ tables
are described in \CDAref{\C 2} and \citet[{}\S 2.2]{Agresti:96}.

For a binary response, where the probability of a ``success'' is $\pi$,
the \emph{odds} of a success is defined as
\begin{equation*}
 \textrm{odds} = \frac{\pi}{1-\pi} \period
\end{equation*}
Hence, $\textrm{odds} = 1$ corresponds to $\pi = 0.5$, or success and
failure equally likely.   When success is more likely than failure
$\pi > 0.5$, and the $\textrm{odds} > 1$;  for instance, when $\pi = 0.75$,
$\textrm{odds} = .75/.25 =3$, so a success is three times as likely
as a failure.  When failure is more likely, $\pi < 0.5$, and the $\textrm{odds} < 1$;  for instance, when $\pi = 0.25$,
$\textrm{odds} = .25/.75 =\frac{1}{3}$.

The odds of success thus vary multiplicatively around 1.  Taking logarithms
gives an equivalent measure which varies additively around 0, called the
\emph{log odds} or \emph{logit}
\begin{equation*}
 \logit (\pi) \equiv \log (\mbox{odds}) = \log \left( \frac{\pi}{1-\pi} \right)
 \period
\end{equation*}
The logit is symmetric about $\pi = 0.5$, in that
$\logit (\pi) = - \logit (1-\pi)$.

A binary response for two groups gives a $2 \times 2$ table, with
Group as the row variable, say.  Let $\pi_1$ and $\pi_2$ be the
success probabilities for Group 1 and Group 2.  The \emph{odds ratio}
is just the ratio of the odds for the two groups:
\begin{equation*}%\label{eq:oddsratio}
 \mbox{odds ratio} \equiv \theta =
 \frac{\mbox{odds}_1} {\mbox{odds}_2} =
 \frac{\pi_1 / (1-\pi_1)} {\pi_2 / (1-\pi_2)}
 \period
\end{equation*}

Like the odds itself, the odds ratio is always non-negative, between
0 and $\infty$.  When $\theta = 1$, the distributions of success and
failure are the same for both groups (so $\pi_1 = \pi_2$);  there is
no association between row and column variables, or the response
is independent of group.
When $\theta > 1$, Group 1 has a greater success probability;
when $\theta < 1$, Group 2 has a greater success probability.

Similarly, the odds ratio may be transformed to a log scale, to give
a measure which is symmetric about 0.
The \emph{log odds ratio}, symbolized by $\psi$, is just the difference
between the logits for Groups 1 and 2:
\begin{equation*}%\label{eq:logoddsratio}
 \mbox{log odds ratio} \equiv \psi
 = \log (\theta)
 = \log \left[ \frac{\pi_1 / (1-\pi_1)} {\pi_2 / (1-\pi_2)} \right]
 = \logit (\pi_1) - \logit(\pi_2)
 \period
\end{equation*}
Independence corresponds to $\psi =0$, and reversing the rows or columns
of the table merely changes the sign of $\psi$.

For sample data, the \emph{sample odds ratio} is the ratio of the sample
odds for the two groups:
\begin{equation}\label{eq:soddsratio}
 \hat{\theta} =  \frac{p_1 / (1-p_1)} {p_2 / (1-p_2)} =
 \frac{ n_{11} / n_{12} }{ n_{21} / n_{22}} =
 \frac{ n_{11} n_{22} } {n_{12} n_{21}}
 \period
\end{equation}

We described the odds ratio for a sampling context of independent binomial
samples, but actually, the odds ratio is an appropriate measure of strength
of association for all the standard sampling schemes, because it treats
the variables symmetrically.  It does not matter whether the row or column
variable is the response, or whether both variables are treated as
responses.  Other measures of strength of association, not described here,
\emph{do} distinguish between explanatory and response variables.

The sample estimate $\hat{\theta}$ in \eqref{eq:soddsratio} is the
maximum likelihood estimator of the true $\theta$.
The sampling distribution of $\hat{\theta}$ is asymptotically normal
as $n \rightarrow \infty$, but may be highly skewed in small to
moderate samples.  Consequently, inference for the odds ratio
is more conveniently carried out in terms of the log odds ratio,
whose sampling distribution is more closely normal, with mean
$\psi = \log (\theta)$, and asymptotic standard error (ASE)
\begin{equation}\label{eq:aselogtheta}
 \mbox{ASE }_{\log (\theta)} \equiv \hat{s} (\hat{\psi} ) =
 {\left\{
 \frac{1}{n_{11}} + \frac{1}{n_{12}} + \frac{1}{n_{21}} + \frac{1}{n_{22}}
 \right \} }^{1/2}
 =   {\left\{ \sum \sum n_{ij}^{-1} \right \} }^{1/2}
\end{equation}
A large-sample $100(1-\alpha)$\% confidence interval for $\log (\theta)$ may therefore
be calculated as $\log (\theta) \pm z_{1-\alpha/2} \, \mbox{ASE }_{\log (\theta)}$,
where $z_{ 1 - \alpha  / 2 }$ is the cumulative normal quantile with
$1-\alpha/2$ in the lower tail.
Confidence intervals for $\theta$ itself are obtained by exponentiating
the end points of the interval for $\log (\theta)$.

However, $\hat{\theta}$ is 0 or $\infty$ if any $n_{ij}=0$.
\citet{Haldane:55} and \citet{GartZweiful:67} showed that improved
estimators of $\theta$ and $\psi = \log (\theta)$ are obtained by
replacing each $n_{ij}$ by $[n_{ij} + \frac{1}{2}]$ in \eqref{eq:soddsratio}
and \eqref{eq:aselogtheta}.
This adjustment is preferred in small samples, and required if any
zero cells occur.  In large samples, the effect of adding 0.5 to each
cell becomes negligible.


\begin{Example}[berkeley1a]{Berkeley admissions}
Odds ratios and many other measures of association not described here
are produced with \PROC{FREQ} when you specify the \opt{MEASURES}{FREQ}
in the \stmt{TABLES}{FREQ}.
For the Berkeley admissions data, the frequency table in \outref{out:berkfreq.1} and the various measures of association
are produced by these statements:
%% input: /Users/friendly/sasuser/catdata/berkfreq.sas
%% last modified: 14-Jul-99  9:36
\begin{listing}
%include catdata(berkeley);

proc freq data=berkeley order=data;
   weight freq;
   tables gender*admit / nocol measures;
   format admit admit. gender $sex.;
\end{listing}

\begin{Output}[htb]
\caption{Admission to Berkeley graduate programs: Odds ratio and relative risk}\label{out:berkfreq.2}
\small
\verbatiminput{ch\thechapter/out/berkfreq.2}
\end{Output}
The odds ratio is displayed in a section of the output labeled
``Estimates of the Relative Risk'', as the value associated
with a Case-Control study.  This portion of the output is shown in
\outref{out:berkfreq.2}.
The value $\hat{\theta} = 1.84 = (1198 \times 1278) / (557 \times 1493)$ indicates that males are nearly
twice as likely to be admitted as females.
We describe a visualization method for odds ratios in $2 \times 2$
tables in \secref{sec:twoway-fourfold}
and return to the Berkeley data in \exref{ex:berkeley2}.
See \CDAref{\S 2.4--2.5} for discussion of relative risk and other
measures of association in $2 \times 2$ tables.
\end{Example}
\ixoff{odds ratio}