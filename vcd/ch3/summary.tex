\section{Chapter summary}
\begin{itemize}
  \item A \ctab{} gives the frequencies of observations
  cross-classified by two or more categorical variables.
  Different types of variables may be distinguished, such as
  response, explanatory and stratifying variables.
  With such data we are typically interested in testing whether
  associations exist, quantifying the strength of association,
  and understanding the nature of the association among these
  variables.

  \item $2 \times 2$ tables may be easily summarized in terms of
  the odds ratio or its logarithm.

  \item Tests of general association between two categorical variables are
  most typically carried out using the Pearson's chi-square or
  \LR{} tests provided by \PROC{FREQ}.
  Stratified tests controlling for one or more background variables, and
  tests for ordinal categories are provided by the
  Cochran-Mantel-Haenszel tests.

  \item For $2 \times 2$ tables, the fourfold display provides a
  visualization of the association between variables in terms of
  the odds ratio.  Confidence rings provide a visual test of
  whether the odds ratio differs significantly from 1. Stratified
  plots for $2 \times 2 \times k$ tables are also provided by the \sasprog{FOURFOLD}.

%  \item Tukey two-way plots attempt to show the association
%  between two categorical variables as deviations from an additive
%  relation of the log frequencies.

  \item Sieve diagrams and association plots provide other useful displays of the pattern of association
  in $r \times c$ tables.

  \item When the row and column variables represent different
  observers rating the same subjects, interest is focused on
  agreement rather than mere association.  Cohen's $\kappa$ is one
  measure of strength of agreement.  The observer agreement chart
  provides a visual display of how the observers agree and
  disagree.

  \item Another specialized display, the trilinear plot is useful
  for three-column frequency tables or compositional data.
\end{itemize}
