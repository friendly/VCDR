\section{Stratified analysis}\label{sec:twoway-strat}
An overall analysis ignores other variables (like sex), by
collapsing over them.  It is possible that the treatment is effective
only for one gender, or even that the treatment has opposite effects
for men and women.
\ixd{arthritis treatment}

A stratified analysis:

\begin{itemize}

\item controls for the effects of one or more background variables.
       This is similar to the use of a blocking variable in an ANOVA
       design.

\item is obtained by including more than two variables in the {\tt
       tables} statement.  List the stratification variables {\bf
       first}.  To examine the association between TREAT and IMPROVE,
       controlling for both SEX and AGE (if available):

\begin{equation*}
   \mbox{\texttt{tables }}
   \overbrace{\rule{0in}{1.5ex}\mbox{\texttt{ age * sex }}}^{\mbox{\scriptsize stratify by}}
   \mbox{\texttt{ * }}
   \overbrace{\rule{0in}{1.5ex}\mbox{\texttt{ treat }}}^{\mbox{\scriptsize explanatory}}
   \mbox{\texttt{ * }}
   \overbrace{\rule{0in}{1.5ex}\mbox{\texttt{ improve;}}}^{\mbox{\scriptsize response}}
\end{equation*}

\end{itemize}

\begin{Example}[arthrit3]{Arthritis treatment}
The statements below request a stratified analysis of the arthritis
treatment data
with CMH tests,
controlling for sex.

{\small
\begin{verbatim}
*-- Stratified analysis, controlling for sex;
proc freq order=data;
   weight count;
   tables sex * treat * improve / cmh chisq nocol nopercent;
   run;
\end{verbatim}
}

\PROC{FREQ} gives a separate table for each level of the stratification
variables (\outref{out:arthfreq.3} and \outref{out:arthfreq.4}), plus overall (partial) tests controlling for the
stratification variables (\outref{out:arthfreq.5}).
\ixd{arthritis treatment}

\begin{Output}[htb]
\caption{Arthritis treatment data, stratified analysis}\label{out:arthfreq.3}
\small
\verbatiminput{ch3/out/arthfreq.3}
\end{Output}

Note that the strength of
association%
\glosstex{meas of assoc}
between treatment and outcome is quite
strong for females (\outref{out:arthfreq.3}).  In contrast, the results for males (\outref{out:arthfreq.4}) shows
a not-quite significant association, even by the 
more powerful Mantel-Haenszel test.
However,
note that there are too few males for the general association
\(\chi^2\) tests to be reliable (the statistic does not follow the
theoretical \(\chi^2\) distribution).
\begin{Output}[htb]
\caption{Arthritis treatment data, stratified analysis}\label{out:arthfreq.4}
\small
\verbatiminput{ch3/out/arthfreq.4}
\end{Output}

The individual tables are followed by the (overall) partial tests of
association controlling for sex, shown in \outref{out:arthfreq.5}.  Unlike the tests for each stratum,
these tests {\bf do not} require large sample size in the individual
strata -- just a large total sample size.  Note that the \(\chi^2\)
values here are slightly larger than those from the initial analysis
that ignored sex.
\ixd{arthritis treatment}

\begin{Output}[htb]
\caption{Arthritis treatment data, stratified analysis}\label{out:arthfreq.5}
\small
\verbatiminput{ch3/out/arthfreq.5}
\end{Output}
\end{Example}

\subsection{Assessing homogeneity of association}
In a stratified analysis
it is often of interest to know if the association between the
primary table variables is the same over all strata.  For \(k \times
2 \times  2\) tables this question reduces to whether the \IX{odds ratio} is
the same in all k strata, and \PROC{FREQ} computes the \IX{Breslow-Day test}
for homogeneity of odds ratios
when you use the \opt{measures}{FREQ} on the {\tt
tables} statement.  \PROC{FREQ} cannot perform tests of homogeneity for
larger tables, but these can be easily done with the \texttt{CATMOD}
procedure.
\ix{homogeneity of association}

\begin{Example}[arthrit4]{Arthritis treatment}
For the arthritis data, homogeneity means that there is no three-way
Sex * Treatment * Outcome association.  That is, the association
between treatment and outcome (\texttt{improve})
is the same for both men and women.
This hypothesis can be stated
as the \loglin\ model,
\begin{equation}\label{eq:STO2}
 \textrm{[SexTreat] [SexOutcome] [TreatOutcome]}
 \period
\end{equation}
This notation (described in \secref{sec:loglin-counts})
lists only the high-order association
terms in a linear model for log frequency.
Thus, the model \eqref{eq:STO2}
allows associations between sex and treatment (e.g., more males
get the Active treatment), between sex and outcome (e.g. females
are more likely to show marked improvement),
and between treatment and outcome,
but no three-way association.
In the \PROC{CATMOD} step
below, the \stmt{LOGLIN}{CATMOD} specifies this \loglin\  model
as \pname{sex|treat|improve@2} (where \texttt{improve} is the Outcome variable)
which means ``all terms up to 2-way
associations''.

\begin{listing}
title2 'Test homogeneity of treat*improve association';
data arth;
   set arth;
   if count=0 then count=1E-20;
proc catmod order=data;
   weight count;
   model sex * treat * improve = _response_ /
         ml noiter noresponse nodesign nogls ;
   loglin sex|treat|improve@2 / title='No 3-way association';
run;
   loglin sex treat|improve   / title='No Sex Associations';
\end{listing}
\ixd{arthritis treatment}

(Frequencies of zero can be regarded as either ``structural
zeros''---a cell which could not occur, or as ``sampling zeros''---a
cell which simply did not occur.  \PROC{CATMOD} treats zero frequencies
as ``structural zeros'', which means that cells with \texttt{count = 0}
are excluded from the analysis.  The DATA step above replaces the one
zero frequency by a small number.)
\ix{structural zeros}
\ix{sampling zeros}
\ix{zeros!structural}
\ix{zeros!sampling}

In the output from \PROC{CATMOD}
shown in \outref{out:arthfreq.7}, the likelihood ratio \(\chi^2\) (the
badness-of-fit for the No 3-Way model) is the test for homogeneity
across sex.  This is clearly non-significant, so the
treatment-outcome association can be considered to be the same for
men and women.

\begin{Output}[htb]
\caption{Arthritis treatment data, testing homogeneity}\label{out:arthfreq.7}
\small
\verbatiminput{ch3/out/arthfreq.7}
\end{Output}

Note that the associations of sex with treatment and sex with outcome
are both small and of borderline significance, which suggests a
stronger form of homogeneity, the log-linear model [Sex]
[TreatOutcome] which says the only association is that between
treatment and outcome.  This model is tested by the second {\tt
loglin} statement given above, which produced the results shown
in \outref{out:arthfreq.8}.
The likelihood ratio test indicates that this model might provide a
reasonable fit.
\ix{homogeneity of association}
\begin{Output}[htb]
\caption{Arthritis treatment data, testing homogeneity}\label{out:arthfreq.8}
\small
\verbatiminput{ch3/out/arthfreq.8}
\end{Output}
\end{Example}
