\section{Chapter summary}
\begin{itemize}
\item The mosaic display depicts the frequencies in a \ctab{} by a collection of rectangular ``tiles''
whose area is proportional to the cell frequency.
The residual from a specified model is portrayed by shading the tile
to show the sign and magnitude of the deviation from the model.

\item For two-way tables, the tiles for the second variable align
at each level of the first variable when the two variables are independent
(see \figref{fig:soccer2}).

\item The perception and understanding of patterns of association
(deviations from independence) are enhanced by reordering the
rows or columns to give the residuals an opposite-corner pattern.

\item For three-way and larger tables, a variety of models can be fit
and visualized.
Starting with a minimal baseline model, the pattern of residuals
will often suggest additional terms which must be added to
``clean the mosaic.''
\item It is often useful to examine the sequential mosaic
displays for the marginal subtables with the variables in a given order.
Sequential models of joint independence provide a breakdown of the
total association in the full table, and are particularly
appropriate when the last variable is a response.

\item Partial association, which refers to the associations among
a subset of variables, within the levels of other variables,
may be easily studied by constructing separate mosaics for the subset
variables for the levels of the other, ``given'' variables.
These displays provide a breakdown of a model of conditional association
for the whole table, and serve as an analog of coplots for quantitative
data.

\item Mosaic matrices, consisting of all pairwise plots of an $n$-way
table, provide a way to visualize all marginal, joint, or conditional 
relations simultaneously.

\item The structural relations among model terms in various \loglin{}
models themselves can also be visualized by mosaic matrices
showing the expected, rather than observed, frequencies under different
models.
\end{itemize}

