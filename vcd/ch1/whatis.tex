\section{What is categorical data?}\label{sec:intro-whatis}

A \emph{categorical variable} is one for which the possible measured
or assigned values
consist of a discrete set of categories.
Some typical examples are:
\begin{itemize*}
\item ``Gender'', with categories ``Male'', ``Female''.
\item ``Marital status'', with categories ``Never married'', ``Married'',
``Separated'', ``Divorced'', ``Widowed''.
\item ``Fielding position'' (in baseball), with categories
``Pitcher'', ``Catcher'', ``1st base'', ``2nd base'',  $\dots$, ``Left field''.
\item ``Side effects'' (in a pharmacological study), with categories
``None'', ``Skin rash'', ``Sleep disorder'', ``Anxiety'', $\dots$.
\item ``Political preference'', with categories ``Left'', ``Center'', ``Right''.
\item ``Treatment outcome'', with categories ``no improvement'', ``some
improvement'', or ``marked improvement''.
\item ``Age'', with categories ``0-9'', ``10-19'', ``20-29'', ``30-39'', 
$\dots$ .
\item ``Number of children'', with categories $0, 1, 2, \dots$ .
\end{itemize*}

As these examples suggest, categorical variables differ in the number of
categories: we often distingish 
\glossterm{binary variables} such as ``Gender''
from those with more than two categories (called \glossterm[polytomous]{polytomous variables}).
For example, \tabref{tab:berk220} gives data on 4526 applicants
to graduate departments at the University of California at Berkeley
in 1973, classified by two binary variables, gender and admission status.
\ixe{Berkeley admissions}
\begin{table}[htb]
\caption{Admissions to Berkeley graduate programs}
\label{tab:berk220}
 \begin{center}
\begin{tabular}{lrr|r}
\hline
  & Admitted & Rejected & Total  \\
\hline
 Males & 1198 & 1493 & 2691  \\
 Females & 557 & 1278 & 1835  \\
\hline
 Total & 1755 & 2771 & 4526  \\
\hline
\end{tabular}
\end{center}
\end{table}

Some categorical variables (``Political preference'', ``Treatment outcome'')
may have ordered categories (and are called \glossterm{ordinal}),
while other (\glossterm{nominal}) variables like ``Marital status''
have unordered categories.%
\footnote{An ordinal variable may be defined as one whose categories are
\emph{unambiguously} ordered along a \emph{single} underlying dimension.
Both marital status and fielding position may be weakly ordered, but
not on a single dimension, and not unambiguously.} 
For example, \tabref{tab:arthrit0} shows a $2 \times 2 \times 3$ table of 
ordered outcomes (``none'', ``some'' or ``marked'' improvement)
to an active treatment for rheumatoid
arthritis compared to a placebo for men and women.
\ixe{Arthritis treatment}
\begin{table}[tb]

\caption{Arthritis treatment data}\label{tab:arthrit0}
\begin{center}
\begin{tabular}{ll|rrr|r}
\hline
     &  & \multicolumn{3}{c|}{Improvement}            &  \\
\hline
   Treatment&  Sex    &None    &Some    &Marked  &  Total \\[1ex]
\hline
   Active   &  Female &      6 &      5 &     16 &     27 \\
            &  Male   &      7 &      2 &      5 &     14 \\ [0.5ex]
%\hline
   Placebo  &  Female &     19 &      7 &      6 &     32 \\
            &  Male   &     10 &      0 &      1 &     11 \\[1ex]
\hline
   Total    &         &     42 &     14 &     28 &     84 \\
\hline
\end{tabular}
\end{center}
\end{table}



Finally, such variables differ in the
fineness or level to which some underlying observation has been
categorized for a particular purpose.
From one point of view, \emph{all} data
may be considered categorical because the precision of measurement
is necessarily finite, or an inherently continuous variable may be recorded only to limited precision.   But this view is not helpful for the applied
researcher because it neglects the phrase ``for a particular purpose''.
Age, for example, might be treated as a quantitative variable in a study of
native language vocabulary, or as an ordered categorical variable in terms of
the efficacy or side-effects of treatment for depression, or even as a
binary variable (``child'' vs.\  ``adult'') in an analysis of survival following
an epidemic or natural disaster.


\subsection{Case form vs.\ Frequency form}
In many circumstances, data is recorded on each individual or experimental
unit.  Data in this form is called case data,
or data in \glossterm{case form}.
The data in \tabref{tab:arthrit0}, for example, were derived from
the individual data listed in \datref{dat:arthrit}.
Whether
or not the data variables, and the questions we ask call for
categorical or quantitative data analysis, we can always trace
any observation back to its individual identifier or data record
when the data are in case form.

Data in \glossterm{frequency form}, such as that shown in \tabref{tab:arthrit0},
has already been tabulated, by counting over the categories of the
table variables.  Data in frequency form may be analyzed by methods
for quantitative data if there is a quantitative response variable
(weighting each group by the cell frequency, with a \texttt{WEIGHT}
or \texttt{FREQ} statement).  Otherwise, such data are generally
best analyzed by methods for categorical data.
In either case, however, an observation in a \Dset\ in
frequency form refers
to all cases in the cell collectively, and cannot be identified individually.
Data in case form can always be reduced to frequency form,
but the reverse is rarely possible.

\subsection{Frequency data vs.\ Count data}
In many cases the observations represent the classifications of events or variables are 
recorded from \emph{operationally independent} experimental units or individuals, typically
a sample from some population.  The tabulated data may be called
\glossterm{frequency data}.  The data in \tabrefs{tab:berk220,tab:arthrit0}
are both examples of frequency data because each observation tabulated
comes from a different person.

However, if several events or variables are observed for the same units or individuals, those events are not
operationally independent, and it is useful to use the term 
\glossterm{count data} in this situation.  These terms (following
\citet{Lindsey:95}) are by no means standard, but
the distinction is often important, particularly in statistical
models for categorical data.  In a tabulation of the number of male
children within families (\tabref{tab:saxdata}), for example,
the number of male children in a given family would be a count variable,
taking values $0, 1, 2, \dots$.  The number of independent families with
a given number of male children is a frequency variable.
Count data also arise when we tabulate a sequence of events over time
or under different circumstances in a number of individuals.


\subsection{Univariate, bivariate, and multivariate data}
\tabref{tab:berk220} is an example of a bivariate (two-way) \ctab\
and \tabref{tab:arthrit0} classifies the observations by three variables.
Yet, we will see that the Berkeley admisssions data also recorded
the department to which potential students applied (giving a three-way
table), and in the arthritis data, the age of subjects was also
recorded.

Any \ctab, therefore records the marginal totals, summed over all
variables not represented in the table.
For data in case form, this means simply ignoring (or not recording)
one or more variables;  the ``observations'' remain the same.
Data in frequency form, however, result in smaller tables when
any variable is ignored;  the ``observations'' are the cells of
the \ctab.

In the limiting case, only one table variable may be recorded or
available, giving the categorical equivalent of univariate data.
For example, \tabref{tab:saxdata} gives data on the distribution
of the number of male children in families with 12 children
discussed in \exref{ex:saxony1}.
These data were part of a large tabulation of the sex distribution
of families in Saxony in the 19th century, but the data in \tabref{tab:saxdata}
have only one discrete classification variable, number of males.
Without further information, the only statistical questions concern
the form of the distribution.
We discuss methods for fitting and graphing such discrete distributions
in \chref{ch:discrete}.
The remaining chapters relate to bivariate and multivariate data.
\ixe{Families in Saxony}
\begin{table}[htb]
 \caption{Number of Males in 6115 Saxony Families of Size 12}\label{tab:saxdata}
 \begin{center}
 \begin{tabular}{lrrrrrrrrrrrrr}
  \hline
  Males & 0 & 1 & 2 & 3 & 4 & 5 & 6 & 7 & 8 & 9 & 10 & 11 & 12 \\ 
  Families & ~~~3 & ~~24 & ~104 & ~286 & ~670 & 1033 & 1343 & 1112 & ~829 & ~478 & 181 & ~~45 & ~~~7 \\ 
  \hline
 \end{tabular}
 \end{center}
\end{table}


\subsection{Explanatory vs.\ Response variables}
\ix{variable!response \~|(}
Many statistical models make a distinction between \emph{response}
(or \emph{dependent}, or \emph{criterion})
variables and
\emph{explanatory}
(or \emph{independent}, or \emph{predictor})
variables.
In the standard (classical) linear models for regression and analysis of variance
(ANOVA), for instance, we treat one (or more) variables as responses,
to be explained by the other, explanatory variables.
The explanatory variables may be quantitative or categorical
(e.g., \texttt{CLASS} variables), but
this affects only the details of how the model is specified for
\PROC{GLM} or \PROC{REG}.
The response variable, treatment outcome, for example, must be
considered quantitative,  and the model attempts to describe how the
\emph{mean} of the distribution of responses changes with the values
or levels of the explanatory variables, such as age or gender.


When the response variable is categorical, however, the standard linear
models do not apply, because they assume a normal (Gaussian) distribution
for the model residuals.  For example, in \tabref{tab:arthrit0} 
the response is Improvement, and even if numerical scores were assigned
to the categories ``none'', ``some'', ``marked'', it may be unlikely
that the assumptions of the classical linear models could be met.

Hence, a categorical \emph{response variable} generally requires analysis
using methods for categorical data, but categorical explanatory variables
may be readily handled by either method.
\ix{variable!response \~|)}
