%% Data Visualization quotes
%% file:  /home/friendly/Library/Documents/tex/quotes.tex

\epigraph{You can see a lot, just by looking.}{Yogi Berra}

\epigraph{Every picture tells a story.}{Rod Stewart, 1971}

\epigraph{A picture is worth a thousand words.}{F. Barnard, 1927}

\epigraph{...But it is not always clear \emph{which} 1000 words.}{J. Tukey, 1973}

\epigraph{Did you ever see such a thing as a drawing of a muchness?}{Dormouse, \emph{Alice in Wonderland}}

\epigraph{The critical requirement of an effective graphical display is that
it stimulate spontaneous perceptions of \emph{structure} in data.}{S. Smith \emph{et al.}, 1990}

\epigraph{Like good writing, producing an effective graphical display
requires an understanding of \emph{purpose}---\emph{what} is to be communicated, and to \emph{whom}.}{M. Friendly, 1991}

\epigraph{Have you ever seen voice mail?}{\emph{The Hackers Test}}

\epigraph{When choosing between two evils, I always like to take the one I've never tried before.}{Mae West, 1941}

\epigraph{A picture is worth a thousand numbers.}{Anon}

\epigraph{Graphics is the visual means of resolving logical problems.}{\citet[p. 16]{Bertin:81}}

\epigraph{The greatest value of a picture is when it forces us to notice what we never expected to see.}{\citet[p. vi]{Tukey:77}}

\epigraph{If one technique of data analysis were to be exalted above all others for its ability to be revealing to the mind in connection
with each of many different models, there is little doubt which one would be chosen.
The simple graph has brought more information to the data analyst's mind than any other device.
It specializes in providing indications of unexpected phenomena.}{\citep[p. 49]{Tukey:1962}}
 
\epigraph{Genius seems to consist merely in trueness of sight.}{Ralph Waldo Emerson, 1840}

\epigraph{The eye obeys exactly the action of the mind.}{Ralph Waldo Emerson, 1860}

\epigraph{Vision is the art of seeing things invisible.}{Johnathan Swift, 1711}

\epigraph{When there is no vision, the people perish.}{Proverbs 29:18}

\epigraph{If I can't picture it, I can't understand it.}{Albert Einstein}

\epigraph{And those who have insight will shine brightly like the brightness of the expanse of Heaven.}{Daniel 12:3}

\epigraph{The one thing that marks the true artist is a clear perception
and a firm, bold hand, in distinction from that imperfect mental vision
and uncertain truth which give up the feeble pictures and the lumpy statues
of the mere artisans on canvas or in stone.}{Oliver Wendell Holmes (1860), \emph{The Professor at the Breakfast Table}, Ticknor and Fields, Boston, MA}

\epigraph{Programming graphics in X is like finding the square root of $\pi$ using
Roman numerals.}{Henry Spencer}

\epigraph{I like your motto: One picture is worth 1,000 denials.}{Ronald Reagan to White House News Photographers Assn, 18 May 1983}

\epigraph{Getting information from a table is like extracting sunlight from a cucumber.}{Farquhar \& Farquhar, 1891} % \cite{FaquharFaquhar:1891}

\epigraph{With brush you paint the possibilities; with pens you scribe the probabilities; for in pictures we find insight; while in numbers find we strength.}{Forrest W. Young}

\epigraph{A graphic should not only show the leaves; it should show the
branches as well as the entire tree.}{\cite[preface]{Bertin:83}}

\epigraph{Tables are like cobwebs, like the
sieve of Danaides; beautifully reticulated, orderly to look upon, but
which will hold no conclusion. Tables are abstractions, and the object a most
concrete one, so difficult to read the essence of.}{From Chartism by Thomas Carlyle (1840), Chapter II, Statistics}

\epigraph{A judicious man looks at Statistics, not to get knowledge, 
but to save himself from having ignorance foisted on him.}{From Chartism by Thomas Carlyle, Chapter II, Statistics}

\epigraph{Although geometrical representations of propositions in the
thermodynamics of fluids are in general use and have done good service in 
disseminating clear notions in this science, yet they have by no means received
the extension in respect to variety and generality of which they are capable.}{J. Willard Gibbs, \emph{Graphical Methods in the Thermodynamics of Fluids}, 1873 \citeyear{Gibbs:1873a}}

\epigraph{Although we often hear that data speak for themselves, their voices
can be soft and sly.}{Mosteller, Fienberg and Rourke, Beginning Statistics with Data Analysis, 1983, Reading MA, p. 234}

\epigraph{Nocturne, of Chopin, so beautiful music. But few people will appreciate the music if I just show them the notes. Most of us need to listen to the music to understand how beautiful it is.
But often that's how we present statistics; we just show the notes, we don't play the music.}{Hans Rosling, OECD World Forum, Istanbul, June 2007}

\epigraph{The greatest possibilities of visual display lie in vividness and inescapability of the intended message.
A visual display can stop your mental flow in its tracks and make you think.
A visual display can force you to notice what you never expected to see.}{John W. Tukey \cite{Tukey:90}}

\epigraph{The purpose of [data] display is comparison (recognition of phenomena), not numbers
... The phenomena are the main actors, numbers are the supporting cast.}{John W. Tukey \cite{Tukey:90}}

\epigraph{If an editor should print bad English he would lose his position.
Many editors are using and printing bad methods of graphic presentation,
but they hold their jobs just the same.}{W. C. Brinton, \emph{Graphic methods of presenting facts}, 1914, p. 3}

\epigraph{
Around the turn of the century, Karl Pearson, an almost elemental force for more and better statistical thought
in all areas of life, including with gusto, matters of social policy, was thinking and lecturing about 
graphical methods. But later in Pearson's life, and certainly in the careers of R. A. Fisher and the
other great statistical minds of the first half of the century, there was a falling away of interest in graphics
and an efflorescence  of devotion to analytical mathematical methods.
Indeed, for many years there was a contagious \emph{snobbery} against so unpopular, vulgar
and elementary a topic as graphics among academic statisticians and their students''}{\citep[p. 144, italics in original]{Kruskal:1978}}

\epigraph{[When] you see excellent graphics, find out how they were done. Borrow strength from demonstrated excellence. 
The idea for information design is: Don't get it original, get it right.}{Edward Tufte}


\epigraph{Si la statistique  graphique, bien que  n�e d'hier, �tend  chaque jour
son  domaine,  c'est  qu'elle remplace  avantageusement  les  longs tableaux  de
chiffres et qu'elle permet, non seulement d'embrasser d'un coup d'oeil la  s�rie
des ph�nom�nes, mais  encore d'en signaler  les rapports ou  les anomalies, d'en
trouver les causes, d'en d�gager les lois.}{\'Emile Cheysson} 

\epigraph{If statistical graphics, although born just yesterday, extends its reach every day, it
is because it replaces long tables of numbers and it allows one not only to embrace
at glance the series of phenomena, but also to signal the correspondences or anomalies, to find
the causes, to identify the laws.}{\'Emile Cheysson, c. 1877}

%% graph people vs. table people

\epigraph{Mr.  Funkhouser  has  made  an  extremely  interesting  and   valuable
contribution to  the history  of statistical  method. I  wish, however,  that he
could have added a  warning, supported by horrid  examples, of the evils  of the
graphical  method  unsupported   by  tables  of   figures.  Both  for   accurate
understanding, and particularly  to facilitate the  use of the  same material by
other people, it is essential that graphs should not be published by themselves,
but only when  supported by the  tables which lead  up to them.  It would be  an
exceedingly good rule to forbid in any scientific periodical the publication  of
graphs unsupported by  tables.}{John Maynard  Keynes, review of  Funkhouser for
The Economic Journal, http://www.jstor.org/stable/2224943}
 
%%%%%%%%%%%%%%%%%%%%%%%%%%%%%%%%%%%%%%%%%%%%%%%%%%%%%%%%%%%%%%%%%%%%%%%%%%%%%%%%%%%%%%%%%%
%% Ellipses

\epigraph{
Mankind is not a circle with a single center but an ellipse with two focal points of which facts are one and ideas the other.}{Victor Hugo}

So, Fabricius, I already have this: that the most true path of the planet [Mars] is an ellipse, which D�rer also calls an oval, or certainly so close to an ellipse that the difference is insensible.
\u2014 Johannes Kepler
Letter to David Fabricius (11 Oct 1605). Johannes Kepler Gesammelte Werke (1937- ), Vol. 15, letter 358, l. 390-92, p. 249.
http://www.todayinsci.com/QuotationsCategories/E_Cat/Ellipse-Quotations.htm


%% Computing

\epigraph{The purpose of computing is insight, not numbers.}{Richard Hamming, \emph{Introduction To Applied Numerical Analysis}}

\epigraph{... to be a good theoretical statistician one must also compute, 
and must therefore have the best computing aids.}{Frank Yates, Sampling Methods for Censuses and Surveys, 1949}

\epigraph{Whatever relates to extent and quantity may be represented by
geometrical figures. Statistical projections which speak to the senses without
fatiguing the mind, possess the advantage of fixing the attention on a great
number of important facts.}{Alexander von Humboldt, \citeyear{Humboldt:1811a}, p. ciii}

\epigraph{Segnius irritant animos demissa per aures,
Quam quae sunt oculus subjecta fidelibus
(Roughly, What we hear excites the mind less than what we see)}{Horace}


%% QUOTES ON STATISTICS -- from http://www.stat.wisc.edu/~limt/quotes


\epigraph{There are two kinds of statistics, the kind
you look up and the kind you make up.}{Rex Stout}

\epigraph{Statistics are like alienists -
they will testify for either side.}{F. H. La Guardia}

\epigraph{You may prove anything by figures.}{Thomas Carlyle}

\epigraph{Thou shalt not sit with statisticians
nor commit a Social Science.}{W.H. Auden}

\epigraph{To understand God's thoughts we must study statistics,
for these are the measure of His purpose.}{Florence Nightingale}

\epigraph{You cannot feed the hungry on statistics.}{David Lloyd George}

\epigraph{A single death is a tragedy, a million deaths is a statistic.}{Stalin}

\epigraph{A judicious man uses statistics, not to get knowledge, but to save himself
from having ignorance foisted upon him.}{Thomas Carlyle}

\epigraph{Statistics are like a bikini.  What they reveal
is suggestive, but what they conceal is vital.}{Aaron Levenstein}

\epigraph{Do not put faith in what statistics say until you have
carefully considered what they do not say.}{William W. Watt}

\epigraph{Facts are stubborn things, but statistics are more pliable.}{Anon}

\epigraph{Statistics are figures used as arguments.}{Leonard L. Levison}

\epigraph{Figures won't lie but liars will figure.}{Charles Grosvenor}

\epigraph{I always find that statistics are hard to swallow
and impossible to digest.  The only one I can remember is
that if all the people who go to sleep in church were laid
end to end they would be a lot more comfortable.}{Mrs Robert A. Taft}

\epigraph{Statistician: Delphic figure who lacks the necessary vocabulary
to converse with mere mortals.}{Rod Nicolson - Psychology Software News}

\epigraph{Get the facts first, and then you can distort them as much as you please.}{Mark Twain}

\epigraph{If you want to inspire confidence, give plenty of statistics. It does not 
matter that they should be accurate, or even intelligible, as long as there 
is enough of them.}{Lewis Carroll}

\epigraph{It is a truth very certain that when it is not in our power to determine
what is true we ought to follow what is most probable.}{Descartes}

\epigraph{Models are to be used, but not to be believed.}{Henry Theill}

\epigraph{A beautiful theory, killed by a nasty, ugly little fact.}{Thomas H. Huxley}


\epigraph{Man must learn to simplify, but not to the point of falsification.}{Aldous Huxley}

\epigraph{Do not put your faith in what statistics say until you have
carefully considered what they do not say.}{William W. Watt}
	
\epigraph{Since small differences in probability cannot be appreciated by the human 
mind, there seems little point in being excessively precise about 
uncertainty.}{Box, G. E. P. & Tiao, G. C. (1973), Bayesian inference in statistical analysis, Addison-Wesley, Reading, MA, p. 65.}


%% Data ----------------------------------------------------

\epigraph{`Data! data!' he cried impatiently.  I can't make bricks without clay.}{Conan-Doyle, \emph{Adventures of SH, The Copper Beeches}}

\epigraph{I have no data yet. It is a capital mistake to theorize before one
has data.}{Conan-Doyle, \emph{Adventures of SH, A Scandal in Bohemia}}

\epigraph{This was an unexpected piece of luck. My data were coming more
quickly than I could have reasonably hoped.}{Conan-Doyle, \emph{Memoirs of SH, The Musgrave Ritual}}


\epigraph{I have not all my facts yet, but I do not think there are
any insuperable difficulties. Still, it is an error to argue in front
of your data. You find yourself insensibly twisting them round to
fit your theories.}{Conan-Doyle, \emph{His Last Bow, Wisteria Lodge}}

 
\epigraph{Not everything that counts can be counted, and not everything that
can be counted counts.}{Albert Einstein}

\epigraph{The only thing we know for sure about a missing data point is that it is not there, and
there is nothing that the magic of statistics can do do change that.  The best that can be managed
is to estimate the extent to which missing data have influenced the inferences we wish to draw.}{Wainer, 2009}

\epigraph{Big data can change the way social science is performed, but will not replace statistical common sense.}{Thomas Landsall-Welfare, ``Nowcasting the mood of the nation,'' Significance, v. 9(4), August 12, 2012, p. 28} 

%% Science --------------------------------------------
\epigraph{Things should be made as simple as possible, but not any simpler}{Albert Einstein}

\epigraph{So much has already been written about everything that you can't find out anything about it.}{James Thurber, 1961}

\epigraph{The practical power of a statistical test is the product
of its' statistical power and the probability of use.}{J. W. Tukey, 1959\nocite{Tukey:59}}

\epigraph{Theory into Practice.}{Mao Tse-Tung, \emph{The Little Red Book}}

\epigraph{``Beauty is truth; truth, beauty.''
  That is all ye know on Earth,
  and all ye need to know.}{John Keats, \emph{Ode on a Grecian urn}}

\epigraph{They consider me to have sharp and penetrating vision because I see
them through the mesh of a sieve.}{Kahlil Gibran}

\epigraph{The journalistic vision sharpens to the point of maximum impact every
event, every individual and social configuration; but the honing is
uniform.}{George Steiner}

\epigraph{Some people weave burlap into the fabric of our lives, and some weave
gold thread. Both contribute to make the whole picture beautiful and
unique.}{Anon.}

\epigraph{Time extracts various values from a painter's work. When these values
are exhausted the pictures are forgotten, and the more a picture has to
give, the greater it is.}{Henri Matisse}

\epigraph{Un croquis vaut mieux qu�un long discours.
Tr.: \emph{A picture is worth a thousand words.}.}{Napoleon}

\epigraph{God is in the details.}{Mies van der Rohe, \emph{New York Times}, August 19, 1969}

\epigraph{The devil is in the details.}{George Schultz} % referring to the intracies of the SALT talks in a speech to the Council of Foreign Relations

\epigraph{One has to be able to count if only so that at fifty one doesn't marry
a girl of twenty.}{Maxim Gorky, \emph{The Zykovs}, 1914}

\epigraph{A man has one hundred dollars and you leave him with two dollars,
that's subtraction.}{Mae West, \emph{My Little Chickadee}, 1940}

\epigraph{In the fields of observation chance favors only the prepared mind.}{Louis Pasteur}

\epigraph{The eye of a human being is a microscope, which makes the world seem
bigger than it really is.}{Kahlil Gibran, \emph{A Handful of Sand on the Shore}}

\epigraph{To the man who only has a hammer in the toolkit, every problem looks
like a nail.}{Abraham Maslow}

\epigraph{Four hostile newspapers are more to be feared than a thousand bayonets.}{Napoleon, Maxims}

\epigraph{When I'm working on a problem, I never think about beauty.
I think only how to solve the problem. But when I have finished,
if the solution is not beautiful, I know it is wrong.}{Richard Buckminster Fuller}

\epigraph{He who asks a question is a fool for five minutes; he who does not ask
a question remains a fool forever.}{Chinese Proverb}

\epigraph{The great tragedy of science -- the slaying of a beautiful hypothesis
by an ugly fact.}{Thomas Huxley}

\epigraph{Give a man to fish and he will eat for a day. Teach a man to fish and
he will eat for the rest of his life.}{Chinese Proverb}

\epigraph{When you have eliminated the impossible, whatever remains, however
improbable, must be the truth.}{Conan Doyle}

\epigraph{Look here, upon this picture, and on this.}{Shakespeare, Hamlet}

\epigraph{If you choose to represent the various parts in life by holes upon a
table, of different shapes---some circular, some triangular, some
square, some oblong---we
shall generally find that the triangular person has got into the square
hole, the oblong into the
 triangular, and a square person has squeezed himself into the round
hole.}{Sydney Smith, 1769-1845}


\epigraph{I know of scarcely anything so apt to impress the imagination as the
wonderful form of cosmic order expressed by the ``Law of Frequency of
Error.'' The law would
have been personified by the Greeks and deified, if they had known of
it. It reigns with serenity and in complete self-effacement, amidst the
wildest confusion. The
huger the mob, and the greater the apparent anarchy, the more perfect is
its sway. It is the supreme law of Unreason. Whenever a large sample of
chaotic elements
are taken in hand and marshaled in the order of their magnitude, an
unsuspected and most beautiful form of regularity proves to have been
latent all along.}{Sir Francis Galton, \emph{Natural Inheritance}, London: Macmillan, 1889. Quoted in J. R. Newman (ed.) The World of Mathematics, New York: Simon and Schuster, 1956. p. 1482.}

\epigraph{In scientific thought we adopt the simplest theory which will explain
all the facts under consideration and enable us to predict new facts of
the same kind. The catch in
this criterion lies in the world "simplest." It is really an aesthetic
canon such as we find implicit in our criticisms of poetry or painting.
The layman finds such a law as
dx/dt = K(d^2x/dy^2) much less simple than "it oozes," of which it is
the mathematical statement. The physicist reverses this judgment, and
his statement is certainly
the more fruitful of the two, so far as prediction is concerned. It is,
however, a statement about something very unfamiliar to the plainman,
namely, the rate of change
of a rate of change.}{John Burdon Sanderson Haldane (1892-1964) Possible Worlds, 1927.}

\epigraph{We [he and Halmos] share a philosophy about linear algebra: we think
basis-free, we write basis-free, but when the chips are down we close
the office door and
compute with matrices like fury.}{Kaplansky, Irving \emph{Paul Halmos: Celebrating 50 Years of Mathematics}}

\epigraph{Oh, what a tangled web we weave,
When first we practice to deceive!}{Sir Walter Scott}

\epigraph{Practice is the best of all instructors.}{Publilius Syrus}

\epigraph{We should go to the masses and learn from them, synthesize their
experience into better, articulated principles and methods, then do
propaganda among the masses,
and call upon them to put these principles and methods into practice so
as to solve their problems and help them achieve liberation and
happiness.}{Chairman Mao Zedong, "Get Organized!" (November 29, 1943), Selected Works, Vol. III, p. 158.}

\epigraph{This paper contains much that is new and much that is true.
Unfortunately, that which is true is not new
and that which is new is not true.}{Wolfgang Pauli (attr.)}

\epigraph{The best thing about being a statistician is that you get to play in everyone's backyard.}{John W. Tukey}% from obituary, NY Times, July 28,00/Tukey

% generalizations
\epigraph{The museum spreads its surfaces everywhere, and becomes an untitled collection of generalizations that mobilize the eye.}{Robert Smithson}

\epigraph{In one word, to draw the rule from experience, one must generalize; this is a necessity that imposes itself on the most circumspect observer.}
{Henri Poincar�, The Value of Science: Essential Writings of Henri Poincare}
% tags: experience, generalizations, necessity, rule, science


Read more at http://www.brainyquote.com/quotes/keywords/generalizations.html#Gx8sZUeFjuotuIaQ.99
%% tables
\epigraph{Let it serve for table-talk.}
{William Shakespeare. The Merchant of Venice, Act III, Sc. 5.}

\epigraph{While memory holds a seat 
In this distracted globe. Remember thee! 
Yea, from the table of my memory 
I 'll wipe away all trivial fond records.}{William Shakespeare. Hamlet, Act I, Sc. 5.}

\epigraph{I drink to the general joy o' the whole table.}{William Shakespeare. Macbeth, Act III, Sc. 4.}

\epigraph{Isolated facts, those that can only be obtained by rough estimate
and that require development, can only be presented in memoires;
but those that can be presented in a body, with details, and on whose
accuracy one can rely, may be expounded in tables.}{E. Duvillard, \emph{M{\'e}moire sur le travail du Bureau de statistique}, 1806}
%cited in Derosi{\`e}res 1998 p.38

%% Averages

\epigraph{Winwood Reade is good upon the subject. He remarks that,
while the individual man is an insoluble puzzle, in the aggregate he
becomes a mathematical certainty. You can, for example, never foretell
what any one man will do, but you can say with precision what an average
number will be up to. Individuals vary, but percentages remain constant.
So says the statistician.}{Arthur Conan-Doyle, \emph{Sign of the Four}}


\epigraph{Exploratory data analysis can never be the whole story, but nothing else
can serve as the foundation stone -- as the first step.}{J. W. Tukey, 1977, p.3.} 

\epigraph{All models are wrong but some are useful.}{G. E. P. Box, 1979, p 201.}

History of this quote, from : http://en.wikiquote.org/wiki/Talk:George_E._P._Box
As documented in: Box, George E. P., J. American Statistical Assoc., Vol 74, Number 365, March 1979, "Some Problems of Statistics of Everyday Life" [1]

An early quote of this idea was: "Models, of course, are never true, but fortunately it is only necessary that they be useful."

The quote "ALL MODELS ARE WRONG BUT SOME ARE USEFUL" is the title of a section of the paper Box, G.E.P. (1979) "Robustness in the strategy of scientific model building" in Robustness in Statistics (R.L. Launer and G.N. Wilkinson, Eds.), Academic Press. [2] But since this is the proceedings of a meeting that took place in April 11-12 1978 [3] , this is likely to be the original quote.


\epigraph{Whenever you can, count.}{Sir Francis Galton, quoted in Newman}

\epigraph{Some people hate the very name of statistics but I find them full of 
beauty and interest.  Whenever they are not brutalised, but 
delicately handled by the higher methods, and are warily interpreted, 
their power of dealing with complicated phenomena is extraordinary.}{Galton, \emph{Natual Inheritance}, 1889 p62}

\epigraph{[Statistics are] the only tools by which an opening may be cut 
through the formidable thicket of difficulties that bars the path of 
those who pursue the Science of Man.}{Galton, quoted in K. Pearson, \emph{The Life, Letters and Labours of Francis Galton} (London, 1914)}

\epigraph{It is difficult to understand why statisticians commonly limit their 
inquiries to Averages, and do not revel in more comprehensive views. 
Their souls seem as dull to the charm of variety as that of the 
native of our flat English counties, whose retrospect of Switzerland 
was that, if its mountains could be thrown into its lakes, two 
nuisances would be got rid of at once.}{Galton in \emph{Natural Inheritance}}

\epigraph{Euclid alone has looked on beauty bare.}{Edna St Vincent Millay}

\epigraph{This book fills a much-needed gap.}{George Miller, from a book review}

\epigraph{If I have seen further, it is by standing on the shoulders of
giants.}{Sir Isaac Newton, in a letter to Robert Hooke, Feb. 5, 1676}

\epigraph{I am only a picture-taster, the way others are wine-or
tea-tasters.}{Bernard Berenson, \emph{Sunset and Twilight} Harcourt, Brace & World, 1963}

\epigraph{The only new thing in the world is the history you don't
know.}{Harry S. Truman, quoted by David McCulloch}

\epigraph{Mathematicians have always been rather of a jealous nature, and undoubtedly jealousy was a family characteristic of the Bernoullis.  There is some excuse for mathematicians, for their reputation stands for posterity largely not on what they did,
 but on what their contemporaries attributed to them.}{Karl Pearson, ``The History of Statistics in the 17th and 18th Centuries'' \cite[p. 226]{Pearson:1978}}

\epigraph{Statisticians, like artists, have the bad habit of falling in love with their models.}{George Box}


%% Georges' quotes

\epigraph{The best thing about being a statistician is that you get to play in
everyone's backyard.}{John W. Tukey}

\epigraph{The business of the statistician is to catalyze the scientific
learning process.}{George E. P. Box}

\epigraph{Humanists believe that the world has a fixed number of
mysteries, so that when one is solved, our sense of wonder is
diminished.  Scientists believe that the world has endless mysteries, so that when one is solved,
there are always new ones to ponder.}{D. O. Hebb, quoted by
Steven Pinker}

\epigraph{Far better an approximate answer to the right question,
which is often vague, than an exact answer to the wrong question,
which can always be made precise.}{John W. Tukey, (1962), ``The future of data analysis,'' \emph{Annals of Mathematical Statistics}, 33, 1-67.}

\epigraph{A bad answer to a good question may be far better than a good
answer to a bad question.}{A graduate class, paraphrasing Tukey's dictum.}

\epigraph{The worst, i.e., most dangerous, feature of 'accepting the null
hypothesis' is the giving up of explicit uncertainty ... Mathematics
can sometimes be put in such black-and-white terms, but our knowledge or belief about the external world never can.}{John W. Tukey. (1991). ``The Philosophy of
Multiple Comparisons,'' \emph{Statistical Science} 6, 100--116.}

\epigraph{All models are wrong but some are useful.}{George E. P. Box}

\epigraph{An elementary demonstration is one that requires no knowledge---
just an infinite amount of intelligence.}{Richard Feynman}

\epigraph{Science may be described as the art of systematic
over-simplification.}{Karl Popper}

\epigraph{Data analysis is an aid to thinking and not a replacement for.}{Richard Shillington}

\epigraph{Science is like sex: sometimes something useful comes out,
but that is not the reason we are doing it.}{Richard Feynman}

\epigraph{Few intellectual pleasures are more keen than those enjoyed by a person
who, while he is occupied in some special inquiry,
suddenly perceives that it admits of a wide generalization, and that his
results hold good in previously-unsuspected directions.
}{Francis Galton, \emph{North American Review}, \emph{150}, 419-431 (1890)}

\epigraph{The elegance of a theorem is directly proportional to the
number of ideas you can see in it and inversely proportional to the
effort it takes to see them.}{George Polya}

\epigraph{A mathematician, like a painter or a poet, is a master of pattern.
 The mathematician's patterns, like the painter's or the poet's, must be
beautiful; the ideas, like the colors or the words, must fit together in
a harmonious way. ...
There is no permanent place in the world for ugly mathematics.}{G. H. Hardy}

\epigraph{An idea which can be used once is a trick. If it can be used more than once it becomes a method.}{George Polya and Gabor Szego}

\epigraph{To err is human---but it feels divine!}{Mae West}

\epigraph{Better to have an approximate answer to the right question than a precise 
answer to the wrong question.}{John Tukey as quoted by John Chambers}

\epigraph{Good judgment comes from experience; experience comes from bad judgment.}{Fred Brooks}

\epigraph{[It is] best to confuse only one issue at a time.}{Kernihan \& Ritchie}

\epigraph{The past only exists insofar as it is present in the records of today.
And what those records are is determined by what questions we ask.  
There is no other history than that.}{Wheeler, 1982:24}
 
\epigraph{I never think of the future - it comes soon enough}{Albert Einstein} % (1879 - 1955)
	
\epigraph{The best way to predict the future is to invent it}{Alan Kay}

\epigraph{Prediction is very difficult, especially about the future}{Niels Bohr}

\epigraph{The future ain't what it used to be}{Yogi Berra}

\epigraph{A generation which ignores history has no past and no future}{Robert Heinlein} % (1907 - 1988), The Notebooks of Lazurus Long


\epigraph{Look not mournfully into the past. It comes not back again. 
Wisely improve the present. It is thine. 
Go forth to meet the shadowy future, without fear.}{Henry Wadsworth Longfellow} % (1807 - 1882)

\epigraph{Let him who would enjoy a good future waste none of his present.}{Roger Babson}

\epigraph{When in doubt, predict that the present trend will continue.}{Merkin's Maxim}

\epigraph{The only use of a knowledge of the past is to equip us for the present.
The present contains all that there is. It is holy ground; 
for it is the past, and it is the future.}{Alfred North Whitehead} % (1861 - 1947)

\epigraph{My past is my wisdom to use today. . . my future is my wisdom yet to 
experience. Be in the present because that is where life resides.}{Gene Oliver, Life and the Artistry of Change}

\epigraph{I have realized that the past and future are real illusions, 
that they exist in the present, which is what there is and all there is.}
    {Alan Watts}
	
\epigraph{History will be kind to me for I intend to write it.}{Winston Churchill}

\epigraph{For my part, I consider that it will be found much better by all parties to leave the past to history, 
especially as I propose to write that history myself.}{Winston Churchill}

\epigraph{If you would understand anything, observe its beginning and its development.}{Aristotle}

\epigraph{God alone knows the future, but only an historian can alter the past.}
{Ambrose Bierce}

\epigraph{Since God himself cannot change the past, he is obliged to tolerate the existence of historians.}{Attributed to Samuel Butler}

\epigraph{At the heart of good history is a naughty little secret: good storytelling.}{Stephen Schiff}

%Yesterday I had the opportunity to attend a seminar by George Box where
%he discussed some of the ideas that will be incorporated in the second
%edition of Box, Hunter, and Hunter "Statistics for Experimenters" due
%out in a few months.

\epigraph{Sometimes the only thing you can do with a poorly designed experiment
is to try to find out what it died of.}{R.A. Fisher}

\epigraph{If there were a probability of only p = 0.04 of finding a crock of
gold behind the next tree, wouldn't you go and look?}{George Box}

\epigraph{Seek computer programs that allow you to do the thinking.}{George Box}

 When the ratio of the largest to smallest observation is large you should question whether the data are being analyzed in the right
 metric (transformation). (George Box) 

\epigraph{A useful type of time series model is a recipe for transforming
serial data into white noise.}{George Box}

\epigraph{The best time to plan an experiment is after you've done it.}{R.A. Fisher}

\epigraph{Every model is an approximation.}{George Box}

\epigraph{All models are wrong, but some are useful}{George E. P. Box}
Reference: Box & Draper (1987), Empirical model-building and response surfaces, Wiley, p. 424.

 It is the data that are real (they actually happened!)

 The model is a hypothetical conjecture that might or might not summarize and/or explain important features of the data. (George Box)

\epigraph{It is a capital mistake to theorize before one has data.}{Sherlock Homes in \emph{Scandal in Bohemia}}

\epigraph{It is not unusual for a well-designed experiment to analyze itself.}{George Box}

\epigraph{Discovering the unexpected is more important than confirming the known.}{George Box} 

\epigraph{One must learn by doing the thing; for though you think you
know it, you have no certainty until you try.}{Sophocles}

\epigraph{The nice thing about standards is that there are so many of
them to choose from.}{Andrew Tanenbaum, \emph{Computer Networks}}

\epigraph{If you only know how to use a hammer, every problem starts to
look like a nail.  Stay away from that trap.}{Richard B. Johnson}

\epigraph{It has been said that though God cannot alter the past, historians can; it is perhaps because they can be useful to Him in this respect that He tolerates their existence.}{Samuel Butler, Erewhon Revisted}

\epigraph{The past is a foreign country: they do things differently there.}{L. P. Hartley, The Go-Between}

\epigraph{[The War Office kept three sets of figures:] one to mislead the public, another to mislead the Cabinet, and the third to mislead itself.}{Herbert Asquith in Alistair Horne, Price of Glory}

\epigraph{History is moving statistics and statistics is frozen history.}{A. L. Schl\"ozer, \emph{Theorie der Statistik}, 1804, p. 86}

\epigraph{When a law is contained in figures, it is buried like metal in an ore; it is necessary
to extract it.  This is the work of graphical representation. It points out the coincidences,
the relationships between phenomena, their anomalies, and we have seen what a powerful means of
control it puts in the hands of the statistician to verify new data, discover and correct errors
with which they have been stained.}{\'Emile Cheysson, \emph{Les m\'ethods de la statistique} (1890), 34-35.}


\epigraph{Why are you testing your data for normality?  For large sample sizes
the normality tests often give a meaningful answer to a meaningless
question (for small samples they give a meaningless answer to a
meaningful question)}{Greg Snow, R-Help, 21 Feb 2014}
	

%% Milestone quotes
 
\epigraph{Direction is more important than speed. We are so busy looking at our speedometers that we forget the milestone.}{Anonymous}

\epigraph{Only sixteen players have hit fifty or more homers in a season. To me, that's a very special milestone.}{Mark McGwire}

\epigraph{As life runs on, the road grows strange with faces new - and near the end. The milestones into headstones change, Neath every one a friend.}{James Russell Lowell}
