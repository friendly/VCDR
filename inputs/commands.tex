% General LaTeX commands for VCDR

%  Math commands
\newcommand{\bvec}[1]{\ensuremath{\mathbf{#1}}}
\renewcommand{\vec}[1]{\ensuremath{\bm{#1}}}
%\newcommand{\mat}[1]{\ensuremath{\mathbf{#1}}}
\newcommand{\mat}[1]{\ensuremath{\bm{#1}}}               % matrix (bold)
\newcommand{\trans}{\ensuremath{^\mathsf{T}}}            % transpose
\newcommand*{\degree}[1]{\ensuremath{{#1}^{\circ}}}
\newcommand{\diag}[1]{\ensuremath{\mathrm{diag}\, #1}}
\def\binom#1#2{{#1 \choose #2}}%
\newcommand*{\comma}{\:\: ,}%                      punct after displaymath
\newcommand*{\period}{\:\: .}
\newcommand*{\given}{\ensuremath{\, | \,}}
\newcommand*{\implies}{\ensuremath{\Longrightarrow}}

\newcommand*{\rank}[1]{\ensuremath{\mathrm{rank} (\mat{#1})}}
\newcommand*{\dev}[1]{(#1 - \bar{#1})}
\newcommand*{\inv}[1]{\ensuremath{\mat{#1}^{-1}}}
\newcommand*{\half}[1]{\ensuremath{\mat{#1}^{1/2}}}
\newcommand*{\nvec}[2]{\ensuremath{{#1}_{1}, {#1}_{2},\ldots,{#1}_{#2}}}
\newcommand*{\E}{\mathcal{E}}
\newcommand*{\V}{\mathcal{V}}

\newcommand{\blacksquare}{\rule{1.4ex}{1.4ex}}

% Coefficient with error underneath
\newcommand{\cwe}[2]{% 
  \mathord{\mathop{#1}\limits_{(#2)}}%
}
\newcommand{\sizedmat}[2]{%
  \mathord{\mathop{\mat{#1}}\limits_{(#2)}}%
}

%%%%%%%%%%%%%%%%%%%%%%%%%%%%%%%%%%%%%%%%%%%%%%%%%%%%%%
% mathematical functions
%%%%%%%%%%%%%%%%%%%%%%%%%%%%%%%%%%%%%%%%%%%%%%%%%%%%%%

\makeatletter
\def\logit{\mathop{\operator@font logit}}
\def\Bin{\mathop{\operator@font Bin}}
\def\Pois{\mathop{\operator@font Pois}}
\def\NBin{\mathop{\operator@font NBin}}
\def\Geom{\mathop{\operator@font Geom}}
\def\sign{\mathop{\operator@font sign}}

%\newcommand{\min}{\operatornamewithlimits{min}}
%\newcommand{\max}{\operatornamewithlimits{max}}
%\newcommand{\argmin}{\operatornamewithlimits{arg\,min}}
%\newcommand{\argmax}{\operatornamewithlimits{arg\,max}}
% the *ed form allows limits above/below, the non*ed form prints these beside the operator
%\DeclareMathOperator*{\argmin}{arg\,min}

%\newcommand{\Xvec}{X_1,X_2, \ldots, X_n }
\newcommand{\iid}{\stackrel{iid}{\sim}}
% should add an argument for n
\newcommand{\sumi}{\sum_{i=1}^n}

\def\ignorespacesafterend{\global\@ignoretrue}
\newenvironment{equation*}
	{\begin{displaymath}}%
%	{\end{displaymath}}%
	{\end{displaymath}\ignorespacesafterend}%
%
% Donald Arseneau recommends:
%\newenvironment{equation*}{\displaymath}{\enddisplaymath}%

%%%%%%%%%%%%%%%%%%%%%%%%%%%%%%%%%%%%%%%%%%%%%%%%%%%%%%
%% common abbreviations
%%%%%%%%%%%%%%%%%%%%%%%%%%%%%%%%%%%%%%%%%%%%%%%%%%%%%%

\newcommand*{\hires}{high-resolution}
\newcommand*{\etal}{\emph{et al.}}
\newcommand*{\loglin}{loglinear\xspace}
\newcommand*{\Loglin}{Loglinear\xspace}
\newcommand*{\ctab}{contingency table\xspace}
\newcommand*{\ctabs}{contingency tables\xspace}
\newcommand*{\mway}{multiway\xspace}
\newcommand*{\LR}{likelihood-ratio\xspace}
\newcommand*{\CA}{Correspondence analysis\xspace}
\newcommand*{\ca}{correspondence analysis\xspace}
\newcommand*{\nway}{\emph{n}-way\xspace}
\newcommand*{\GSQ}{\ensuremath{G^2}\xspace}
\newcommand*{\chisq}{\ensuremath{\chi^2}\xspace}
\newcommand*{\scat}{scatterplot\xspace}
\newcommand*{\scats}{scatterplots\xspace}
\newcommand*{\scatmat}{\scat{} matrix\xspace}
\newcommand*{\df}{degrees of freedom\xspace}
\newcommand*{\Dset}{data set\xspace}
\newcommand*{\Dsets}{data set\xspace}

\newcommand*{\llmterm}[1]{\ensuremath{[}#1\ensuremath{]}}
\newcommand*{\llmtwo}[2]{\llmterm{#1} \llmterm{#2}}
\newcommand*{\llmthree}[3]{\llmterm{#1} \llmterm{#2} \llmterm{#3}}
\newcommand*{\llmfour}[4]{\llmterm{#1} \llmterm{#2} \llmterm{#3} \llmterm{#4}}

%% \LLM{A,B,C} --> [A] [B] [C] for loglin models
\DeclareRobustCommand*{\LLM}[1]{%
%\def\LLM#1{%
	\@for\@term:=#1\do{%
	\llmterm{\@term}%
	}
}
\makeatother

% deprecated, but maybe used somewhere
\newcommand*{\boldital}[1]{\textit{\textbf{#1}}}

%%%%%%%%%%%%%%%%%%%%%%%%%%%%%%%%%%%%%%%%%%%%%%%%%%%%%%%%%%%%%%%%%%
% precept -- something to stand out in the text
%   could use a box or something else
%%%%%%%%%%%%%%%%%%%%%%%%%%%%%%%%%%%%%%%%%%%%%%%%%%%%%%%%%%%%%%%%%%

\newcommand{\precept}[1]{%
\begin{quote}
\centering
\textbf{#1}
\end{quote}
}


%%%%%%%%%%%%%%%%%%%%%%%%%%%%%%%%%%%%%%%%%%%%%%%%%%%%%%%%%%%%%%%%%%
% \glossterm -- used for terms that should be highlighted in the
% text and index, and which might go into a glossary (but only 
% if glosstex is run)
% The original definition did not allow for such terms at the beginning
% of a sentence.
%\newcommand{\glossterm}[1]{\textit{\textbf{#1}}\glosstex{#1}}

% Simple variant, just for formatting; can also use \marginpar{}
% and glossterm
\newcommand{\term}[1]{\textit{\textbf{#1}}\index{#1}}

%\glossterm[print-form]{gloss-form}
\makeatletter
\def\glossterm{\@dblarg\@glossterm}
\def\@glossterm[#1]#2{\textit{\textbf{#1}}\glosstex{#2}\index{#2|textbf}}
\makeatother

% Author's notes -- to disappear in production
\newcommand{\aunote}[1]{\marginpar{\footnotesize\textbf{Au:} #1}}

% Dummy command for changes
%\newenvironment{changebar}{}{}%
%\newcommand{\changebar}[1]{#1}

%%%%%%%%%%%%%%%%%%%%%%%%%%%%%%%%%%%%%%%%%%%%%%%%%%%%%%%%%%%%%%%%%%%%%%
% Commands to simplify cross-references
%%%%%%%%%%%%%%%%%%%%%%%%%%%%%%%%%%%%%%%%%%%%%%%%%%%%%%%%%%%%%%%%%%%%%%

\newcommand*{\eqref}[1]{Eqn.~(\ref{#1})}
\newcommand*{\exref}[1]{Example~\ref{#1}}
\newcommand*{\chref}[1]{Chapter~\ref{#1}}
\newcommand*{\secref}[1]{Section~\ref{#1}}
\newcommand*{\figref}[1]{Figure~\ref{#1}}
\newcommand*{\tabref}[1]{Table~\ref{#1}}
\newcommand*{\outref}[1]{Output~\ref{#1}}
\newcommand*{\datref}[1]{Appendix~\ref{#1}}
%\newcommand*{\macref}[1]{Appendix~\ref{#1}}
\newcommand*{\appref}[1]{Appendix~\ref{#1}}

% Reference a range of refs
\newcommand*{\chrange}[2]{Chapters~\ref{#1}--\ref{#2}}
\newcommand*{\figrange}[2]{Figures~\ref{#1}--\ref{#2}}
\newcommand*{\tabrange}[2]{Tables~\ref{#1}--\ref{#2}}
%
% Reference a list of figs, examples, etc., not necessarily sequential
\newcommand{\figrefs}[1]{\dorefs{#1}{Figures}}
\newcommand{\tabrefs}[1]{\dorefs{#1}{Tables}}
\newcommand{\exrefs}[1]{\dorefs{#1}{Examples}}
\makeatletter
\newcommand{\dorefs}[2]{%
  \let\@dummy\@empty
  #2~%
  \@for\@term:=#1\do{%
    \@dummy
    \edef\@dummy{\ref{\@term}, }}%
  \expandafter\format@last\@dummy}
\def\format@last#1, {and #1}
\makeatother


%%%%%%%%%%%%%%%%%%%%%%%%%%%%%%%%%%%%%%%%%%%%%%%%%%%%%%%%%%%%%%%%%%%%%%%%%%%
% multiline headers in tables 
% use as: Variable & DF & \multilineC{Parameter \\ Estimate} & ...
%%%%%%%%%%%%%%%%%%%%%%%%%%%%%%%%%%%%%%%%%%%%%%%%%%%%%%%%%%%%%%%%%%%%%%%%%%%

\newcommand{\multilineR}[1]{\begin{tabular}[b]{@{}r@{}}#1\end{tabular}} 
\newcommand{\multilineL}[1]{\begin{tabular}[b]{@{}l@{}}#1\end{tabular}} 
\newcommand{\multilineC}[1]{\begin{tabular}[b]{@{}c@{}}#1\end{tabular}} 

%% table stuff, another way
% to make it easier to use & \brk{this\\or\\that} & in \tabular

\newcommand{\brk}[2][l]{%
   \begin{tabular}{@{}#1@{}}#2%
   \end{tabular}%
}

%%%%%%%%%%%%%%%%%%%%%%%%%%%%%%%%%%%%%%%%%%%%%%%%%%%%%%%%%%%%%%%%%%%%%%%%%%%%
% colored tables
%%%%%%%%%%%%%%%%%%%%%%%%%%%%%%%%%%%%%%%%%%%%%%%%%%%%%%%%%%%%%%%%%%%%%%%%%%%%
% requires:
%\usepackage{xcolor,colortbl}  %% used ub Ch 01

%\newcommand{\tableheader}{\rowcolor[gray]{.85}}
\newcommand{\tableheader}{\rowcolor[HTML]{FFFFC7}} % light yellow background

\newcommand{\cell}[2]{\multicolumn{1}%
   {>{\columncolor{#1}}r}{#2}}

\newcommand{\C}{Chapter\xspace}

\newcommand{\chapterprelude}[1]{%
\textsf{#1}
\newline
\rule{\textwidth}{0.4pt}
}


%\renewcommand{\S}{Section }

%%%%%%%%%%%%%%%%%%%%%%%%%%%%%%%%%%%%%%%%%%%%%%%%%%%%%%%%%%%%%%%
% R terminology

% writing about R stuff; these can be modified to add indexing, etc.
\newcommand{\var}[1]{\textit{\texttt{#1}}}

% Data sets -- print and index
%\newcommand{\data}[1]{\texttt{#1}}
\newcommand*{\data}[1]{\texttt{#1}\ixd{#1}}

\newcommand{\class}[1]{\textsf{"#1"}}

% may need a more robust version of \code to handle special chars
% this doesn't quite handle it.
\makeatletter
\newcommand\code{\bgroup\@makeother\_\@makeother\~\@makeother\$\@codex}
\def\@codex#1{{\normalfont\ttfamily\hyphenchar\font=-1 #1}\egroup}
\makeatother
%\newcommand{\code}[1]{\texttt{#1}}

% R functions: use \code{} and also \index{}
\newcommand{\func}[1]{\code{#1()}\ixfunc{#1()}}

\let\proglang=\textsf
\newcommand{\R}{\proglang{R}\xspace}

% should redefine \pkg to also cite the package, but this requires
% an extra, optional argument, unless it is assured that the package
% name is the bibtex key; also: add indexing
%\newcommand{\pkg}[1]{{\normalfont\fontseries{b}\selectfont #1}}
\newcommand{\pkg}[1]{\textsf{#1}\ixp{#1}}
\newcommand{\Rpackage}[1]{\pkg{#1} package}

\newcommand{\help}[1]{\code{help(#1)}}     % reference R help

\newcommand*{\VCDR}{\emph{VCDR} }
\newcommand*{\argument}[1]{\texttt{#1} argument}
%\newcommand*{\sasprog}[1]{\texttt{#1} program\ixp{#1}}
%\newcommand*{\default}[1]{\texttt{[}Default: \url{#1}\texttt{]}}

%%%%%%%%%%%%%%%%%%%%%%%%%%%%%%%%%%%%%%%%%%%%%%%%%%%%%%%%%%%%%%%%%%%%%%%
% Index generation
% Indexentry for a word/phrase (Word inserted into the text)
%%%%%%%%%%%%%%%%%%%%%%%%%%%%%%%%%%%%%%%%%%%%%%%%%%%%%%%%%%%%%%%%%%%%%%%
\newcommand{\IX}[1]{\index{#1}#1}
\newcommand{\ix}[1]{\index{#1}}
\newcommand{\ixmain}[1]{\index{#1|textbf}}

%\newcommand{\ixm}[1]{%
%   \index{#1@\texttt{#1} macro}%
%   \index{macros!#1@\texttt{#1}}%
%	}

% R functions
\newcommand{\ixfunc}[1]{%
  \index{#1@\texttt{#1}}%
%  \index{functions!#1@\texttt{#1}}%
 }

% R packages:  indexed under both package name and packages!
\newcommand{\ixp}[1]{%
   \index{#1@\textsf{#1} package}%
   \index{package!#1@\textsf{#1}}%
	}

% data sets: 
\newcommand{\ixd}[1]{%
        \index{data sets!#1}}

% Examples Index
\newcommand{\ixe}[1]{%
        \index[xmp]{#1}}

\newcommand{\ixon}[1]{\index{#1|(}}
\newcommand{\ixoff}[1]{\index{#1|)}}


%\newcommand*\seealso[2]{\emph{\alsoname} #1}
% and then:
%\index{foo|seealso{bar}}
% If \alsoname isn't defined, you would have to add:
%\newcommand{\alsoname}{see also}

% This puts the argument in italics in the text, in boldface in the
% index, and if you give an optional argument, that goes in the index,
% so you can write:

%\define{gnat}
%\define[animals|gnats]{gnat}

\makeatletter
\newcommand{\define}{\@ifnextchar[\@dfna\@dfnb}
\def\@dfna[#1]#2{\textit{#2}\index{#1|textbf}}
\def\@dfnb#1{\@dfna[#1]{#1}}
\makeatother

%%%%%%%%%%%%%%%%%%%%%%%%%%%%%%%%%%%%%%%%%%%%%%%%%%%%%%%%%%%%%%%%%%%%%%%%%%%
% Some convenience macros for figures --- not used here
% because knitr seems to do things reasonably well without them.

% Define the current fig directory
\newcommand{\figdir}{ch\thechapter/fig/}
% Redefine the current fig directory
\newcommand{\newfigdir}[1]{\renewcommand{\figdir}{#1/fig/}}

% Command to collect graphics file info - ignored for now, but used
% whereever I abbreviate the graphics file 
% from {chX/fig/figure.eps} to {figure}
\newcommand{\graphicsfile}[2]{\relax}

%% \SASfig{file}{include_opts}{label}{caption}
%  This command is no longer used -- all figures use \includegraphics directly

%\newcommand{\SASfig}[4]{%
%  \centering%
%  \includegraphics[#2]{#1}\graphicsfile{\figdir#1}{}%
%  \caption{#4}\label{fig:#3}%
%  }

%% \fig{file}{include_opts}{shortcaption}[extended caption]
%  label is fig:file
\makeatletter
  \newcommand{\fig}[3]{\@ifnextchar[%]
    {\@extr@fig{#1}{#2}{#3}}%
    {\@norm@fig{#1}{#2}{#3}}%
  }
  \def\@extr@fig#1#2#3[#4]{%
    \begin{figure}[htb]%
    \centering%
    \includegraphics[#2]{\figdir#1}%
    \caption[#3]{#3. #4}\label{fig:#1}%
    \end{figure}%
    }
  \newcommand{\@norm@fig}[3]{%
    \begin{figure}[htb]%
    \centering%
    \includegraphics[#2]{\figdir#1}%
    \caption{#3}\label{fig:#1}%
    \end{figure}%
    }


%%%%%%%%%%%%%%%%%%%%%%%%%%%%%%%%%%%%%%%%%%%%%%%%%%%%%%%%%%%%%%%%%%%%%%%%%%
% Specialized kinds of lists
%%%%%%%%%%%%%%%%%%%%%%%%%%%%%%%%%%%%%%%%%%%%%%%%%%%%%%%%%%%%%%%%%%%%%%%%%%


% APA Seriations: ONE level of seriation only.
%  \begin{seriate} \item ... \end{seriate}
%           within a paragraph or sentence

\newcounter{APAenum}
\def\seriate{\@bsphack\begingroup%
   \setcounter{APAenum}{0}%
   \def\item{\addtocounter{APAenum}{1}(\alph{APAenum})\space}%
   \ignorespaces}
\def\endseriate{\endgroup\@esphack}

\makeatother

% definition lists for programs or arguments, with suitable indenting

\newenvironment{proglist}%
 {\begin{list}{}{%
    \settowidth{\labelwidth}{\texttt{PROGRAMSxx}}
         \setlength{\leftmargin}{\labelwidth}
         \addtolength{\leftmargin}{\labelsep}
         \setlength{\parsep}{0.2ex plus0.2ex minus0.2ex}
         \setlength{\itemsep}{0pt}
         \renewcommand{\makelabel}[1]{\texttt{##1\hfill}}}}
 {\end{list}}


%%%%%%%%%%%%%%%%%%%%%%%%%%%%%%%%%%%%%%%%%%%%%%%%%%%%%%%%%%%%%%%%%%%%%%%
% Numbered examples that can be referenced
%%%%%%%%%%%%%%%%%%%%%%%%%%%%%%%%%%%%%%%%%%%%%%%%%%%%%%%%%%%%%%%%%%%%%%%
%
\newcounter{example}[chapter]
\renewcommand{\theexample}{\thechapter.\arabic{example}}
\newenvironment{Example}[2][\theexample]{%
  \refstepcounter{example}%
  \label{ex:#1}%
  \def\theexamplename{#2}%
  \begin{trivlist}%
  \item[%
  % \hskip-\labelsep % idiosyncrasy that needs learning
    \textbf{\textsc{Example \theexample}:}] %
	\textbf{#2}\par
  \ixe{#2|(}%
  }{%
	\expandafter\ixe\expandafter{\theexamplename|)}%   magic from Bernd
  \hfill$\triangle$
%  The triangle used to mark the end of examples can be replaced by any
%  other character, ... e.g.,
%  \hfill\blacksquare
%	\ding{110}% filled black square (using pifont package)
  \end{trivlist}%
}

%%%%%%%%%%%%%%%%%%%%%%%%%%%%%%%%%%%%%%%%%%%%%%%%%%%%%%%%%%%%%%%%%%
% Define new list type for exercises
% from: http://tex.stackexchange.com/questions/196199/exercise-list-using-enumitem-how-control-indentation-and-labeling-of-sublists
% by: Daniel Wunderlich
%%%%%%%%%%%%%%%%%%%%%%%%%%%%%%%%%%%%%%%%%%%%%%%%%%%%%%%%%%%%%%%%%%
%
\usepackage{enumitem}      % this should be loaded in book.Rnw
%
\newlist{Exercises}{enumerate}{2}
% set list style parameters
\setlist[Exercises]{%
  label=\textbf{Exercise \thechapter.\arabic*}~,  % Label: Exercise Chapter.exercise
  ref=\thechapter.\arabic*, % References: Chapter.exercise (important!)
  align=left,               % Left align labels
  labelindent=0pt,          % No space betw. margin of list and label
  leftmargin=0pt,           % No space betw. margin of list and following lines
  itemindent=!,             % Indention of item computed automatically
  itemsep=3pt,
}

\newcommand{\exercise}{%
  \item \label{lab:\arabic{chapter}.\arabic{Exercisesi}}      % Append label to item
  \setlist[enumerate, 1]{label=(\alph*),itemsep=0pt}          % Label for subexercises, but only within an exercise
}

% references to exercises
\newcommand{\labref}[1]{Exercise~\ref{#1}}

%%%%%%%%%%%%%%%%%%%%%%%%%%%%%%%%%%%%%%%%%%%%%%%%%%%%%%%%%%%%%%%%%%%%
%  Author notes, etc

\newcommand{\TODO}[1]{\noindent{\color{red}\textbf{TODO}: #1}}
\newcommand{\DONE}[1]{\noindent{\color{blue}\textbf{Done}: #1}}
% convert these to ignore the supplied text when no longer needed
%\newcommand{\TODO}[1]{\relax}
%\newcommand{\DONE}[1]{\relax}



%% Latex notes, p 73
\newlength{\boxedparwidth}
\setlength{\boxedparwidth}{.92\textwidth}
\newenvironment{boxedtext}%
        {\begin{center}%
         \begin{tabular}{|@{\hspace{.15in}}c@{\hspace{.15in}}|}
         \hline \\ begin{minipage}[t]{\boxedparwidth}
         }
         {\end{minipage} \\ \\ \hline \end{tabular} \end{center}}




%%%%%%%%%%%%%%%%%%%%%%%%%%%%%%%%%%%%%%%%%%%%%%%%%%%%%%%%%%%%%%%%%%%%%%
% Symbols for hard or difficult sections and problems
%  - for lack of something better, just use the 'dangerous bend' symbol
%    from the TeX book

% \usepackage{manfnt}
% \newcommand{\hard}{\marginpar{\dbend}}
% \newcommand{\veryhard}{\marginpar{\dbend \dbend}}
\newcommand{\hard}{$\star$\xspace}
\newcommand{\veryhard}{$\star\star$\xspace}



\endinput
