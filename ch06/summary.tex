%\section{Chapter summary}
\begin{itemize}
\item Correspondence analysis is an exploratory technique, designed to
show the row and column categories in a two- (or three-) dimensional
space.  These graphical displays, and various extensions, provide
ways to interpret the patterns of association and explore visually
the adequacy of certain \loglin models.

\item The scores assigned to the categories of each variable are optimal
in several equivalent ways.
Among other properties,
they maximize the (canonical) correlations between the quantified
variables (weighted by cell frequencies), and make the regressions
of each variable on the other most nearly linear, for each CA dimension.

\item Multi-way tables may be analyzed in several ways.
In the ``stacking'' approach, two or more variables may be combined
interactively in the rows and/or columns of an \nway table.
Simple CA of the restructured table reveals associations between
the row and column categories of the restructured table,
but hides associations between the variables combined interactively.
Each way of stacking corresponds to a particular \loglin model
for the full table.

\item Multiple \ca is a generalization of CA to two or more variables
based on representing the data as an indicator matrix, or the Burt matrix.
The usual MCA provides an analysis of the joint, bivariate relations
between all pairs of variables.

% \item An extended form of MCA provides a means to display higher-order
% associations among multiple categorical variables.
% For $2^Q$ tables composed of $Q$ binary variables, this analysis yields
% simple geometric relations that may be interpreted in terms of odds ratios.
% \TODO{Delete this if \secref{sec:ca-mcainter} is not included.}

\item The biplot is a related technique for visualizing the elements of
a data array by points or vectors in a joint display of their row and
column categories. A standard CA biplot represents the contributions to
lack of independence as the projection of the points for rows
(or columns) on vectors for the other categories.
Another application of the biplot to \ctab data is described, based on analysis
of log frequency.
This analysis also serves to diagnose patterns of independence and
partial independence in two-way and larger tables.
\end{itemize}
