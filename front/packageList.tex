\begin{description}

  \item [AER] \emph{Applied Econometrics with R} (v. 1.2-3).
	Functions, data sets, examples, demos, and vignettes for the book
             Christian Kleiber and Achim Zeileis (2008),
	     Applied Econometrics with R, Springer-Verlag, New York.
	     ISBN 978-0-387-77316-2. (See the vignette for a package overview.)
	\textbf{Authors}: Christian Kleiber, Achim Zeileis


  \item [agridat] \emph{Agricultural Datasets} (v. 1.11).
	Datasets from books, papers, and websites related to agriculture.
    Example analyses are included.  Includes functions for plotting field
    designs and GGE biplots.
	\textbf{Authors}: Kevin Wright


  \item [animation] \emph{A gallery of animations in statistics and utilities to create
animations} (v. 2.3).
	This package contains a variety of functions for animations in
    statistics, covering areas such as probability theory, mathematical
    statistics, multivariate statistics, nonparametric statistics, sampling
    survey, linear models, time series, computational statistics, data mining
    and machine learning. These functions might be helpful in teaching
    statistics and data analysis. Also provided in this package are several
    approaches to save animations to various formats, e.g. Flash, GIF, HTML
    pages, PDF and videos (saveSWF(), saveGIF(), saveHTML(), saveLatex(), and
    saveVideo() respectively). PDF animations can be inserted into Sweave/knitr
    easily.
	\textbf{Authors}: Lijia Yu, Weicheng Zhu, Yihui Xie


  \item [ca] \emph{Simple, Multiple and Joint Correspondence Analysis} (v. 0.58).
	A package for computation and visualization of simple, multiple and joint correspondence analysis.
	\textbf{Authors}: Michael Greenacre, Oleg Nenadic, Michael Friendly


  \item [car] \emph{Companion to Applied Regression} (v. 2.0-25).
	Functions and Datasets to Accompany J. Fox and S. Weisberg, 
  An R Companion to Applied Regression, Second Edition, Sage, 2011.
	\textbf{Authors}: John Fox, Sanford Weisberg, Daniel Adler, Douglas Bates, Gabriel Baud-Bovy, Steve Ellison, David Firth, Michael Friendly, Gregor Gorjanc, Spencer Graves, Richard Heiberger, Rafael Laboissiere, Georges Monette, Duncan Murdoch, Henric Nilsson, Derek Ogle, Brian Ripley, William Venables, Achim Zeileis, R-Core


  \item [colorspace] \emph{Color Space Manipulation} (v. 1.2-6).
	Carries out mapping between assorted color spaces including
             RGB, HSV, HLS, CIEXYZ, CIELUV, HCL (polar CIELUV),
	     CIELAB and polar CIELAB. Qualitative, sequential, and
	     diverging color palettes based on HCL colors are provided.
	\textbf{Authors}: Ross Ihaka, Paul Murrell, Kurt Hornik, Jason C. Fisher, Achim Zeileis


  \item [corrplot] \emph{Visualization of a correlation matrix} (v. 0.73).
	The corrplot package is a graphical display of a
        correlation matrix, confidence interval. It also contains some
        algorithms to do matrix reordering.
	\textbf{Authors}: Taiyun Wei


  \item [countreg] \emph{Count Data Regression} (v. 0.1-2).
	Regression models for count data, including zero-inflated,
             zero-truncated, and hurdle models as well as generalized count
             data regression. Drivers for combination with flexmix and mboost are
	     also provided. Furthermore, rootograms for assessing goodness of fit
	     for (regression) models are available.
	\textbf{Authors}: Achim Zeileis, Christian Kleiber


  \item [directlabels] \emph{Direct labels for multicolor plots in lattice or ggplot2} (v. 2013.6.15).
	An extensible framework
 for automatically placing direct labels onto multicolor lattice or
 ggplot2 plots.
 Label positions are described using Positioning Methods
 which can be re-used across several different plots.
 There are heuristics for examining "trellis" and "ggplot" objects
 and inferring an appropriate Positioning Method. 
	\textbf{Authors}: Toby Dylan Hocking


  \item [effects] \emph{Effect Displays for Linear, Generalized Linear, and Other Models} (v. 3.0-4).
	Graphical and tabular effect displays, e.g., of interactions, for 
  various statistical models with linear predictors.
	\textbf{Authors}: John Fox, Sanford Weisberg, Michael Friendly, Jangman Hong, Robert Andersen, David Firth, Steve Taylor


  \item [extracat] \emph{Categorical Data Analysis and Visualization} (v. 1.7-1).
	Categorical Data Analysis and Visualization
	\textbf{Authors}: Alexander Pilhoefer \url{email:alexander.pilhoefer@math.uni-augsburg.de}


  \item [FactoMineR] \emph{Multivariate Exploratory Data Analysis and Data Mining} (v. 1.29).
	Exploratory data analysis methods such as principal component methods and clustering
	\textbf{Authors}: Francois Husson, Julie Josse, Sebastien Le, Jeremy Mazet


  \item [foreign] \emph{Read Data Stored by Minitab, S, SAS, SPSS, Stata, Systat, Weka,
dBase, ...} (v. 0.8-63).
	Functions for reading and writing data stored by some versions of
	Epi Info, Minitab, S, SAS, SPSS, Stata, Systat and Weka
	and for reading and writing some dBase files.
	\textbf{Authors}: R Core Team, Roger Bivand, Vincent J. Carey, Saikat DebRoy, Stephen Eglen, Rajarshi Guha, Nicholas Lewin-Koh, Mark Myatt, Ben Pfaff, Brian Quistorff, Frank Warmerdam, Stephen Weigand, Free Software Foundation, Inc.


  \item [gdata] \emph{Various R programming tools for data manipulation} (v. 2.13.3).
	Various R programming tools for data manipulation
	\textbf{Authors}: Gregory R. Warnes, Ben Bolker, Gregor Gorjanc, Gabor Grothendieck, Ales Korosec, Thomas Lumley, Don MacQueen, Arni Magnusson, Jim Rogers, and others


  \item [GGally] \emph{Extension to ggplot2.} (v. 0.5.0).
	GGally is designed to be a helper to ggplot2. It contains
    templates for different plots to be combined into a plot matrix, a parallel
    coordinate plot function, as well as a function for making a network plot.
	\textbf{Authors}: Barret Schloerke \url{email:schloerke@gmail.com}, Jason Crowley \url{email:crowley.jason.s@gmail.com}, Di Cook \url{email:dicook@iastate.edu}, Heike Hofmann \url{email:hofmann@iastate.edu}, Hadley Wickham \url{email:h.wickham@iastate.edu}, and Francois Briatte \url{email:f.briatte@gmail.com}, Moritz Marbach \url{email:mmarbach@mail.uni-mannheim.de}, and Edwin Thoen \url{email:edwinthoen@@gmail.com}


  \item [ggplot2] \emph{An Implementation of the Grammar of Graphics} (v. 1.0.1).
	An implementation of the grammar of graphics
    in R. It combines the advantages of both base and
    lattice graphics: conditioning and shared axes are
    handled automatically, and you can still build up a
    plot step by step from multiple data sources. It also
    implements a sophisticated multidimensional
    conditioning system and a consistent interface to map
    data to aesthetic attributes. See http://ggplot2.org
    for more information, documentation and examples.
	\textbf{Authors}: Hadley Wickham, Winston Chang


  \item [ggtern] \emph{An extension to ggplot2, for the creation of ternary diagrams.} (v. 1.0.3.2).
	ggtern is a package that extends the functionality of ggplot2, giving the capability to plot ternary diagrams for (subset of) the ggplot2 geometries. Additionally, ggtern has implemented several NEW geometries which are unavailable to the standard ggplot2 release. For further examples and documentation, please proceed to the ggtern website. 
	\textbf{Authors}: Nicholas Hamilton \url{email:nick@ggtern.com}


  \item [ggvis] \emph{Interactive Grammar of Graphics} (v. 0.4.1).
	An implementation of an interactive grammar of graphics, taking the
    best parts of ggplot2, combining them with shiny's reactive framework and
    drawing web graphics using vega.
	\textbf{Authors}: Winston Chang, Hadley Wickham, RStudio, jQuery Foundation (jQuery library and jQuery UI library), jQuery contributors (jQuery library; authors listed in inst/www/lib/jquery/AUTHORS.txt), jQuery UI contributors (jQuery UI library; authors listed in inst/www/lib/jquery-ui/AUTHORS.txt), Mike Bostock (D3 library), D3 contributors (D3 library; authors listed at https://github.com/mbostock/d3/graphs/contributors), Trifacta Inc. (Vega library), Vega contributors (Vega library; authors listed at https://github.com/trifacta/vega/graphs/contributors)


  \item [gnm] \emph{Generalized Nonlinear Models} (v. 1.0-7).
	Functions to specify and fit generalized nonlinear models,
        including models with multiplicative interaction terms such as
        the UNIDIFF model from sociology and the AMMI model from crop
        science, and many others.  Over-parameterized representations
        of models are used throughout; functions are provided for
        inference on estimable parameter combinations, as well as
        standard methods for diagnostics etc.
	\textbf{Authors}: Heather Turner and David Firth


  \item [googleVis] \emph{R Interface to Google Charts} (v. 0.5.8).
	R interface to Google Charts API, allowing users
    to create interactive charts based on data frames. Charts 
    are displayed locally via the R HTTP help server. A modern
    browser with Internet connection is required and for some
    charts a Flash player. The data remains local and is not
    uploaded to Google.
	\textbf{Authors}: Markus Gesmann, Diego de Castillo, Joe Cheng


  \item [gpairs] \emph{gpairs: The Generalized Pairs Plot} (v. 1.2).
	Produces a generalized pairs (gpairs) plot.
	\textbf{Authors}: John W. Emerson and Walton A. Green


  \item [heplots] \emph{Visualizing Hypothesis Tests in Multivariate Linear Models} (v. 1.0-12).
	Provides HE plot functions for visualizing hypothesis tests in multivariate linear models. They represents sums-of-squares-and-products matrices for linear hypotheses and for error using ellipses (in two dimensions) and ellipsoids (in three dimensions).
	\textbf{Authors}: John Fox, Michael Friendly, Georges Monette


  \item [HistData] \emph{Data sets from the history of statistics and data visualization} (v. 0.7-5).
	The HistData package provides a collection of small data sets
    that are interesting and important in the history of statistics and data visualization.
    The goal of the package is to make these available, both for instructional use
    and for historical research.  Some of these present interesting challenges for graphics
    or analysis in R.
	\textbf{Authors}: Michael Friendly, Stephane Dray, Hadley Wickham, James Hanley, Dennis Murphy


  \item [hmmm] \emph{hierarchical multinomial marginal models} (v. 1.0-3).
	Functions for specifying and fitting marginal models for contingency tables proposed 
	by Bergsma and Rudas (2002) here called hierarchical multinomial marginal models (hmmm) and their extensions presented by Bartolucci et al. 
	(2007); multinomial Poisson homogeneous (mph) models and homogeneous linear predictor (hlp) models for contingency
 	tables proposed by Lang (2004) and (2005); hidden Markov models where the distribution of the observed variables 
	is described by a marginal model. 
	Inequality constraints on the parameters are allowed and can be tested.
	\textbf{Authors}: Colombi Roberto and Sabrina Giordano and Manuela Cazzaro, with contributions from Joseph Lang


  \item [iplots] \emph{iPlots - interactive graphics for R} (v. 1.1-7).
	Interactive plots for R
	\textbf{Authors}: Simon Urbanek \url{email:simon.urbanek@r-project.org}, Tobias Wichtrey \url{email:tobias@tarphos.de}


  \item [KernSmooth] \emph{Functions for Kernel Smoothing Supporting Wand \& Jones (1995)} (v. 2.23-14).
	Functions for kernel smoothing (and density estimation)
  corresponding to the book: 
  Wand, M.P. and Jones, M.C. (1995) "Kernel Smoothing".
	\textbf{Authors}: Matt Wand, Brian Ripley (R port and updates)


  \item [knitr] \emph{A General-Purpose Package for Dynamic Report Generation in R} (v. 1.9).
	This package provides a general-purpose tool for dynamic report
    generation in R, which can be used to deal with any type of (plain text)
    files, including Sweave, HTML, Markdown, reStructuredText, AsciiDoc, and
    Textile. R code is evaluated as if it were copied and pasted in an R
    terminal thanks to the evaluate package (e.g., we do not need to explicitly
    print() plots from ggplot2 or lattice). R code can be reformatted by the
    formatR package so that long lines are automatically wrapped, with indent
    and spaces added, and comments preserved. A simple caching mechanism is
    provided to cache results from computations for the first time and the
    computations will be skipped the next time. Almost all common graphics
    devices, including those in base R and add-on packages like Cairo,
    cairoDevice and tikzDevice, are built-in with this package and it is
    straightforward to switch between devices without writing any special
    functions. The width and height as well as alignment of plots in the output
    document can be specified in chunk options (the size of plots for graphics
    devices is also supported). Multiple plots can be recorded in a single code
    chunk, and it is also allowed to rearrange plots to the end of a chunk or
    just keep the last plot. Warnings, messages and errors are written in the
    output document by default (can be turned off). The large collection of
    hooks in this package makes it possible for the user to control almost
    everything in the R code input and output. Hooks can be used either to
    format the output or to run R code fragments before or after a code chunk.
    The language in code chunks is not restricted to R (there is simple support
    to Python and shell scripts, etc). Many features are borrowed from or
    inspired by Sweave, cacheSweave, pgfSweave, brew and decumar.
	\textbf{Authors}: Adam Vogt, Alastair Andrew, Alex Zvoleff, Andre Simon (the CSS files under inst/themes/ were derived from the Highlight package http://www.andre-simon.de), Ashley Manton, Brian Diggs, Cassio Pereira, David Robinson, Donald Arseneau (the framed package at inst/misc/framed.sty), Duncan Murdoch, Fabian Hirschmann, Fitch Simeon, Frank E Harrell Jr (the Sweavel package at inst/misc/Sweavel.sty), Gregoire Detrez, Hadley Wickham, Heewon Jeon, Henrik Bengtsson, Jake Burkhead, James Manton, Jared Lander, Jeff Arnold, Jeremy Ashkenas (the CSS file at inst/misc/docco-classic.css), Jeremy Stephens, Jim Hester, Joe Cheng, Johannes Ranke, John Honaker, Jonathan Keane, JJ Allaire, Johan Toloe, Joseph Larmarange, Julien Barnier, Kevin K. Smith, Kirill Mueller, Kohske Takahashi, Michael Friendly, Michel Kuhlmann, Nacho Caballero, Nick Salkowski, Noam Ross, Qiang Li, Ramnath Vaidyanathan, Richard Cotton, Romain Francois, Scott Kostyshak, Sietse Brouwer, Simon de Bernard, Taiyun Wei, Thibaut Assus, Thibaut Lamadon, Thomas Leeper, Tom Torsney-Weir, Trevor Davis, Viktoras Veitas, Weicheng Zhu, Wush Wu, Yihui Xie


  \item [Lahman] \emph{Sean Lahman's Baseball Database} (v. 3.0-1).
	This package provides the tables from Sean Lahman's
        Baseball Database as a set of R data.frames.  It uses the data
        on pitching, hitting and fielding performance and other tables
        from 1871 through 2013, as recorded in the 2014 version of the
        database.
	\textbf{Authors}: Michael Friendly, Dennis Murphy, Martin Monkman, Chris Dalzell


  \item [lattice] \emph{Lattice Graphics} (v. 0.20-31).
	Lattice is a powerful and elegant high-level data
  visualization system, with an emphasis on multivariate data, that is
  sufficient for typical graphics needs, and is also flexible enough
  to handle most nonstandard requirements.  See ?Lattice for an
  introduction.
	\textbf{Authors}: Deepayan Sarkar \url{email:deepayan.sarkar@r-project.org}


  \item [lme4] \emph{Linear mixed-effects models using Eigen and S4} (v. 1.1-7).
	Fit linear and generalized linear mixed-effects models.
    The models and their components are represented using S4 classes and
    methods.  The core computational algorithms are implemented using the
    Eigen C++ library for numerical linear algebra and RcppEigen "glue".
	\textbf{Authors}: Douglas Bates, Martin Maechler, Ben Bolker, Steven Walker, Rune Haubo Bojesen Christensen, Henrik Singmann, Bin Dai


  \item [lmtest] \emph{Testing Linear Regression Models} (v. 0.9-33).
	A collection of tests, data sets, and examples
 for diagnostic checking in linear regression models. Furthermore,
 some generic tools for inference in parametric models are provided.
	\textbf{Authors}: Torsten Hothorn, Achim Zeileis, Richard W. Farebrother (pan.f), Clint Cummins (pan.f), Giovanni Millo, David Mitchell


  \item [logmult] \emph{Log-Multiplicative Models, Including Association Models} (v. 0.6.1).
	Functions to fit log-multiplicative models using gnm, with
  support for convenient printing, plots, and jackknife/bootstrap
  standard errors. For complex survey data, models can be fitted from
  design objects from the 'survey' package. Currently supported models
  include UNIDIFF (Erikson & Goldthorpe), a.k.a. log-multiplicative
  layer effect model (Xie), and several association models: Goodman's
  row-column association models of the RC(M) and RC(M)-L families
  with one or several dimensions; two skew-symmetric association
  models proposed by Yamaguchi and by van der Heijden & Mooijaart.
	\textbf{Authors}: Milan Bouchet-Valat


  \item [lsmeans] \emph{Least-Squares Means} (v. 2.16).
	Obtain least-squares means for many linear, generalized linear, and mixed models. Compute contrasts or linear functions of least-squares means, and comparisons of slopes. Plots and compact letter displays.
	\textbf{Authors}: Russell V. Lenth, Maxime Hervé


  \item [manipulate] \emph{Interactive Plots for RStudio} (v. 0.98.977).
	Interactive Plots for RStudio
	\textbf{Authors}: RStudio


  \item [MASS] \emph{Support Functions and Datasets for Venables and Ripley's MASS} (v. 7.3-40).
	Functions and datasets to support Venables and Ripley,
  'Modern Applied Statistics with S' (4th edition, 2002).
	\textbf{Authors}: Brian Ripley, Bill Venables, Douglas M. Bates, Kurt Hornik (partial port ca 1998), Albrecht Gebhardt (partial port ca 1998), David Firth


  \item [mgcv] \emph{Mixed GAM Computation Vehicle with GCV/AIC/REML Smoothness
Estimation} (v. 1.8-6).
	GAMs, GAMMs and other generalized ridge regression with 
             multiple smoothing parameter estimation by GCV, REML or UBRE/AIC. 
             Includes a gam() function, a wide variety of smoothers, JAGS 
             support and distributions beyond the exponential family.
	\textbf{Authors}: Simon Wood \url{email:simon.wood@r-project.org}


  \item [mlogit] \emph{multinomial logit model} (v. 0.2-4).
	Estimation of the multinomial logit model
	\textbf{Authors}: Yves Croissant


  \item [nlme] \emph{Linear and Nonlinear Mixed Effects Models} (v. 3.1-120).
	Fit and compare Gaussian linear and nonlinear mixed-effects models.
	\textbf{Authors}: Jos� Pinheiro (S version), Douglas Bates (up to 2007), Saikat DebRoy (up to 2002), Deepayan Sarkar (up to 2005), EISPACK authors (src/rs.f), R-core


  \item [nnet] \emph{Feed-Forward Neural Networks and Multinomial Log-Linear Models} (v. 7.3-9).
	Software for feed-forward neural networks with a single
  hidden layer, and for multinomial log-linear models.
	\textbf{Authors}: Brian Ripley, William Venables


  \item [plyr] \emph{Tools for splitting, applying and combining data} (v. 1.8.1).
	plyr is a set of tools that solves a common
    set of problems: you need to break a big problem down
    into manageable pieces, operate on each pieces and then
    put all the pieces back together.  For example, you
    might want to fit a model to each spatial location or
    time point in your study, summarise data by panels or
    collapse high-dimensional arrays to simpler summary
    statistics. The development of plyr has been generously
    supported by BD (Becton Dickinson).
	\textbf{Authors}: Hadley Wickham \url{email:h.wickham@gmail.com}


  \item [poLCA] \emph{Polytomous variable Latent Class Analysis} (v. 1.4.1).
	Latent class analysis and latent class regression models 
    for polytomous outcome variables.  Also known as latent structure analysis.
	\textbf{Authors}: Drew Linzer \url{email:drew@votamatic.org}, Jeffrey Lewis \url{email:jblewis@ucla.edu}.


  \item [popbio] \emph{Construction and analysis of matrix population models} (v. 2.4).
	Construct and analyze projection matrix models from a
        demography study of marked individuals classified by age or
        stage. The package covers methods described in Matrix
        Population Models by Caswell (2001) and Quantitative
        Conservation Biology by Morris and Doak (2002).
	\textbf{Authors}: Chris Stubben, Brook Milligan, Patrick Nantel


  \item [pscl] \emph{Political Science Computational Laboratory, Stanford University} (v. 1.4.9).
	Bayesian analysis of item-response theory (IRT) models,
	     roll call analysis; computing highest density regions; maximum
	     likelihood estimation of zero-inflated and hurdle models for count
	     data; goodness-of-fit measures for GLMs; data sets used
	     in writing	and teaching at the Political Science
	     Computational Laboratory; seats-votes curves.
	\textbf{Authors}: Simon Jackman, with contributions from Alex Tahk, Achim Zeileis, Christina Maimone and Jim Fearon


  \item [psych] \emph{Procedures for Psychological, Psychometric, and Personality
Research} (v. 1.5.1).
	A general purpose toolbox for personality, psychometrics and experimental psychology.   Functions are primarily for scale construction using factor analysis, principal component analysis, cluster analysis and reliability analysis, although others provide basic descriptive statistics. Item Response Theory is done using  factor analysis of tetrachoric and polychoric correlations. Functions for analyzing data at multi-levels include within and between group statistics, including correlations and factor analysis.   Functions for simulating particular item and test structures are included. Several functions serve as a useful front end for structural equation modeling.  Graphical displays of path diagrams, factor analysis and structural equation models are created using basic graphics. Some of the functions are written to support a book on psychometrics as well as publications in personality research. For more information, see the personality-project.org/r webpage.
	\textbf{Authors}: William Revelle \url{email:revelle@northwestern.edu}


  \item [rCharts] \emph{Interactive Charts using Javascript Visualization Libraries} (v. 0.4.5).
	Create interactive visualizations with a plotting interface
    familiar to R users using several javascript visualization libraries.
	\textbf{Authors}: Thomas Reinholdsson, Kenton Russell, Ramnath Vaidyanathan


  \item [reshape2] \emph{Flexibly Reshape Data: A Reboot of the Reshape Package.} (v. 1.4.1).
	Flexibly restructure and aggregate data using just two
    functions: melt and dcast (or acast).
	\textbf{Authors}: Hadley Wickham \url{email:h.wickham@gmail.com}


  \item [rggobi] \emph{Interface between R and GGobi} (v. 2.1.20).
	The rggobi package provides a command-line interface to GGobi, an
    interactive and dynamic graphics package. Rggobi complements GGobi's
    graphical user interface, providing a way to fluidly transition between
    analysis and exploration, as well as automating common tasks.
	\textbf{Authors}: Duncan Temple Lang \url{email:duncan@research.bell-labs.com}, Debby Swayne \url{email:dfs@research.att.com}, Hadley Wickham \url{email:h.wickham@gmail.com}, Michael Lawrence \url{email:michafla@gene.com}


  \item [rgl] \emph{3D visualization device system (OpenGL)} (v. 0.95.1201).
	Provides medium to high level functions for 3D interactive graphics, including
    functions modelled on base graphics (plot3d(), etc.) as well as functions for 
    constructing representations of geometric objects (cube3d(), etc.).  Output
    may be on screen using OpenGL, or to various standard 3D file formats including 
    WebGL, PLY, OBJ, STL as well as 2D image formats, including PNG, Postscript, SVG, PGF.
	\textbf{Authors}: Daniel Adler \url{email:dadler@uni-goettingen.de}, Duncan Murdoch \url{email:murdoch@stats.uwo.ca}, and others (see README)


  \item [rms] \emph{Regression Modeling Strategies} (v. 4.3-0).
	Regression modeling, testing, estimation, validation,
	graphics, prediction, and typesetting by storing enhanced model design
	attributes in the fit.  rms is a collection of functions that
	assist with and streamline modeling.  It also contains functions for
	binary and ordinal logistic regression models, ordinal models for
  continuous Y with a variety of distribution families, and the Buckley-James
	multiple regression model for right-censored responses, and implements
	penalized maximum likelihood estimation for logistic and ordinary
	linear models.  rms works with almost any regression model, but it
	was especially written to work with binary or ordinal regression
	models, Cox regression, accelerated failure time models,
	ordinary linear models,	the Buckley-James model, generalized least
	squares for serially or spatially correlated observations, generalized
	linear models, and quantile regression.
	\textbf{Authors}: Frank E Harrell Jr \url{email:f.harrell@vanderbilt.edu}


  \item [rsm] \emph{Response-surface analysis} (v. 2.07).
	Provides functions to generate response-surface designs, fit first- and second-order response-surface models, make surface plots, obtain the path of steepest ascent, and do canonical analysis.
	\textbf{Authors}: Russell V. Lenth


  \item [sandwich] \emph{Robust Covariance Matrix Estimators} (v. 2.3-3).
	Model-robust standard error estimators for cross-sectional, time series, and longitudinal data.
	\textbf{Authors}: Thomas Lumley, Achim Zeileis


  \item [shiny] \emph{Web Application Framework for R} (v. 0.11.1).
	Shiny makes it incredibly easy to build interactive web
    applications with R. Automatic "reactive" binding between inputs and
    outputs and extensive pre-built widgets make it possible to build
    beautiful, responsive, and powerful applications with minimal effort.
	\textbf{Authors}: Winston Chang, Joe Cheng, JJ Allaire, Yihui Xie, Jonathan McPherson, RStudio, jQuery Foundation (jQuery library and jQuery UI library), jQuery contributors (jQuery library; authors listed in inst/www/shared/jquery-AUTHORS.txt), jQuery UI contributors (jQuery UI library; authors listed in inst/www/shared/jqueryui/1.10.4/AUTHORS.txt), Mark Otto (Bootstrap library), Jacob Thornton (Bootstrap library), Bootstrap contributors (Bootstrap library), Twitter, Inc (Bootstrap library), Alexander Farkas (html5shiv library), Scott Jehl (Respond.js library), Stefan Petre (Bootstrap-datepicker library), Andrew Rowls (Bootstrap-datepicker library), Dave Gandy (Font-Awesome font), Brian Reavis (selectize.js library), Kristopher Michael Kowal (es5-shim library), es5-shim contributors (es5-shim library), Denis Ineshin (ion.rangeSlider library), SpryMedia Limited (DataTables library), John Fraser (showdown.js library), John Gruber (showdown.js library), Ivan Sagalaev (highlight.js library), R Core Team (tar implementation from R)


  \item [Sleuth2] \emph{Data sets from Ramsey and Schafer's "Statistical Sleuth (2nd
ed)"} (v. 1.0-7).
	Data sets from Ramsey, F.L. and Schafer, D.W. (2002), "The
        Statistical Sleuth: A Course in Methods of Data Analysis (2nd
        ed)", Duxbury.
	\textbf{Authors}: Original by F.L. Ramsey and D.W. Schafer, modifications by Daniel W. Schafer, Jeannie Sifneos and Berwin A. Turlach


  \item [TeachingDemos] \emph{Demonstrations for teaching and learning} (v. 2.9).
	This package is a set of demonstration functions that can
        be used in a classroom to demonstrate statistical concepts, or
        on your own to better understand the concepts or the
        programming.
	\textbf{Authors}: Greg Snow


  \item [texreg] \emph{Conversion of R regression output to LaTeX or HTML tables.} (v. 1.34).
	texreg converts coefficients, standard errors, significance stars, and goodness-of-fit statistics of statistical models into LaTeX tables or HTML tables/MS Word documents or to nicely formatted screen output for the R console for easy model comparison. A list of several models can be combined in a single table. The output is highly customizable. New model types can be easily implemented.
	\textbf{Authors}: Philip Leifeld


  \item [UBbipl] \emph{UNDERSTANDING BIPLOTS: DATA SETS AND FUNCTIONS} (v. 3.0.4).
	Package consisting of all data sets and functions discussed in Understanding Biplots.
	\textbf{Authors}: Niel le Roux and Sugnet Lubbe


  \item [vcd] \emph{Visualizing Categorical Data} (v. 1.3-3).
	Visualization techniques, data sets, summary and inference
        procedures aimed particularly at categorical data. Special
        emphasis is given to highly extensible grid graphics. The
        package was inspired by the book "Visualizing Categorical Data"
        by Michael Friendly.
	\textbf{Authors}: David Meyer, Achim Zeileis, Kurt Hornik, Florian Gerber, Michael Friendly


  \item [vcdExtra] \emph{vcd Extensions and Additions} (v. 0.6-7).
	Provides additional data sets, methods and documentation to complement the vcd package for Visualizing Categorical Data
    and the gnm package for Generalized Nonlinear Models.
	In particular, vcdExtra extends mosaic, assoc and sieve plots from vcd to handle glm() and gnm() models and
	adds a 3D version in mosaic3d.  Additionally, methods are provided for comparing and visualizing lists of
	glm and loglm objects.
	\textbf{Authors}: Michael Friendly, Heather Turner, Achim Zeileis, Duncan Murdoch, David Firth


  \item [VGAM] \emph{Vector Generalized Linear and Additive Models} (v. 0.9-7).
	An implementation of about 6 major classes of
    statistical regression models. At the heart of it are the
    vector generalized linear and additive model (VGLM/VGAM)
    classes.  Many (150+) models and distributions are estimated
    by maximum likelihood estimation (MLE) or penalized MLE, using
    Fisher scoring. VGLMs can be loosely thought of as multivariate
    GLMs. VGAMs are data-driven VGLMs (i.e., with smoothing). The
    other classes are RR-VGLMs (reduced-rank VGLMs), quadratic
    RR-VGLMs, reduced-rank VGAMs, RCIMs (row-column interaction
    models)---these classes perform constrained and unconstrained
    quadratic ordination (CQO/UQO) models in ecology, as well
    as constrained additive ordination (CAO). Note that these
    functions are subject to change, especially before version
    1.0.0 is released; see the NEWS file for latest changes.
	\textbf{Authors}: Thomas W. Yee \url{email:t.yee@auckland.ac.nz}


  \item [XLConnect] \emph{Excel Connector for R} (v. 0.2-11).
	Provides comprehensive functionality to read, write and format Excel data.
	\textbf{Authors}: Mirai Solutions GmbH, Martin Studer, The Apache Software Foundation (Apache POI, Apache Commons Codec), Stephen Colebourne (Joda-Time Java library)


  \item [xlsx] \emph{Read, write, format Excel 2007 and Excel 97/2000/XP/2003 files} (v. 0.5.7).
	Provide R functions to read/write/format Excel 2007 and Excel 97/2000/XP/2003 file formats.
	\textbf{Authors}: Adrian A. Dragulescu


  \item [xtable] \emph{Export tables to LaTeX or HTML} (v. 1.7-4).
	Coerce data to LaTeX and HTML tables
	\textbf{Authors}: David B. Dahl \url{email:dahl@stat.byu.edu}


\end{description}

