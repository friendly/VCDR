\chapter*{Preface}
%\addcontentsline{toc}{chapter}{\numberline{}Preface}

%\TODO{The preface has not yet been written.  This is just a stub.}

\epigraph{The greatest value of a picture is when it forces us to notice what we never expected to see.}{\citet[p. vi]{Tukey:77}}

\section*{Data analysis and graphics}

This book stems from the conviction that data analysis and statistical graphics should
go hand-in-hand in the process of understanding and communicating statistical data.
Statistical summaries compress a data set into a few numbers, the result of an
hypothesis test, or coefficients in a fitted statistical model,
while graphical methods help to explore patterns and trends, see the unexpected,
identify problems in an analysis and communicate results and conclusions in 
principled and effective ways. 

This interplay between analysis and visualization has long been a part of 
statistical practice for \emph{quantitative data}.  Indeed, the origin
of correlation, regression and linear models (regression, ANOVA) can
arguably be traced to Francis Galton's \citeyearpar{Galton:1886}
visual insight from a scatterplot of heights of children and their parents
on which he overlaid smoothed contour curves of roughly equal bivariate frequencies
and lines for the means of $Y \given X$ and $X \given Y$
(described in \citet{FriendlyDenis:05:scat,Friendly-etal:ellipses:2013}).

The analysis of discrete data is a much more recent arrival, beginning in the
1960s and giving rise to a few seminal books in the 1970s
\citep{Bishop-etal:75,Haberman:74, Goodman:1978,Fienberg:80}.
\citet[\C 17]{Agresti:2013} presents a brief historical overview of the
development of these methods from their early roots around the beginning
of the 20$^{th}$ Century.

Yet curiously, associated graphical methods for categorical data were 
much slower to develop. This began to change as it was recognized that
counts, frequencies and discrete variables required different
schemes for mapping numbers into useful visual representations \citep{Friendly:95,Friendly:97},
some quite novel.
The special nature of discrete variables
and frequency data vis-a-vis statistical graphics is now more widely accepted,
and many of these new graphical methods (e.g., mosaic displays, fourfold plots, diagnostic
plots for generalized linear models) have become, if not main stream, then at
least more widely used in research, teaching and communication.

Much of what had been developed through the 1990s for graphical methods for
discrete data was described in the book \emph{Visualizing Categorical Data}
\citep{Friendly:00:VCD} and was implemented in SAS\textsuperscript{\textregistered} software. 
Since that time,  
there has been considerable growth in both statistical methods for the
analysis of categorical data (e.g., generalized linear models, zero-inflation
models, mixed models for hierarchical and longitudinal data with discrete
outcomes), along with some new graphical methods for visualizing and
interpreting the results (3D mosaic plots, effect plots, diagnostic plots, etc.) 
The bulk of these developments have been implemented in \R, and the time is
right for an in-depth treatment of modern graphical methods for the analysis
of categorical data, to which you are now invited.

\section*{Goals}

\section*{Overview and organization of this book}
%\section*{Overview and organization of this book}

This book is divided into three parts. Part I, Chapters 1--3, contains 
introductory material on graphical methods for discrete data, 
basic \R skills needed for the book, and methods for fitting
and visualizing one-way discrete distributions.

Part II, Chapters 4--6, is concerned largely with 
simple, traditional non-parametric tests and exploratory methods for
visualizing patterns of association in two-way and larger frequency tables.
Some of the discussion here introduces ideas and notation for \loglin
models that are treated more generally in Part III.

Part III, Chapters 7--11, discusses model-based methods for the
analysis of discrete data.  These are all examples of generalized
linear models.  However, for our purposes, it has proved more convenient
to develop this topic from the specific cases (logistic regression, \loglin models)
to the general rather than the reverse.

\begin{description}
\item[\chref{ch:intro}: \emph{Introduction}.]
Categorical data require different statistical and graphical methods
than commonly used for quantitative data.
This chapter outlines the basic orientation
of the book toward visualization methods and some key distinctions regarding the
analysis and visualization of categorical data.

\item[\chref{ch:working}: \emph{Working with Categorical Data}.]
Categorical data can be represented in various forms:
case form, frequency form, and table form.  This chapter
describes and illustrates the skills and techniques in \R
needed to input, create, and manipulate \R data objects
to represent categorical data, and convert these from one
form to another for the purposes of statistical analysis
and visualization, which are the subject of the remainder of the book.

\item[\chref{ch:discrete}: \emph{Fitting and Graphing Discrete Distributions}.]
Understanding and visualizing discrete data distributions provides a
building block for model-based methods discussed in Part III.
This chapter introduces the well-known discrete distributions---
the binomial, Poisson, negative-binomial, and others---
in the simplest case of a one-way frequency table.

\item[\chref{ch:twoway}: \emph{Two-Way Contingency Tables}.]
The analysis of two-way frequency tables concerns the association
between two variables.  A variety of specialized graphical
displays help to visualize the pattern of association,
using area of some region to represent the frequency in a cell.
Some of these methods are focused
on visualizing an odds ratio (for $2 \times 2$ tables), or the general
pattern of association, or the agreement between row and column
categories in square tables.

\item[\chref{ch:mosaic}: \emph{Mosaic Displays for $n$-Way Tables}.]
This chapter introduces mosaic displays, designed to
help to visualize the pattern of associations
among variables in two-way and larger tables.  
Extensions of
this technique can reveal partial associations and marginal associations,
and shed light on the structure of \loglin\ models themselves.

\item[\chref{ch:corresp}: \emph{Correspondence Analysis}.]
Correspondence analysis provides visualizations of associations in a two-way \ctab
in a small number of dimensions.
Multiple correspondence analysis extends this technique to \nway
tables.  Other graphical methods, including mosaic matrices and biplots,
provide complementary views of \loglin models for two-way and \nway
\ctabs.

\item[\chref{ch:logistic}: \emph{Logistic Regression Models}.]
This chapter introduces the modeling framework for categorical data in the simple
situation where we have a categorical response variable, often binary, and one or
more explanatory variables. A fitted model provides both statistical
inference and prediction, accompanied by measures of uncertainty.
Data visualization methods for discrete response data must often rely
on smoothing techniques, including both direct, non-parametric smoothing
and the implicit smoothing that results from a fitted parametric model.
Diagnostic plots help us to detect influential observations that may distort
our results.

\item[\chref{ch:polytomous}: \emph{Models for Polytomous Responses}.]
This chapter generalizes logistic regression models for a binary response to
handle a multi-category (polytomous) response.  Different models are available depending on
whether the response categories are nominal or ordinal.
Visualization methods for such models are mostly straightforward extensions
of those used for binary responses presented in \chref{ch:logistic}.

\item[\chref{ch:loglin}: \emph{Loglinear and Logit Models for Contingency Tables}.]
This chapter extends the model-building approach to \loglin and logit
models. These comprise another special case of generalized linear models
designed for \ctabs of frequencies.  They
are most easily interpreted through
visualizations, including mosaic displays and effect plots of associated
logit models.  

\item[\chref{ch:loglin2}: \emph{Extending Loglinear Models}.]
Loglinear models have special forms to represent additional structure in the
variables in contingency tables.  Models for ordinal factors allow a more
parsimonious description of associations.  Models for square tables
allow a wide range of specific models for the relationship between
variables with the same categories.  Another extended class of
models arise when there are two or more response variables.


\item[\chref{ch:glm}: \emph{Generalized Linear Models}.]
Generalized linear models extend the familiar linear models of
regression and ANOVA to
include counted data, frequencies, and other data for which the
assumptions of independent, normal errors are not reasonable.
We rely on the analogies between ordinary and generalized linear
models (GLMs) to develop visualization methods to explore the data,
display the fitted relationships, and check model assumptions.
The main focus of this chapter is on models for count data.

\end{description}







 
\section*{Audience}
This book has been written to appeal to two broad audiences wishing to learn to apply methods for
discrete data analysis:
% It is written to appeal to two audiences:
 \begin{itemize*}
   \item Advanced undergraduate, graduate students in the social and health sciences, epidemiology,
     economics, business and (bio)statistics
 	\item Substantive researchers, methodologists and consultants in various disciplines wanting to be able to use
 	 these methods with their own data and analyses.
 \end{itemize*}

It assumes the reader has a basic understanding of statistical concepts at least at an
intermediate undergraduate level including regression and analysis of variance
(for example, at the level of \citet{Neter-etal:90,MendenhallSincich:2003}).
It is less technically demanding than other modern texts covering
cagtegorical data analysis at a graduate level, such as
 \citet{Agresti:2013}, \emph{Categorical Data Analysis},
 \citet{PowersXie:2008}, \emph{Statistical Methods for Categorical Data Analysis}, and
 \citet{Christensen:97}, \emph{Log-Linear Models and Logistic Regression}.
Nevertheless, there are some topics that are a bit more advanced or technical, and
these are marked as \hard or \veryhard sections.

In addition, it is not practical to include all details of using \R effectively for
data analysis. It is assumed that the reader has at least basic knowledge of the \R language and
environment, including interacting with the \R console (RGui for Windows, R.app for Mac OS X)
or other graphical user interface (e.g., RStudio), using \R functions in packages,
getting help for these from \R, etc.  One introductory chapter (\chref{ch:working}) is devoted
to covering the particular topics most important to categorical data analysis,
beyond such basic skills needed in the book.

\section*{Textbook use}
This book is most directly suitable for one-semester applied 
advanced undergraduate or graduate
course on categorical data analysis with a strong emphasis
on the use of graphical methods to understand and explain data and
results of analysis.
A detailed outline of such a course, together with lecture notes
and assignments,
is available at the first author's
web page, \url{http://euclid.psych.yorku.ca/www/psy6136/}, using this
book as the main text.  This course also uses 
\citet{Agresti:2007:ICDA}, \emph{An Introduction to Categorical Data Analysis}
for additional readings.

For instructors teaching a more traditional course using one of the books
mentioned above as the main text, this book would be a welcomed supplement,
because almost all other texts treat graphical methods only perfunctorily,
if at all.
A few of these contain a brief appendix mentioning software, or have
a related web site with some data sets and software examples.
Moreover, none actually describe how to do these analyses and graphics with \R.


\section*{Acknowledgements}

