\chapter{Preface}
%\addcontentsline{toc}{chapter}{\numberline{}Preface}



\section*{Audience}
This book assumes basic understanding of statistical concepts at least at an
intermediate undergraduate level including regression and analysis of variance
(for example, at the level of \citet{Neter-etal:90,MendenhallSincich:2003}).

It is written to appeal to two audiences:
\begin{itemize*}
  \item Students and methodologists in the social and health sciences, epidemiology,
    economics, business
	and (bio)statistics
	\item Substantive researchers in various disciplines wanting to be able to
	apply these methods to their own data
\end{itemize*}

It is also assumed that the reader has at least basic knowledge of the \R language and
environment, including interacting with the \R console (RGui for Windows, R.app for Mac OS X)
or other graphical user interface (e.g., R Studio), using \R functions in packages,
getting help for these from \R, etc.  One introductory chapter (\chref{ch:working}) is devoted
to covering topics beyond such basic skills needed in the book.

\section*{Overview}

\section*{Acknowledgements}

